\pdfobjcompresslevel 0
\documentclass[a4paper, 11pt, twocolumn, oneside, openright]{report}

%====================== PACKAGES ======================
\usepackage{amsfonts}
\usepackage{amsthm}
\usepackage{float}
\usepackage[top=1.5cm,bottom=2cm,left=2.5cm,right=2.5cm,includehead]{geometry}
\usepackage{tikz-cd}
\usetikzlibrary{calc}
\usepackage{circuitikz}
\usepackage{makecell}
\usepackage{amsmath}
\usepackage{amssymb}
\usepackage[french]{babel}
\usepackage[T1]{fontenc}
\usepackage[utf8]{inputenc}
\usepackage[]{mdframed}
\usepackage{enumitem}
\usepackage{graphicx}
\usepackage[colorlinks=true, allcolors=magenta]{hyperref}
\usepackage[skip=10pt plus1pt, indent=20pt]{parskip}

\usepackage{colortbl}

\DeclareMathOperator{\IN}{\mathbb{N}}
\DeclareMathOperator{\IZ}{\mathbb{Z}}
\DeclareMathOperator{\IQ}{\mathbb{Q}}
\DeclareMathOperator{\IR}{\mathbb{R}}
\DeclareMathOperator{\IK}{\mathbb{K}}
\DeclareMathOperator{\IC}{\mathbb{C}}
\DeclareMathOperator{\IF}{\mathbb{F}}
\DeclareMathOperator{\IP}{\mathbb{P}}

\DeclareMathOperator{\Vect}{Vect}
\DeclareMathOperator{\im}{im}
\DeclareMathOperator{\rank}{rank}
\DeclareMathOperator{\codim}{codim}
\DeclareMathOperator{\corank}{corank}
\DeclareMathOperator{\codomain}{codomain}
\DeclareMathOperator*{\argmin}{arg\,min}
\DeclareMathOperator*{\card}{Card}

\theoremstyle{definition}
\newtheorem{theorem}{Théorème}[section]
\newtheorem{corollary}{Corollaire}[theorem]
\newtheorem{lemma}[theorem]{Lemme}
\newtheorem*{remark}{Remarque}
\newtheorem*{example}{Exemple}
\newtheorem*{application}{Application}
\newtheorem*{objectif}{Objectif}

\newtheorem{definition}{Définition}[section]
\newtheorem{notation}{Notation}[section]
\newtheorem{proposition}{Proposition}[section]


\begin{document}

\title{Cartographie des mathématiques pour l'agrégation}
\author{Gautier Laisné}
\date{}
\maketitle

\begingroup
\hypersetup{linkcolor=black}
\tableofcontents
\endgroup

\part{Théorie des nombres}
\chapter{Arithmétique : La Structure Cachée des Nombres Entiers}

\section{Division Euclidienne et PGCD : L'Algorithme Fondateur}

\begin{objectif}
    Établir le socle de toute l'arithmétique. La division euclidienne est une propriété d'apparence modeste, mais elle est le moteur de tout le reste. D'elle découlent les notions de PGCD, les relations de Bézout et de Gauss, qui sont les outils de base pour explorer la structure multiplicative de $\mathbb{Z}$.
\end{objectif}

\begin{theorem}[Division Euclidienne dans $\mathbb{Z}$]
    Pour tout $(a,b) \in \mathbb{Z} \times \mathbb{Z}^*$, il existe un unique couple $(q,r) \in \mathbb{Z}^2$ (quotient, reste) tel que $a = bq + r$ et $0 \leq r < |b|$.
\end{theorem}

\begin{remark}[Le Point de Départ de la Théorie des Anneaux]
    Cette propriété fait de l'anneau $\mathbb{Z}$ un \textbf{anneau euclidien}. C'est la propriété la plus forte dans la hiérarchie des anneaux (Euclidien $\implies$ Principal $\implies$ Factoriel), ce qui explique pourquoi l'arithmétique dans $\mathbb{Z}$ est si "bien comportée".
\end{remark}

\begin{definition}[PGCD et Algorithme d'Euclide]
    Le Plus Grand Commun Diviseur (PGCD) de $a$ et $b$ est le plus grand entier positif qui divise à la fois $a$ et $b$.
    L'\textbf{algorithme d'Euclide} est une procédure itérative qui calcule le PGCD de deux nombres en utilisant des divisions euclidiennes successives. C'est l'un des plus anciens algorithmes non triviaux de l'histoire.
\end{definition}

\begin{theorem}[Théorème de Bézout]
    Soient $a, b \in \mathbb{Z}$. L'ensemble des combinaisons linéaires $\{ax+by \mid x,y \in \mathbb{Z}\}$ est exactement l'ensemble des multiples de leur PGCD, $\mathrm{pgcd}(a,b)\mathbb{Z}$.
    En particulier, $a$ et $b$ sont premiers entre eux si et seulement s'il existe $u,v \in \mathbb{Z}$ tels que $au+bv=1$.
\end{theorem}

\begin{application}[Inverse modulaire]
    L'identité de Bézout est constructive (grâce à l'algorithme d'Euclide étendu). Elle permet de calculer l'inverse d'un élément dans l'anneau $\mathbb{Z}/n\mathbb{Z}$. L'inverse de $a$ modulo $n$ existe si $\mathrm{pgcd}(a,n)=1$, et l'algorithme nous donne $u,v$ tels que $au+nv=1$, ce qui signifie $au \equiv 1 \pmod n$. L'inverse de $a$ est $u$.
\end{application}

\begin{theorem}[Lemme de Gauss]
    Si un entier $a$ divise le produit $bc$ et que $a$ est premier avec $b$, alors $a$ divise $c$.
\end{theorem}

\section{Nombres Premiers et Factorisation Unique}

\begin{objectif}
    Introduire les "atomes" de l'arithmétique : les nombres premiers. Le résultat central de cette section, le Théorème Fondamental de l'Arithmétique, est la pierre angulaire de la théorie des nombres, affirmant que tout entier se décompose de manière unique en un "assemblage" de ces atomes.
\end{objectif}

\begin{definition}[Nombre Premier]
    Un entier $p > 1$ est \textbf{premier} s'il n'admet que deux diviseurs positifs : 1 et lui-même.
\end{definition}

\begin{lemma}[Lemme d'Euclide]
    Un entier $p > 1$ est premier si et seulement si pour tous entiers $a,b$, on a : $p|ab \implies p|a$ ou $p|b$.
\end{lemma}
\begin{remark}[La Vraie Définition de la Primalité]
    Cette propriété est conceptuellement plus forte que la définition par les diviseurs. C'est elle qui garantit l'unicité de la décomposition. Dans les anneaux plus généraux, c'est cette propriété qui définit un "élément premier".
\end{remark}

\begin{theorem}[Théorème Fondamental de l'Arithmétique]
    Tout entier $n \ge 2$ se décompose de manière unique (à l'ordre près des facteurs) en un produit de nombres premiers.
\end{theorem}

\begin{theorem}[Théorème d'Euclide sur l'infinité des nombres premiers]
    Il existe une infinité de nombres premiers. La preuve classique par l'absurde (considérer $P+1$ où $P$ est le produit des premiers) est un modèle d'élégance mathématique.
\end{theorem}

\begin{theorem}[Théorème des Nombres Premiers (admis)]
    Soit $\pi(x)$ le nombre de nombres premiers inférieurs ou égaux à $x$. Alors, on a l'équivalent asymptotique :
    $$ \pi(x) \underset{x\to\infty}{\sim} \frac{x}{\ln(x)} $$
\end{theorem}
\begin{remark}[Le Pont entre Arithmétique et Analyse]
    Ce théorème est un résultat profond qui illustre que pour comprendre la distribution (très irrégulière) des nombres premiers, des outils d'analyse (en l'occurrence, l'analyse complexe et la fonction $\zeta$ de Riemann) sont nécessaires.
\end{remark}

\section{Arithmétique Modulaire : Les Mondes Finis de $\mathbb{Z}/n\mathbb{Z}$}

\begin{objectif}
    Étudier la structure de l'anneau $\mathbb{Z}/n\mathbb{Z}$. C'est le monde de l'"arithmétique de l'horloge", où l'on s'intéresse aux restes de la division. Ces structures finies sont d'une richesse incroyable et sont au cœur de la cryptographie moderne.
\end{objectif}

\begin{definition}[Anneau des entiers modulo $n$]
    La relation de congruence modulo $n$ est une relation d'équivalence sur $\mathbb{Z}$. L'ensemble des classes d'équivalence, noté $\mathbb{Z}/n\mathbb{Z}$, est muni d'une structure d'anneau commutatif unitaire.
\end{definition}

\begin{proposition}[Groupe des inversibles]
    Un élément $\bar{k} \in \mathbb{Z}/n\mathbb{Z}$ est inversible si et seulement si $\mathrm{pgcd}(k,n)=1$. L'ensemble des inversibles, noté $(\mathbb{Z}/n\mathbb{Z})^\times$, forme un groupe multiplicatif d'ordre $\varphi(n)$, où $\varphi$ est l'indicateur d'Euler.
\end{proposition}

\begin{corollary}
    $\mathbb{Z}/n\mathbb{Z}$ est un corps si et seulement si $n$ est un nombre premier. Ce corps est noté $\mathbb{F}_n$.
\end{corollary}

\begin{theorem}[Théorème d'Euler]
    Soit $n \ge 2$. Pour tout entier $a$ premier avec $n$, on a $a^{\varphi(n)} \equiv 1 \pmod n$.
    En particulier (Petit Théorème de Fermat), si $p$ est premier et $p \nmid a$, alors $a^{p-1} \equiv 1 \pmod p$.
\end{theorem}

\begin{theorem}[Théorème des Restes Chinois]
    Si $m$ et $n$ sont premiers entre eux, alors l'application naturelle de $\mathbb{Z}/mn\mathbb{Z}$ vers $\mathbb{Z}/m\mathbb{Z} \times \mathbb{Z}/n\mathbb{Z}$ est un isomorphisme d'anneaux.
\end{theorem}

\begin{application}[Cryptosystème RSA]
    Le système RSA est basé sur ces résultats. On choisit deux grands nombres premiers $p,q$ et on pose $n=pq$. $\varphi(n)=(p-1)(q-1)$ est facile à calculer si on connaît $p,q$, mais très difficile sinon. On choisit un exposant de chiffrement $e$ premier avec $\varphi(n)$ et on calcule son inverse $d$ modulo $\varphi(n)$ (grâce à Bézout). Le message $M$ est chiffré par $C = M^e \pmod n$. Le déchiffrement se fait par $C^d = (M^e)^d = M^{ed} = M^{1+k\varphi(n)} \equiv M \pmod n$ par le théorème d'Euler.
\end{application}

\section{Résidus Quadratiques et la Loi de Réciprocité}

\begin{objectif}
    Aborder un des plus beaux résultats de la théorie des nombres, qualifié par Gauss de "Théorème d'Or". La loi de réciprocité quadratique révèle une symétrie cachée et profonde entre les nombres premiers.
\end{objectif}

\begin{definition}[Résidu Quadratique et Symbole de Legendre]
    Soit $p$ un premier impair. Un entier $a$ est un \textbf{résidu quadratique} modulo $p$ s'il est un carré non nul modulo $p$.
    Le \textbf{symbole de Legendre} $\left(\frac{a}{p}\right)$ vaut $1$ si $a$ est un résidu quadratique, $-1$ sinon, et $0$ si $p|a$.
\end{definition}

\begin{theorem}[Loi de Réciprocité Quadratique de Gauss]
    Soient $p$ et $q$ deux nombres premiers impairs distincts. Alors :
    $$ \left(\frac{p}{q}\right) \left(\frac{q}{p}\right) = (-1)^{\frac{p-1}{2}\frac{q-1}{2}} $$
    \textbf{Lois complémentaires :} $\left(\frac{-1}{p}\right) = (-1)^{(p-1)/2}$ et $\left(\frac{2}{p}\right) = (-1)^{(p^2-1)/8}$.
\end{theorem}

\begin{remark}[Un Dialogue entre les Nombres Premiers]
    Cette loi est stupéfiante. Elle affirme que la question "est-ce que $p$ est un carré modulo $q$ ?" est directement liée à la question "est-ce que $q$ est un carré modulo $p$ ?". Cette symétrie n'a aucune raison apparente d'exister. La chercher à la démontrer a conduit au développement de pans entiers de l'algèbre et de la théorie des nombres.
\end{remark}

\begin{application}[Calcul efficace du symbole de Legendre]
    La loi de réciprocité, utilisée conjointement avec la multiplicativité du symbole et les lois complémentaires, permet de calculer $\left(\frac{a}{p}\right)$ de manière très efficace, sans avoir à lister tous les carrés modulo $p$, en utilisant un algorithme similaire à l'algorithme d'Euclide.
\end{application}

\section{Introduction aux Équations Diophantiennes}

\begin{objectif}
    Donner un aperçu du domaine qui cherche les solutions entières ou rationnelles d'équations polynomiales. C'est un domaine d'une richesse et d'une difficulté extrêmes, qui connecte l'arithmétique à la géométrie algébrique.
\end{objectif}

\begin{proposition}[Équations linéaires $ax+by=c$]
    Cette équation admet des solutions entières $(x,y)$ si et seulement si $\mathrm{pgcd}(a,b)$ divise $c$. L'algorithme d'Euclide étendu permet de trouver une solution particulière.
\end{proposition}

\begin{theorem}[Triplets Pythagoriciens]
    Les solutions entières primitives de l'équation $x^2+y^2=z^2$ sont entièrement paramétrées par la formule $(u^2-v^2, 2uv, u^2+v^2)$ où $u, v$ sont des entiers premiers entre eux de parité contraire.
\end{theorem}

\begin{theorem}[Théorème des deux carrés de Fermat]
    Un nombre premier impair $p$ peut s'écrire comme une somme de deux carrés, $p = a^2+b^2$, si et seulement si $p \equiv 1 \pmod 4$.
\end{theorem}

\begin{remark}[Le Pont vers l'Arithmétique des Anneaux]
    Ce théorème se démontre élégamment en étudiant l'arithmétique de l'anneau des entiers de Gauss $\mathbb{Z}[i]$. Un premier $p$ est une somme de deux carrés si et seulement s'il n'est plus "premier" dans $\mathbb{Z}[i]$ (il se factorise en $(a+ib)(a-ib)$). C'est l'un des premiers exemples où l'étude d'un anneau plus grand que $\mathbb{Z}$ permet de résoudre un problème sur $\mathbb{Z}$.
\end{remark}

\begin{theorem}[Grand Théorème de Fermat (Wiles, 1994)]
    Pour tout entier $n \ge 3$, l'équation $x^n+y^n=z^n$ n'a pas de solution en entiers non nuls.
\end{theorem}
\begin{remark}[Un Sommet des Mathématiques]
    La preuve de ce théorème, longue de plus de 100 pages, est un monument des mathématiques du XXe siècle. Elle repose sur des outils extrêmement sophistiqués (courbes elliptiques, formes modulaires), montrant que des questions d'arithmétique élémentaire peuvent être d'une profondeur insondable.
\end{remark}

\part{Théorie des groupes}
\input{./chapters/groupes/théorie_des_groupes.tex}
\chapter{Théorie de Galois : La Symétrie Cachée des Nombres}

\section{Le Langage des Corps et des Extensions}

\begin{objectif}
    Construire le vocabulaire de base. L'idée révolutionnaire est de considérer les corps non pas comme des objets isolés, mais dans leurs relations les uns avec les autres. En voyant une extension de corps $L/K$ comme un $K$-espace vectoriel, on importe la puissance de l'algèbre linéaire (dimension, bases) pour "mesurer" la complexité des extensions.
\end{objectif}

\begin{definition}[Extension et son Degré]
    Une \textbf{extension de corps} est un couple de corps $(K,L)$ tel que $K$ est un sous-corps de $L$. Le \textbf{degré} de l'extension, noté $[L:K]$, est la dimension de $L$ en tant que $K$-espace vectoriel.
\end{definition}

\begin{definition}[Élément Algébrique et Transcendant]
    Un élément $\alpha \in L$ est \textbf{algébrique} sur $K$ s'il est racine d'un polynôme non nul de $K[X]$. Sinon, il est \textbf{transcendant}. Une extension est algébrique si tous ses éléments le sont.
\end{definition}

\begin{proposition}[Polynôme Minimal]
    Si $\alpha$ est algébrique sur $K$, il existe un unique polynôme unitaire et irréductible de $K[X]$, noté $\pi_\alpha$, qui annule $\alpha$. C'est le \textbf{polynôme minimal} de $\alpha$ sur $K$. De plus, $[K(\alpha):K] = \deg(\pi_\alpha)$.
\end{proposition}

\begin{theorem}[Tour de Multiplicativité des Degrés]
    Pour une tour d'extensions $K \subset L \subset M$, on a : $[M:K] = [M:L] \times [L:K]$.
\end{theorem}

\begin{application}[Impossibilité de constructions géométriques]
    Cette formule simple est déjà d'une grande puissance. Un nombre est constructible à la règle et au compas s'il appartient à une extension de $\mathbb{Q}$ de degré $2^k$. Comme $[\mathbb{Q}(\sqrt[3]{2}):\mathbb{Q}]=3$, qui n'est pas une puissance de 2, la duplication du cube est impossible.
\end{application}

\section{Construire des Extensions : Le Monde des Racines}

\begin{objectif}
    Comprendre comment sont "fabriquées" les extensions. Le plus souvent, on part d'un corps de base et on lui "adjoint" des racines de polynômes pour créer un monde plus vaste. Il faut distinguer le fait d'ajouter une seule racine (corps de rupture) de celui de les ajouter toutes (corps de décomposition).
\end{objectif}

\begin{definition}[Corps de Rupture vs. Corps de Décomposition]
    Soit $P$ un polynôme irréductible sur $K$.
    Un \textbf{corps de rupture} de $P$ est une extension $L/K$ minimale contenant \textit{au moins une} racine de $P$. (e.g., $\mathbb{Q}(\sqrt[3]{2})$ pour $X^3-2$).
    Un \textbf{corps de décomposition} de $P$ est une extension $L/K$ minimale contenant \textit{toutes} les racines de $P$. (e.g., $\mathbb{Q}(\sqrt[3]{2}, j)$ pour $X^3-2$).
\end{definition}

\begin{theorem}[Théorème de l'Élément Primitif]
    Toute extension finie et séparable (voir section suivante) est monogène. C'est-à-dire que si $L/K$ est une telle extension, il existe un élément $\alpha \in L$ (l'élément primitif) tel que $L=K(\alpha)$.
\end{theorem}

\begin{remark}[Simplification conceptuelle]
    Ce théorème est très important car il nous dit que même les extensions les plus compliquées (construites en ajoutant de nombreux éléments) peuvent être vues comme une extension "simple", engendrée par un seul élément. Cela simplifie de nombreuses preuves.
\end{remark}

\section{Les "Bonnes" Extensions : Le Cadre de la Théorie de Galois}

\begin{objectif}
    Isoler les propriétés "idéales" pour une extension. Pour que la correspondance de Galois fonctionne, il faut que l'extension soit "juste assez grande" pour contenir toutes les racines de ses polynômes (normalité) et que ces racines soient bien "distinctes" (séparabilité). Une extension de Galois est une extension qui possède ces deux propriétés.
\end{objectif}

\begin{definition}[Extension Normale]
    Une extension finie $L/K$ est \textbf{normale} si elle est le corps de décomposition d'un polynôme de $K[X]$. De manière équivalente, tout polynôme irréductible de $K[X]$ qui a une racine dans $L$ est complètement scindé dans $L$.
\end{definition}

\begin{definition}[Extension Séparable]
    Un polynôme irréductible est \textbf{séparable} si ses racines sont toutes distinctes. Une extension algébrique $L/K$ est \textbf{séparable} si le polynôme minimal de tout élément de $L$ est séparable.
\end{definition}

\begin{remark}[Une technicité cruciale en caractéristique $p$]
    En caractéristique nulle, tout polynôme irréductible est séparable. La notion de séparabilité peut donc sembler superflue. Elle devient essentielle en caractéristique $p>0$, où des polynômes comme $X^p - a$ peuvent être irréductibles mais avoir des racines multiples.
\end{remark}

\begin{definition}[Extension de Galois et Groupe de Galois]
    Une extension finie $L/K$ est une \textbf{extension de Galois} si elle est normale et séparable.
    Le \textbf{groupe de Galois} de l'extension, noté $\mathrm{Gal}(L/K)$, est le groupe des $K$-automorphismes de $L$. Si l'extension est galoisienne, $|\mathrm{Gal}(L/K)| = [L:K]$.
\end{definition}

\section{Le Théorème Fondamental : Le Dictionnaire Algèbre-Groupe}

\begin{objectif}
    Dévoiler le résultat central de la théorie : une "dualité" ou "dictionnaire" qui traduit parfaitement la structure des sous-extensions d'une extension de Galois en la structure des sous-groupes de son groupe de Galois. C'est l'un des plus beaux théorèmes des mathématiques.
\end{objectif}

\begin{theorem}[Théorème Fondamental de la Théorie de Galois]
    Soit $L/K$ une extension de Galois finie. Il existe une bijection renversant l'inclusion entre l'ensemble des sous-groupes de $\mathrm{Gal}(L/K)$ et l'ensemble des corps intermédiaires $K \subset F \subset L$.
    Un sous-groupe $H$ correspond au corps des points fixes $L^H = \{x \in L \mid \forall \sigma \in H, \sigma(x)=x\}$.
    Une sous-extension $F$ correspond au sous-groupe $\mathrm{Gal}(L/F)$.
    De plus, une sous-extension $F/K$ est elle-même galoisienne si et seulement si son groupe correspondant $\mathrm{Gal}(L/F)$ est distingué dans $\mathrm{Gal}(L/K)$. Dans ce cas, $\mathrm{Gal}(F/K) \cong \mathrm{Gal}(L/K)/\mathrm{Gal}(L/F)$.
\end{theorem}

\begin{example}[La Géométrie de $X^4-2$]
    Soit $P(X)=X^4-2$ sur $\mathbb{Q}$. Le corps de décomposition est $L = \mathbb{Q}(\sqrt[4]{2}, i)$. C'est une extension de degré 8. Le groupe de Galois $G = \mathrm{Gal}(L/\mathbb{Q})$ est isomorphe au groupe diédral $D_4$, le groupe des isométries du carré.
    Le treillis des 10 sous-groupes de $D_4$ est en bijection inversée avec le treillis des 10 sous-corps de $L$. Par exemple :
    \begin{itemize}
        \item Le sous-groupe des rotations $\langle r \rangle \cong \mathbb{Z}/4\mathbb{Z}$ correspond au sous-corps $\mathbb{Q}(i)$.
        \item Le sous-groupe engendré par une réflexion $\langle s \rangle \cong \mathbb{Z}/2\mathbb{Z}$ correspond au sous-corps $\mathbb{Q}(\sqrt[4]{2})$.
        \item Le centre du groupe $Z(G) \cong \mathbb{Z}/2\mathbb{Z}$ correspond au sous-corps $\mathbb{Q}(\sqrt{2}, i)$.
    \end{itemize}
\end{example}

\section{Applications : La Puissance du Dictionnaire}

\begin{objectif}
    Utiliser le dictionnaire pour résoudre des problèmes classiques. La stratégie est toujours la même : traduire un problème (algébrique, géométrique) en un problème sur le groupe de Galois, le résoudre dans le monde (plus simple) des groupes finis, puis retraduire la solution.
\end{objectif}

\begin{theorem}[Théorie de Galois des corps finis]
    L'extension $\mathbb{F}_{p^n}/\mathbb{F}_p$ est galoisienne. Son groupe de Galois est cyclique d'ordre $n$, engendré par l'automorphisme de Frobenius $\sigma: x \mapsto x^p$.
\end{theorem}

\begin{corollary}
    Par la correspondance de Galois, les sous-extensions de $\mathbb{F}_{p^n}$ correspondent aux sous-groupes de $\mathbb{Z}/n\mathbb{Z}$. Pour chaque diviseur $d$ de $n$, il y a un unique sous-groupe d'ordre $d$, et donc un unique sous-corps de degré $d$ sur $\mathbb{F}_p$ : le corps $\mathbb{F}_{p^d}$.
\end{corollary}

\begin{theorem}[Critère de Galois pour la résolubilité par radicaux]
    Un polynôme est résoluble par radicaux si et seulement si son groupe de Galois est un groupe résoluble (i.e. peut être décomposé en une suite de quotients abéliens).
\end{theorem}

\begin{application}[Impossibilité de la résolution de la quintique]
    Le groupe de Galois du polynôme "général" de degré $n$ est $\mathfrak{S}_n$. Pour $n \ge 5$, le groupe $\mathfrak{S}_n$ n'est pas résoluble car $\mathcal{A}_n$ est simple et non-abélien. Il n'existe donc pas de formule générale par radicaux pour les équations de degré 5 ou plus.
\end{application}

\begin{remark}[Le Problème Inverse de Galois]
    Nous savons associer un groupe de Galois à une extension. Le problème inverse est-il possible ? "Tout groupe fini peut-il être réalisé comme le groupe de Galois d'une extension de $\mathbb{Q}$ ?" C'est une question ouverte et l'un des problèmes les plus profonds de l'arithmétique moderne.
\end{remark}
\input{./chapters/groupes/représentations_de_groupes.tex}

\part{Géométrie}
\chapter{Algèbre Bilinéaire : La Géométrie Cachée des Espaces Vectoriels}

\section{Formes Bilinéaires et Dualité}

\begin{objectif}
    Enrichir la structure d'un espace vectoriel en introduisant une nouvelle opération : une manière de "multiplier" deux vecteurs pour obtenir un scalaire. Cette opération, la forme bilinéaire, est le concept le plus fondamental de la géométrie algébrique. On explore le lien intime entre cette nouvelle structure et la dualité.
\end{objectif}

\begin{definition}[Forme bilinéaire]
    Soit $E$ un $K$-espace vectoriel. Une application $\phi: E \times E \to K$ est une \textbf{forme bilinéaire} si elle est linéaire par rapport à chaque variable.
    Elle est \textbf{symétrique} si $\phi(x,y)=\phi(y,x)$ et \textbf{antisymétrique} si $\phi(x,y)=-\phi(y,x)$.
\end{definition}

\begin{definition}[Matrice d'une forme bilinéaire]
    Dans une base $\mathcal{B}=(e_i)$ de $E$, la matrice de $\phi$ est $M = (\phi(e_i, e_j))_{i,j}$. La formule de changement de base est $M' = P^T M P$, où $P$ est la matrice de passage. On dit que les matrices $M$ et $M'$ sont \textbf{congruentes}.
\end{definition}

\begin{definition}[Forme non-dégénérée et Rang]
    Le \textbf{noyau} d'une forme bilinéaire $\phi$ est $N = \{ x \in E \mid \forall y \in E, \phi(x,y)=0 \}$.
    $\phi$ est \textbf{non-dégénérée} si son noyau est réduit à $\{0\}$.
    Le \textbf{rang} de $\phi$ est le rang de sa matrice dans n'importe quelle base.
\end{definition}

\begin{proposition}[Lien avec la dualité]
    Une forme bilinéaire $\phi$ induit deux morphismes canoniques de $E$ dans son dual $E^*$, donnés par $\phi_g: x \mapsto \phi(x, \cdot)$ et $\phi_d: y \mapsto \phi(\cdot, y)$.
    $\phi$ est non-dégénérée si et seulement si ces morphismes sont des isomorphismes.
\end{proposition}
\begin{remark}[La Géométrie vient de l'Identification à l'Espace Dual]
    C'est une idée très profonde. Une forme bilinéaire non-dégénérée permet d'identifier l'espace $E$ (points) avec son dual $E^*$ (formes linéaires / "hyperplans"). C'est cette identification qui est à l'origine de toute la géométrie de l'espace.
\end{remark}

\section{Formes Quadratiques : L'Étude des "Longueurs au Carré"}

\begin{objectif}
    Se concentrer sur le cas des formes bilinéaires symétriques, qui sont plus riches. Toute forme bilinéaire symétrique $\phi$ définit une "fonction longueur au carré", la forme quadratique $q(x) = \phi(x,x)$, qui capture l'essentiel de la géométrie. On cherche à simplifier ces formes quadratiques en trouvant une "bonne" base.
\end{objectif}

\begin{definition}[Forme quadratique]
    Une application $q: E \to K$ est une \textbf{forme quadratique} s'il existe une forme bilinéaire $\phi$ telle que $q(x)=\phi(x,x)$. Si la caractéristique du corps est différente de 2, il existe une unique forme bilinéaire symétrique (la forme polaire) associée à $q$, donnée par la formule de polarisation :
    $$ \phi(x,y) = \frac{1}{2} (q(x+y) - q(x) - q(y)) $$
\end{definition}

\begin{definition}[Orthogonalité]
    Deux vecteurs $x,y$ sont \textbf{orthogonaux} pour $\phi$ si $\phi(x,y)=0$. L'orthogonal d'une partie $A \subset E$ est $A^\perp = \{ x \in E \mid \forall a \in A, \phi(x,a)=0 \}$. Une base est orthogonale si ses vecteurs sont deux à deux orthogonaux.
\end{definition}

\begin{theorem}[Réduction de Gauss]
    Toute forme quadratique sur un espace de dimension finie (en car. $\neq 2$) est "diagonalisable". C'est-à-dire qu'il existe une base orthogonale $(e_i)$ dans laquelle la forme quadratique s'écrit comme une somme de carrés :
    $$ q(x_1 e_1 + \dots + x_n e_n) = \sum_{i=1}^n \lambda_i x_i^2 \quad (\text{avec } \lambda_i = q(e_i)) $$
    L'algorithme de décomposition de Gauss est une procédure effective pour trouver une telle base.
\end{theorem}

\section{La Loi d'Inertie de Sylvester : Classifier les Géométries}

\begin{objectif}
    Aller au-delà de la "diagonalisation" et classifier les formes quadratiques sur les corps $\mathbb{R}$ et $\mathbb{C}$. Le résultat central est la loi d'inertie de Sylvester, qui affirme qu'une forme quadratique réelle est entièrement caractérisée par un couple d'entiers : sa signature.
\end{objectif}

\begin{theorem}[Classification sur $\mathbb{C}$]
    Deux formes quadratiques sur un $\mathbb{C}$-espace vectoriel de dimension finie sont équivalentes (i.e. peuvent être représentées par la même matrice dans des bases bien choisies) si et seulement si elles ont le même rang. Toute forme quadratique de rang $r$ peut s'écrire $x_1^2 + \dots + x_r^2$ dans une certaine base.
\end{theorem}

\begin{theorem}[Loi d'Inertie de Sylvester]
    Soit $q$ une forme quadratique sur un $\mathbb{R}$-espace vectoriel de dimension finie. Il existe une base orthogonale dans laquelle $q(x)$ s'écrit :
    $$ q(x) = \sum_{i=1}^s x_i^2 - \sum_{i=s+1}^{s+t} x_i^2 $$
    Le couple d'entiers $(s,t)$ ne dépend pas de la base choisie. C'est la \textbf{signature} de la forme quadratique. Le rang est $r=s+t$.
\end{theorem}

\begin{remark}[Une Classification des Géométries]
    La signature est l'invariant fondamental qui classifie toutes les géométries "bilinéaires" sur un espace vectoriel réel.
    \begin{itemize}
        \item Signature $(n,0)$ : Géométrie \textbf{Euclidienne}. La "longueur au carré" est toujours positive.
        \item Signature $(n-1,1)$ : Géométrie \textbf{Lorentzienne} ou de \textbf{Minkowski}. C'est le cadre de la relativité restreinte, où le temps a un signe opposé aux dimensions d'espace.
        \item Signature $(p,q)$ quelconque : Géométrie \textbf{pseudo-riemannienne}.
    \end{itemize}
\end{remark}

\begin{definition}[Forme définie et Cône isotrope]
    Une forme quadratique $q$ est :
    \begin{itemize}
        \item \textbf{Définie positive} si $q(x)>0$ pour tout $x \neq 0$. (Signature $(n,0)$)
        \item \textbf{Définie négative} si $q(x)<0$ pour tout $x \neq 0$. (Signature $(0,n)$)
    \end{itemize}
    Le \textbf{cône isotrope} est l'ensemble des vecteurs non nuls $x$ tels que $q(x)=0$. Pour une forme définie, il est vide. Pour la métrique de Minkowski, c'est le "cône de lumière".
\end{definition}

\chapter{Géométrie Euclidienne : La Structure de notre Intuition}

\section{Le Produit Scalaire : La Mesure des Longueurs et des Angles}

\begin{objectif}
    Se spécialiser au cas le plus important et le plus intuitif de l'algèbre bilinéaire : les espaces euclidiens, où la forme bilinéaire est symétrique et définie positive. Cette positivité est la clé qui permet de définir des notions familières de longueur, distance et angle, jetant un pont entre l'algèbre abstraite et la géométrie de notre perception.
\end{objectif}

\begin{definition}[Produit Scalaire et Espace Euclidien]
    Un \textbf{produit scalaire} sur un $\mathbb{R}$-espace vectoriel $E$ est une forme bilinéaire symétrique définie positive, notée $\langle \cdot, \cdot \rangle$.
    Un espace vectoriel de dimension finie muni d'un produit scalaire est un \textbf{espace euclidien}.
\end{definition}

\begin{definition}[Norme et Distance Euclidiennes]
    La norme euclidienne associée est $\|x\| = \sqrt{\langle x, x \rangle}$. La distance est $d(x,y) = \|x-y\|$.
\end{definition}

\begin{theorem}[Inégalité de Cauchy-Schwarz]
    Pour tout $x,y \in E$, on a $|\langle x, y \rangle| \le \|x\| \|y\|$, avec égalité si et seulement si les vecteurs sont colinéaires.
\end{theorem}
\begin{remark}[La Notion d'Angle]
    Cette inégalité est fondamentale car elle garantit que la quantité $\frac{\langle x, y \rangle}{\|x\| \|y\|}$ est toujours dans $[-1, 1]$, ce qui permet de \textit{définir} l'angle $\theta$ entre deux vecteurs non nuls par $\cos(\theta) = \frac{\langle x, y \rangle}{\|x\| \|y\|}$.
\end{remark}

\section{Orthogonalité et Bases Orthonormées}

\begin{objectif}
    Exploiter la notion d'orthogonalité pour construire les "meilleures" bases possibles pour un espace euclidien : les bases orthonormées. Dans une telle base, tous les calculs (projections, coordonnées, distances) deviennent extraordinairement simples.
\end{objectif}

\begin{definition}[Base Orthonormée]
    Une base $(e_i)$ est \textbf{orthonormée} si $\langle e_i, e_j \rangle = \delta_{ij}$ (symbole de Kronecker).
\end{definition}

\begin{proposition}[Calculs dans une base orthonormée]
    Si $\mathcal{B}=(e_i)$ est une base orthonormée, les coordonnées d'un vecteur $x$ sont simplement $x_i = \langle x, e_i \rangle$. Le produit scalaire et la norme se calculent comme à l'accoutumée :
    $$ \langle x, y \rangle = \sum_i x_i y_i \quad \text{et} \quad \|x\|^2 = \sum_i x_i^2 $$
\end{proposition}
\begin{remark}[La Simplification Ultime]
    Une base orthonormée est le "système de coordonnées cartésiennes" idéal. Elle transforme la géométrie euclidienne abstraite en l'algèbre vectorielle simple de $\mathbb{R}^n$ avec son produit scalaire usuel.
\end{remark}

\begin{theorem}[Procédé d'Orthonormalisation de Gram-Schmidt]
    A partir de n'importe quelle base d'un espace euclidien, on peut construire une base orthonormée.
\end{theorem}

\begin{corollary}
    Tout espace euclidien admet une base orthonormée.
\end{corollary}

\begin{theorem}[Projection Orthogonale]
    Soit $F$ un sous-espace vectoriel d'un espace euclidien $E$. Alors $E = F \oplus F^\perp$. Tout vecteur $x \in E$ se décompose de manière unique en $x = p_F(x) + p_{F^\perp}(x)$, où $p_F(x)$ est la projection orthogonale de $x$ sur $F$. De plus, $p_F(x)$ est le point de $F$ le plus proche de $x$.
\end{theorem}

\section{Les Transformations Euclidiennes : Isométries et Similitudes}

\begin{objectif}
    Étudier les transformations qui préservent la structure euclidienne. Ce sont les "symétries" de l'espace euclidien. On se concentrera sur les isométries (qui préservent les distances) et on montrera qu'elles forment un groupe fondamental, le groupe orthogonal.
\end{objectif}

\begin{definition}[Isométrie (Endomorphisme Orthogonal)]
    Un endomorphisme $u \in \mathcal{L}(E)$ est une \textbf{isométrie} (ou est orthogonal) s'il préserve le produit scalaire : $\forall x,y, \langle u(x), u(y) \rangle = \langle x, y \rangle$.
    Cela équivaut à préserver la norme.
\end{definition}

\begin{definition}[Groupe Orthogonal $O(E)$]
    L'ensemble des isométries de $E$ forme un groupe pour la composition, appelé \textbf{groupe orthogonal} et noté $O(E)$.
    La matrice d'une isométrie dans une base orthonormée est une matrice orthogonale ($M^T M = I_n$).
\end{definition}

\begin{theorem}[Classification des isométries en dimension 2]
    Toute isométrie du plan euclidien est soit une \textbf{rotation} (si $\det=1$), soit une \textbf{réflexion} (si $\det=-1$).
\end{theorem}

\begin{theorem}[Théorème de Rotation d'Euler]
    Toute isométrie directe (de déterminant 1) de l'espace euclidien de dimension 3 est une \textbf{rotation} autour d'un axe.
\end{theorem}

\section{Endomorphismes d'un Espace Euclidien et Théorème Spectral}

\begin{objectif}
    Utiliser la structure euclidienne pour étudier les endomorphismes. L'existence d'un produit scalaire permet de définir l'adjoint d'un endomorphisme, ce qui mène à la classe très importante des endomorphismes auto-adjoints (ou symétriques). Le théorème spectral est le résultat culminant, montrant que ces endomorphismes sont précisément ceux que l'on peut diagonaliser dans une "bonne" base géométrique (orthonormée).
\end{objectif}

\begin{proposition}[Adjoint d'un endomorphisme]
    Pour tout $u \in \mathcal{L}(E)$, il existe un unique endomorphisme $u^*$, appelé l'\textbf{adjoint} de $u$, tel que :
    $$ \forall x,y \in E, \quad \langle u(x), y \rangle = \langle x, u^*(y) \rangle $$
    Dans une base orthonormée, la matrice de $u^*$ est la transposée de la matrice de $u$.
\end{proposition}

\begin{definition}[Endomorphisme auto-adjoint (ou symétrique)]
    Un endomorphisme $u$ est \textbf{auto-adjoint} (ou symétrique) si $u=u^*$.
\end{definition}

\begin{theorem}[Théorème Spectral]
    Un endomorphisme d'un espace euclidien est auto-adjoint si et seulement s'il existe une base orthonormée de $E$ formée de vecteurs propres de $u$.
\end{theorem}
\begin{remark}[Le Triomphe de la Géométrie]
    C'est la version "euclidienne" de la réduction. Alors que la simple diagonalisation ne garantit rien sur la géométrie de la base de vecteurs propres, le théorème spectral garantit que pour les endomorphismes symétriques, on peut trouver une base qui est "parfaite" à la fois pour l'algèbre (elle diagonalise) et pour la géométrie (elle est orthonormée).
\end{remark}

\begin{application}[Décomposition en valeurs singulières (SVD)]
    Même si un endomorphisme $u$ n'est pas symétrique, l'endomorphisme $u^*u$ l'est. En appliquant le théorème spectral à $u^*u$, on peut démontrer la décomposition en valeurs singulières : tout endomorphisme $u$ peut s'écrire $u = O_1 D O_2$, où $O_1, O_2$ sont des isométries et $D$ est "diagonale". C'est un outil fondamental en analyse de données (ACP), en traitement d'images, etc.
\end{application}


\chapter{Géométrie Différentielle : Le Calcul Infinitésimal sur les Espaces Courbes}

\section{Étude Locale des Courbes}

\begin{objectif}
    Appliquer les outils du calcul différentiel pour comprendre la géométrie d'une courbe au voisinage d'un point. L'idée est de trouver le "meilleur" repère local possible (le repère de Frénet) pour décrire comment la courbe se tord et se courbe dans l'espace.
\end{objectif}

\begin{definition}[Arc Paramétré]
    Un arc paramétré de classe $\mathcal{C}^k$ est une application $\gamma: I \to \mathbb{R}^n$ de classe $\mathcal{C}^k$. Un point est régulier si $\gamma'(t) \neq 0$.
\end{definition}

\begin{definition}[Abscisse Curviligne]
    L'abscisse curviligne $s(t) = \int_{t_0}^t \|\gamma'(u)\| du$ mesure la longueur de la courbe depuis un point de départ. Un paramétrage par l'abscisse curviligne est un paramétrage "naturel" où la vitesse est toujours de norme 1.
\end{definition}

\begin{theorem}[Repère de Frénet et Courbures]
    Pour une courbe de $\mathbb{R}^3$ birégulière et paramétrée par l'abscisse curviligne $\gamma(s)$, il existe un unique repère orthonormé direct mobile $(T,N,B)$ (Tangente, Normale, Binormale) le long de la courbe tel que :
    $$
    \begin{pmatrix} T' \\ N' \\ B' \end{pmatrix} =
    \begin{pmatrix} 0 & \kappa & 0 \\ -\kappa & 0 & \tau \\ 0 & -\tau & 0 \end{pmatrix}
    \begin{pmatrix} T \\ N \\ B \end{pmatrix}
    $$
    \begin{itemize}
        \item $\kappa(s)$ est la \textbf{courbure}. Elle mesure la façon dont la courbe "tourne" dans son plan osculateur.
        \item $\tau(s)$ est la \textbf{torsion}. Elle mesure la façon dont la courbe "sort" de son plan osculateur, son "gauchissement".
    \end{itemize}
\end{theorem}
\begin{remark}[La Géométrie Locale Complètement Décrite]
    Ce résultat est fondamental : les deux fonctions $\kappa(s)$ et $\tau(s)$ déterminent entièrement la géométrie locale de la courbe. Deux courbes ayant les mêmes fonctions courbure et torsion sont identiques à une isométrie près.
\end{remark}
\begin{example}
    \begin{itemize}
        \item Un cercle de rayon $R$ a une courbure constante $1/R$ et une torsion nulle.
        \item Une hélice circulaire a une courbure et une torsion constantes.
    \end{itemize}
\end{example}

\section{Étude Locale des Surfaces}

\begin{objectif}
    Généraliser l'étude locale des courbes aux surfaces. On cherche à définir des quantités (les formes fondamentales) qui encodent toute la géométrie locale d'une surface (longueurs, angles, et surtout, courbure).
\end{objectif}

\begin{definition}[Nappe Paramétrée et Plan Tangent]
    Une nappe paramétrée est une application $\sigma: U \to \mathbb{R}^3$ (où $U \subset \mathbb{R}^2$ est un ouvert). Un point est régulier si les vecteurs $\frac{\partial\sigma}{\partial u}$ et $\frac{\partial\sigma}{\partial v}$ sont linéairement indépendants. En un point régulier, ils engendrent le \textbf{plan tangent} $T_p S$.
\end{definition}

\begin{definition}[Première Forme Fondamentale]
    La première forme fondamentale, notée $I_p$, est la forme quadratique sur le plan tangent $T_p S$ qui mesure la "longueur au carré" des vecteurs tangents. C'est la restriction du produit scalaire de $\mathbb{R}^3$ au plan tangent. Sa matrice dans la base $(\sigma_u, \sigma_v)$ est :
    $$ \begin{pmatrix} E & F \\ F & G \end{pmatrix} = \begin{pmatrix} \langle \sigma_u, \sigma_u \rangle & \langle \sigma_u, \sigma_v \rangle \\ \langle \sigma_u, \sigma_v \rangle & \langle \sigma_v, \sigma_v \rangle \end{pmatrix} $$
\end{definition}
\begin{remark}[La Métrique Intrinsèque]
    La première forme fondamentale est \textbf{intrinsèque} : elle peut être calculée par un "habitant" de la surface sans avoir à connaître l'espace ambiant $\mathbb{R}^3$. Elle lui permet de mesurer des longueurs de courbes tracées sur la surface et des angles entre vecteurs tangents.
\end{remark}

\begin{definition}[Seconde Forme Fondamentale]
    La seconde forme fondamentale, notée $II_p$, est une forme quadratique sur $T_p S$ qui mesure comment la surface "s'écarte" de son plan tangent. Elle dépend du choix d'un vecteur normal unitaire $n$.
    $$ II_p(v) = - \langle dn_p(v), v \rangle $$
    Sa matrice est $\begin{pmatrix} L & M \\ M & N \end{pmatrix}$ avec $L=\langle \sigma_{uu}, n \rangle$, etc.
\end{definition}

\begin{definition}[Application de Weingarten et Courbures]
    L'endomorphisme auto-adjoint (symétrique) $W_p = -dn_p$ du plan tangent est l'\textbf{endomorphisme de Weingarten} (ou de forme).
    \begin{itemize}
        \item Ses valeurs propres $k_1, k_2$ sont les \textbf{courbures principales}.
        \item La \textbf{courbure de Gauss} est $K = \det(W_p) = k_1 k_2$.
        \item La \textbf{courbure moyenne} est $H = \frac{1}{2} \mathrm{Tr}(W_p) = \frac{k_1+k_2}{2}$.
    \end{itemize}
\end{definition}

\begin{theorem}[Theorema Egregium de Gauss]
    La courbure de Gauss $K$ ne dépend \textbf{que} de la première forme fondamentale et de ses dérivées.
\end{theorem}
\begin{remark}[Le "Théorème Remarquable"]
    C'est l'un des résultats les plus profonds de la géométrie. Il affirme qu'une quantité (la courbure de Gauss) qui semble dépendre de la manière dont la surface est plongée dans $\mathbb{R}^3$ (via la seconde forme fondamentale) est en fait une propriété \textbf{intrinsèque} de la surface.
\end{remark}

\begin{application}[Cartographie]
    Un habitant d'une sphère (courbure constante positive) peut, par des mesures de triangles sur sa surface (la somme des angles sera > 180°), savoir qu'il vit sur une surface courbe. Le Theorema Egregium implique qu'il est impossible de dessiner une carte plane d'une portion de la Terre sans déformer les distances. Toute projection cartographique est une distorsion.
\end{application}
\chapter{Surfaces et Variétés : De la Géométrie Locale au Global}

\section{La Notion de Variété : L'Abstraction d'une "Surface"}

\begin{objectif}
    Généraliser la notion de surface. Une variété est un espace topologique qui "ressemble localement" à $\mathbb{R}^n$, mais dont la structure globale peut être très compliquée (tore, sphère...). C'est le cadre naturel pour faire de la géométrie et de l'analyse sur des espaces courbes, sans forcément les voir comme des objets plongés dans un $\mathbb{R}^N$ plus grand.
\end{objectif}

\begin{definition}[Sous-variété de $\mathbb{R}^N$]
    Une partie $M \subset \mathbb{R}^N$ est une \textbf{sous-variété} de dimension $n$ si, pour tout point $p \in M$, il existe un voisinage $U$ de $p$ dans $\mathbb{R}^N$ et une submersion $f: U \to \mathbb{R}^{N-n}$ telle que $M \cap U = f^{-1}(0)$.
    (Autres définitions équivalentes : redressement par un difféomorphisme, graphe local).
\end{definition}

\begin{example}
    La sphère $\mathbb{S}^2 = \{(x,y,z) \mid x^2+y^2+z^2=1\}$ est une sous-variété de dimension 2 de $\mathbb{R}^3$. Le groupe orthogonal $O(n)$ est une sous-variété de l'espace des matrices $\mathcal{M}_n(\mathbb{R})$.
\end{example}

\begin{definition}[Variété Différentielle Abstraite]
    Une \textbf{variété différentielle} de dimension $n$ est un espace topologique séparé à base dénombrable $M$ muni d'un \textbf{atlas}, c'est-à-dire une collection de cartes $(U_i, \phi_i)$ où les $U_i$ recouvrent $M$, $\phi_i: U_i \to \mathbb{R}^n$ est un homéomorphisme sur un ouvert, et les applications de changement de cartes $\phi_j \circ \phi_i^{-1}$ sont des difféomorphismes.
\end{definition}
\begin{remark}[La Terre est une variété]
    La Terre est une sphère. On ne peut pas la représenter avec une seule carte plane sans singularités (aux pôles). Un atlas est une collection de cartes qui se chevauchent et décrivent la totalité du globe. Les changements de cartes assurent que les notions de "dérivabilité" sont compatibles entre les différentes cartes.
\end{remark}

\section{L'Espace Tangent et les Géodésiques}

\begin{objectif}
    Définir l'analogue d'un "vecteur vitesse" ou d'un "plan tangent" pour une variété abstraite. L'espace tangent en un point est l'espace vectoriel qui constitue la meilleure approximation linéaire de la variété au voisinage de ce point.
\end{objectif}

\begin{definition}[Espace Tangent $T_p M$]
    L'espace tangent en un point $p \in M$ peut être défini de plusieurs manières équivalentes : par les classes d'équivalence de courbes paramétrées passant par $p$, ou de manière plus abstraite comme l'espace des dérivations au point $p$. C'est un espace vectoriel de même dimension que la variété.
\end{definition}

\begin{definition}[Géodésique]
    Une \textbf{géodésique} est une courbe tracée sur une variété qui généralise la notion de "ligne droite". C'est une courbe qui minimise localement la distance entre les points. De manière équivalente, c'est une courbe dont le vecteur vitesse reste parallèle à lui-même (accélération nulle dans la variété).
\end{definition}

\begin{example}
    \begin{itemize}
        \item Les géodésiques du plan euclidien sont les droites.
        \item Les géodésiques de la sphère sont les grands cercles.
        \item Les géodésiques d'un cylindre sont les hélices.
    \end{itemize}
\end{example}

\section{Le Théorème de Gauss-Bonnet : Le Pont entre Géométrie et Topologie}

\begin{objectif}
    Présenter l'un des plus beaux et des plus importants théorèmes de la géométrie, qui relie une quantité purement \textbf{géométrique} (la courbure, qui est locale) à une quantité purement \textbf{topologique} (la caractéristique d'Euler, qui est globale et invariante par déformation).
\end{objectif}

\begin{definition}[Caractéristique d'Euler-Poincaré]
    Pour une surface polyédrique, la caractéristique d'Euler est $\chi = S - A + F$ (Sommets - Arêtes + Faces). Pour une surface différentielle compacte, elle peut être définie par une triangulation. C'est un invariant topologique.
    \begin{itemize}
        \item Pour une sphère, $\chi = 2$.
        \item Pour un tore, $\chi = 0$.
        \item Pour un tore à $g$ trous, $\chi = 2 - 2g$.
    \end{itemize}
\end{definition}
\begin{remark}[Un Invariant Topologique]
    La caractéristique d'Euler ne change pas si on déforme la surface continûment. Un cube et une sphère ont tous deux $\chi=2$. C'est une mesure du nombre de "trous" de la surface.
\end{remark}

\begin{theorem}[Formule de Gauss-Bonnet pour un triangle géodésique]
    Soit $T$ un triangle dont les côtés sont des géodésiques sur une surface $S$, et dont les angles aux sommets sont $\alpha_1, \alpha_2, \alpha_3$. Alors :
    $$ \iint_T K dA + (\pi - \alpha_1) + (\pi - \alpha_2) + (\pi - \alpha_3) = 2\pi $$
    Où $K$ est la courbure de Gauss. Ceci peut se réécrire : $\sum \alpha_i - \pi = \iint_T K dA$.
    La somme des angles d'un triangle n'est plus $\pi$ ! L'excès angulaire est exactement l'intégrale de la courbure.
\end{theorem}

\begin{theorem}[Théorème de Gauss-Bonnet Global]
    Soit $S$ une surface compacte, orientable, sans bord. Alors :
    $$ \iint_S K dA = 2\pi \chi(S) $$
\end{theorem}
\begin{remark}[La Connexion Ultime]
    Ce théorème est un résultat d'une profondeur immense.
    \begin{itemize}
        \item Le membre de gauche est de nature \textbf{géométrique/analytique}. Il dépend de la métrique de la surface et se calcule par une intégrale.
        \item Le membre de droite est de nature \textbf{topologique/combinatoire}. C'est un entier qui ne dépend que de la forme globale de la surface.
    \end{itemize}
    Le théorème affirme que peu importe comment on "cabosse" ou déforme une sphère, si on intègre sa courbure de Gauss sur toute sa surface, on trouvera \textit{toujours} $4\pi$ (car $\chi(\mathbb{S}^2)=2$). Pour un tore, on trouvera toujours 0. C'est le point de départ de vastes généralisations au XXe siècle (théorie de l'indice, etc.).
\end{remark}

\part{Algèbre}
\chapter{Algèbre linéaire}

\section{Espace vectoriel}

Soit $\IK$ un corps commutatif.
\begin{definition}[Axiomes d'un $\IK$-espace vectoriel]
\end{definition}

\begin{definition}[Famille libre]
\end{definition}

\begin{definition}[Famille génératrice]
\end{definition}

\begin{definition}[Base]
\end{definition}

\begin{definition}[Dimension]
	$n\in\IN$ si base est de cardinal fini.
\end{definition}

\begin{definition}[Sous-espace vectoriel]
	Soit $A$ une partie de $E$ un $\IK$-e.v.
	On note $\Vect A$ le \emph{sous-espace vectoriel}
	$$\Vect A = \cap_{S\subset E} S$$
	de tout $\IK$-e.v. $S$ contenant $A$.
\end{definition}

\begin{definition}[Stabilité]
\end{definition}

\begin{proposition}[Critère pour déterminer si c'est un e.v.]
	Critère pour déterminer si un ensemble est un e.v. : 
	en utilisant le fait que c'est un s.e.v d'un e.v. le contenant.
	Exemple : fonctions continues de R dans R est un sev de l'ev des fonctions de R dans R.
\end{proposition}

\begin{definition}[Morphisme d'espaces vectoriel]
\end{definition}

\begin{definition}[Endomorphisme]
\end{definition}

\begin{definition}[Isomorphisme]
\end{definition}

\begin{definition}[Automorphisme]
\end{definition}

\begin{proposition}[Critère d'injectivité/surjectivité dans le cas fini]
	E s'injecte dans F si $\card E \leq \card F$.
	Surjecte si $\geq$.
	Ne s'applique pas si une des bases est de dimension infinie.
\end{proposition}

\begin{theorem}[Théorème des bases isomorphes]
	Deux bases sont de même cardinal s'il existe un isomorphisme entre ces deux bases.
\end{theorem}

\begin{example}
	Ici on donne un calcul explicite de comment on "calcul" ou complète une famille libre
	pour former une base, en utilisant l'élimination de Gauss.
\end{example}

\begin{definition}[s.e.v. supplémentaires]
\end{definition}

\begin{remark}
	Ca permet de décomposer un EV compliqué en sev potentiellement plus simples.
\end{remark}

\begin{proposition}[Projection d'un vecteur sur un s.e.v]
	Soit $E = U \oplus V$ de dimension finie.
	Alors l'écriture de $x\in E$ sur $U$ et $V$ est unique.
\end{proposition}

\begin{remark}
	Ici en fait on donne un premier exemple de récupération stratégique d'information
	sur des vecteurs. Application potentielle qu'on verra un peu plus tard : 
	décomposer une rotation en angles d'Euler selon une séquence choisie.
\end{remark}


\section{Dualité d'un espace vectoriel}
Soit $\IK$ un corps commutatif.

-> Objectif : faire comprendre que l'espace des formes linéaires est 
le point de vue 'relationnel' des e.v.
\begin{definition}[Forme linéaire]	
\end{definition}

\begin{definition}[Espace vectoriel dual]
\end{definition}

\begin{theorem}[Double dual d'un e.v.]
	Soit $E$ un $\IR$-e.v.
	Dans le cas fini, $\left(E^*\right)^* = E$.
\end{theorem}

\begin{remark}
	Insister sur la naturalité de cet isomorphisme (i.e. il n'est pas de nature constructiviste, contrairement à l'isomorphisme entre un EV et son dual).
	Egalement, insister sur l'impact de la dualité : le dual d'un e.v. permet d'avoir 
	une approche plus 'fonctionnelle' des vecteurs, et il se trouve que des fois, il est beaucoup plus
	facile de travailler dans des espaces de fonctions (notemment parce qu'on peut avoir une structure d'anneau sur cet espace).
\end{remark}


\section{Théorie des endomorphismes}

On suppose $E$ un e.v. de dimension finie.

--> Objectifs : on sépare en 2 parties le cours.
La première partie consiste à simplement faire de la théorie de matrice.
La deuxième à faire sur les endomorphismes, et donc à mettre en lien avec les matrices.

Le gros objectif de cette sous-leçon est de faire donner un point de vue un 
peu plus "géométrique" des e.v., ce qui est -potentiellement- plus naturel pour le jeune mathématicien.
Sauf que 'géométrique' implique des notions déjà vue au lycée comme l'"orthogonalité" ou le parallélisme.
Ces notions prennent une forme particulière si on veut s'affranchir de toute notion de norme (et donc ca amène 
aux espaces quotients).

\subsection{Matrices}
A FAIRE, et à voir si on ne le met pas en même temps que les endomorphismes.

\subsection{Endomorphisme d'espaces vectoriels}
\begin{definition}[Groupe linéaire]
\end{definition}

\begin{definition}[Noyau et image d'un endomorphisme]
\end{definition}

\begin{proposition}[Critère d'injectivité / surjectivité via noyau/image]
\end{proposition}

\begin{definition}[Dimension et rang d'un endomorphisme]
\end{definition}

\begin{definition}[Espace vectoriel quotient]
\end{definition}

\begin{remark}
	Vulgariser la notion de quotient comme la notion de complémentaire structuré.
	Commencer à apporter le fait qu'on peut étudier un sev V de E via son 'complémentaire' E/V.
	C'est une nouvelle approche lorsque V est trop compliqué: mtn on a soit V, soit son dual, soit E/V, ou soit $(E/V)^*$ comme outils d'études.
\end{remark}

\begin{definition}[Codimension, conoyau, corang]
\end{definition}

\begin{definition}[]
	
\end{definition}

\begin{definition}[Application transposée]
\end{definition}

\begin{theorem}[Relations entre $\im f$, $\ker f$, $\im f^t$ et $\ker f^t$]
	relations entre domaine/codomaine des endomorphismes et leurs transposés.
\end{theorem}
\begin{remark}
	Quel est l'intérêt de ce théorème ?
	On peut étudier un s.e.v en tant que l'image d'un endomorphisme, et par conséquent
	l'étudier comme le noyau de cet endo transposé (cf théorème d'après).
\end{remark}

\begin{theorem}[Théorèmes d'isomorphismes d'e.v. 1 2 et 3]
	$1. E/\ker f \cong \im f$

	Pour $U\subset V$ deux s.e.v de E,
	$2. \left(E/V\right) / \left(E/U\right) \cong V/U$

	Et enfin, l'intersection "linéarisée" pour deux $U$ et $V$ s.e.v. quelconques de $E$:
	$3. E / \left(U \cap V\right) \cong \left(E/U\right)\oplus\left(E/V\right)$
\end{theorem}

\begin{remark}
	Ca y est on a une approche quand même beaucoup plus géométrique.
	Allons encore plus loin : on a réussi à encoder l'intersection avec les e.v.,
	maintenant on va voir que les endomorphismes d'e.v. encodent également d'autres
	notions qui ont l'air initialement "métriques" : le volume (ça amène au déterminant et 
	donc on avoir une partie plus calculatoire sur les matrices en exercices.)
\end{remark}

\begin{definition}[Déterminant d'un endomorphisme, d'une matrice]
\end{definition}

\begin{remark}
	Le signe du déterminant encode l'orientation (utiliser l'exemple
	des 2 bases canoniques de $\IR^3$ avec les deux mains: il y en 2 qui sont naturelles !)

	Egalement, la valeur du déterminant correspond au volume du parallélogramme formé par la base.
\end{remark}

\begin{example}
	Ah ba la plein de calculs exo etc
\end{example}

Maintenant qu'on a les bases, on passe à ce qui est pratique:
\section{Réduction d'endomorphisme}

\begin{definition}[Valeurs propres d'un endomorphisme]
\end{definition}

\begin{definition}[Polynôme caractéristique d'un endomorphisme]
\end{definition}

\begin{definition}[Sous-espaces stables associés]
\end{definition}

\begin{theorem}[Supplémentarité des sous-espaces stables]
\end{theorem}


\begin{theorem}[Théorème des noyaux]
	Preuve utilise surement le théorème précédent.
\end{theorem}

\begin{remark}
	Bon maintenant on fait quoi en fait ?
	Ca a servi à quoi ? Ca c'est une question légitime.
	On va donc donner des exemples concrets qui sont géométriques.
	Plus particulièrement, on regarde un sous-groupe de l'e.v. des fonctions,
	et on va décomposer un vecteur de $\IR^3$ selon une homothétie et une rotation de $O_3(\IR)$
\end{remark}


\begin{definition}[Diagonalisation]
\end{definition}

\begin{definition}[Trigonalisation]
\end{definition}

\begin{definition}[Groupe orthogonal]
\end{definition}

\begin{definition}[Homothétie]
\end{definition}

\begin{proposition}[Théorème de rotation d'Euler]
	Décomposition d'un élément du groupe orthogonal (i.e. d'une rotation).
	Unicité selon la séquence choisie (NB : il n'y a pas de décomposition "canonique" !)
\end{proposition}

\begin{remark}
	on en arrive à des décompositions d'endomorphisme en plusieurs parties : on entame donc sur les plus classiques
\end{remark}

\begin{theorem}[Décomposition de Jordan]
\end{theorem}

\begin{definition}[Endomorphisme nilpotent]
\end{definition}

\begin{theorem}[Décomposition de Dunford]
\end{theorem}

\begin{theorem}[Décomposition de Schur]
\end{theorem}

\begin{remark}
	Utiliser ce théorème d'un point de vue calculatoire en utilisant la décomposition de Schur
	afin de retrouver une première décomposition de la rotation d'un vecteur selon des plans 2-à-2 orthogonaux.
\end{remark}


\chapter{Algèbre Générale : L'Architecture des Structures}

\section{Le Zoo des Structures Algébriques}

\begin{objectif}
    Poser les fondations. L'algèbre est l'étude des ensembles munis de lois. Nous allons cartographier le "zoo" des structures algébriques, en partant des plus générales (pauvres en axiomes) vers les plus riches et contraignantes. L'idée est de comprendre que chaque axiome ajouté n'est pas une complication, mais une source de nouveaux théorèmes puissants.
\end{objectif}

\begin{definition}[Le chemin vers les groupes]
    \begin{itemize}
        \item \textbf{Magma $(E, *)$ :} Un simple ensemble avec une loi de composition interne. (Ex: $(\mathbb{Z}, -)$)
        \item \textbf{Demi-groupe (Semigroup) :} Un magma où la loi est associative. (Ex: $(\mathbb{N}^*, +)$)
        \item \textbf{Monoïde :} Un demi-groupe avec un élément neutre. (Ex: $(\mathbb{N}, +)$, l'ensemble des mots sur un alphabet avec la concaténation)
        \item \textbf{Groupe :} Un monoïde où tout élément est inversible. (Ex: $(\mathbb{Z}, +)$, $(\mathbb{Q}^*,\cdot)$)
    \end{itemize}
\end{definition}

\begin{definition}[Les structures à deux lois]
    \begin{itemize}
        \item \textbf{Anneau (Ring) $(A, +, \cdot)$ :} $(A,+)$ est un groupe abélien, $(A,\cdot)$ est un monoïde, et la multiplication est distributive sur l'addition. (Ex: $\mathbb{Z}$, $\mathbb{Z}/n\mathbb{Z}$, les matrices carrées $\mathcal{M}_n(\mathbb{R})$)
        \item \textbf{Corps (Field) $(K, +, \cdot)$ :} Un anneau commutatif où tout élément non nul est inversible pour la multiplication. (Ex: $\mathbb{Q}, \mathbb{R}, \mathbb{C}, \mathbb{F}_p$)
    \end{itemize}
\end{definition}

\begin{remark}[La Richesse vient des Contraintes]
    Un corps est une structure très "rigide" : presque tout est fixé. C'est pourquoi on a une théorie très puissante (l'algèbre linéaire) sur les corps. Les anneaux sont beaucoup plus variés et "sauvages". Par exemple, dans un anneau général, $ab=0$ n'implique pas $a=0$ ou $b=0$ (diviseurs de zéro). La hiérarchie des structures est une échelle de "rigidité" croissante.
\end{remark}

\section{Le Triptyque Universel : Sous-structures, Morphismes, Quotients}

\begin{objectif}
    Dégager les trois concepts transversaux qui forment la grammaire de toute l'algèbre. Quelle que soit la structure, on cherche toujours à comprendre ses parties (sous-structures), les applications qui la préservent (morphismes), et la manière de la simplifier (quotients).
\end{objectif}

\begin{definition}[Sous-structures]
    Une sous-structure est une partie d'une structure qui est elle-même une structure du même type avec les lois induites.
    \begin{itemize}
        \item \textbf{Sous-groupe :} Stable par la loi et l'inverse.
        \item \textbf{Sous-anneau :} Stable par $+$ et $\cdot$, contient le neutre multiplicatif.
        \item \textbf{Idéal :} Le concept clé pour les anneaux. Un sous-groupe additif $I$ de $A$ tel que pour tout $a \in A$ et $x \in I$, $ax \in I$ et $xa \in I$.
    \end{itemize}
\end{definition}

\begin{remark}[Pourquoi les idéaux ?]
    Un sous-anneau n'est pas la "bonne" notion pour construire un quotient d'anneau. L'idéal est exactement la structure qui se comporte comme le noyau d'un morphisme d'anneau. Il est "absorbant", ce qui permet de définir une multiplication cohérente sur les classes d'équivalence.
\end{remark}

\begin{definition}[Morphismes]
    Un morphisme est une application entre deux structures de même type qui "respecte" les lois. Le \textbf{noyau} est l'ensemble des éléments envoyés sur l'élément neutre. L'\textbf{image} est l'ensemble des éléments atteints.
\end{definition}

\begin{theorem}[Le Premier Théorème d'Isomorphisme (Forme Générale)]
    Pour tout morphisme $\phi: A \to B$, l'image $\mathrm{Im}(\phi)$ est isomorphe à l'ensemble de départ $A$ "quotienté par ce qui est rendu trivial", c'est-à-dire le noyau $\ker(\phi)$.
    \begin{itemize}
        \item Pour les groupes : $G/\ker(\phi) \cong \mathrm{Im}(\phi)$.
        \item Pour les anneaux : $A/\ker(\phi) \cong \mathrm{Im}(\phi)$.
    \end{itemize}
\end{theorem}

\section{L'Action : Quand les Structures Agissent sur le Monde}

\begin{objectif}
    Montrer que l'algèbre devient véritablement puissante quand on l'utilise pour décrire les transformations d'autres objets. L'étude des actions de groupes est fondamentale, mais on peut généraliser cette idée aux anneaux via la notion de module, unifiant ainsi l'algèbre linéaire et la théorie des groupes abéliens.
\end{objectif}

\begin{definition}[Module sur un anneau]
    Soit $A$ un anneau. Un \textbf{$A$-module} est la généralisation d'un espace vectoriel où les scalaires sont pris dans l'anneau $A$ au lieu d'un corps. C'est un groupe abélien $(M,+)$ muni d'une "multiplication par les scalaires" de $A$ qui est distributive et associative.
\end{definition}

\begin{example}
    \begin{itemize}
        \item Si $A$ est un corps $K$, un $A$-module est un $K$-espace vectoriel.
        \item Tout groupe abélien est un $\mathbb{Z}$-module.
        \item Soit $u$ un endomorphisme d'un $K$-espace vectoriel $V$. Alors $V$ peut être muni d'une structure de $K[X]$-module où le polynôme $X$ agit comme l'endomorphisme $u$.
    \end{itemize}
\end{example}

\begin{theorem}[Théorème de Structure des Modules de Type Fini sur un Anneau Principal]
    Soit $A$ un anneau principal et $M$ un $A$-module de type fini. Alors $M$ se décompose de manière unique en une somme directe :
    $$ M \cong A^r \oplus A/(a_1) \oplus A/(a_2) \oplus \dots \oplus A/(a_k) $$
    où $r$ est le rang de $M$ et les $(a_i)$ sont les facteurs invariants de $M$ ($a_1 | a_2 | \dots | a_k$).
\end{theorem}

\begin{remark}[Un Théorème "Monstre" Unificateur]
    Ce théorème abstrait est l'un des plus puissants de l'algèbre. Il unifie des pans entiers des mathématiques. Ses deux applications principales sont des résultats majeurs que l'on démontre habituellement par des voies très différentes.
\end{remark}

\begin{application}[Classification des groupes abéliens de type fini]
    Un groupe abélien est un $\mathbb{Z}$-module. $\mathbb{Z}$ est un anneau principal. Le théorème s'applique directement et nous dit que tout groupe abélien de type fini est isomorphe à un produit de la forme $\mathbb{Z}^r \times \mathbb{Z}/n_1\mathbb{Z} \times \dots \times \mathbb{Z}/n_k\mathbb{Z}$. C'est la classification complète de ces groupes.
\end{application}

\begin{application}[Réduction des endomorphismes]
    Soit $u \in \mathcal{L}(V)$ où $V$ est un $K$-e.v. de dimension finie. On munit $V$ de sa structure de $K[X]$-module. $K[X]$ est un anneau principal. Le théorème s'applique et la décomposition en facteurs invariants correspond à la \textbf{décomposition de Frobenius} (avec les polynômes compagnons). La décomposition en facteurs primaires (utilisant le lemme des restes chinois) correspond à la \textbf{décomposition de Jordan}. Ce théorème contient donc toute la théorie de la réduction.
\end{application}

\section{Vers l'Algèbre Universelle et les Catégories}

\begin{objectif}
    Conclure en ouvrant la porte à une vision plus moderne et abstraite, celle qui définit les objets non par leur contenu, mais par leurs relations avec les autres objets, via les "propriétés universelles". C'est l'essence de la théorie des catégories.
\end{objectif}

\begin{definition}[Propriété Universelle]
    Une propriété universelle caractérise un objet $U$ par l'existence et l'unicité d'un morphisme vers (ou depuis) tout autre objet $X$ de la même catégorie.
\end{definition}

\begin{example}[Propriété universelle du groupe quotient]
    Soit $H \triangleleft G$. Le groupe quotient $G/H$ et la surjection canonique $\pi: G \to G/H$ sont caractérisés par la propriété universelle suivante : pour tout morphisme de groupes $\phi: G \to K$ tel que $H \subset \ker(\phi)$, il existe un \textbf{unique} morphisme $\tilde{\phi}: G/H \to K$ tel que $\phi = \tilde{\phi} \circ \pi$.
    Cela signifie que $G/H$ est la manière la plus "économique" et "universelle" de rendre $H$ trivial.
\end{example}

\begin{remark}[La Pensée Catégorique]
    Cette approche est au cœur de l'algèbre moderne (et de la géométrie algébrique, topologie algébrique...). On ne construit plus les objets "à la main", on les définit par la propriété qu'ils doivent vérifier. Le produit tensoriel, le groupe libre, la localisation d'anneaux... tous sont définis par des propriétés universelles. C'est un changement de perspective très puissant qui unifie l'ensemble de l'algèbre.
\end{remark}
\chapter{Anneaux et Polynômes : L'Héritage de l'Arithmétique}

\section{Le Bestiaire des Anneaux : Une Hiérarchie de "Gentillesse"}

\begin{objectif}
    Construire une hiérarchie des anneaux commutatifs unitaires. L'idée est de partir de la structure la plus générale et d'ajouter progressivement des axiomes qui nous rapprochent des propriétés familières de l'arithmétique des entiers $\mathbb{Z}$. Chaque nouvel axiome définit une classe d'anneaux plus "gentille", où des théorèmes plus puissants s'appliquent.
\end{objectif}

\begin{definition}[Anneau intègre (Integral Domain)]
    Un anneau commutatif unitaire $A$ est \textbf{intègre} si $ab=0 \implies a=0$ ou $b=0$. C'est le cadre minimal pour faire de l'arithmétique, car on peut "simplifier" ($ab=ac \implies b=c$ si $a \neq 0$).
\end{definition}

\begin{definition}[Anneau factoriel (Unique Factorization Domain - UFD)]
    Un anneau intègre $A$ est \textbf{factoriel} si tout élément non nul et non inversible se décompose de manière \textbf{unique} (à l'ordre et aux inversibles près) en un produit d'éléments irréductibles. C'est le cadre où la "décomposition en facteurs premiers" a un sens.
\end{definition}

\begin{definition}[Anneau principal (Principal Ideal Domain - PID)]
    Un anneau intègre $A$ est \textbf{principal} si tout idéal de $A$ est principal, c'est-à-dire engendré par un seul élément.
\end{definition}

\begin{definition}[Anneau euclidien (Euclidean Domain)]
    Un anneau intègre $A$ est \textbf{euclidien} s'il existe une fonction $v: A \setminus \{0\} \to \mathbb{N}$ (un "stathme") telle que pour tous $a, b \in A$ avec $b \neq 0$, il existe $q, r \in A$ avec $a = bq + r$ et ($r=0$ ou $v(r) < v(b)$). C'est le cadre où l'algorithme d'Euclide et la division euclidienne fonctionnent.
\end{definition}

\begin{theorem}[La Hiérarchie des Anneaux]
    On a la chaîne d'implications strictes :
    $$ \text{Corps} \implies \text{Anneau Euclidien} \implies \text{Anneau Principal} \implies \text{Anneau Factoriel} \implies \text{Anneau Intègre} $$
\end{theorem}

\begin{example}[Exemples et Contre-exemples Fondamentaux]
    \begin{itemize}
        \item $\mathbb{Z}$ et $K[X]$ (où $K$ est un corps) sont euclidiens. L'anneau des entiers de Gauss $\mathbb{Z}[i]$ l'est aussi.
        \item L'anneau $\mathbb{Z}[\frac{1+i\sqrt{19}}{2}]$ est un exemple célèbre d'anneau principal qui n'est pas euclidien.
        \item L'anneau des polynômes à plusieurs indéterminées $K[X,Y]$ ou l'anneau $\mathbb{Z}[X]$ sont factoriels mais \textbf{non principaux}. Par exemple, l'idéal $(2,X)$ de $\mathbb{Z}[X]$ n'est pas engendré par un seul polynôme.
        \item L'anneau $\mathbb{Z}[i\sqrt{5}]$ n'est \textbf{pas factoriel}. On y a deux décompositions distinctes de 6 en irréductibles : $6 = 2 \cdot 3 = (1+i\sqrt{5})(1-i\sqrt{5})$. L'unicité de la factorisation est perdue.
    \end{itemize}
\end{example}

\section{L'Arithmétique dans les Anneaux Factoriels}

\begin{objectif}
    Généraliser les notions de divisibilité et de "nombres premiers" au cadre des anneaux factoriels. On clarifiera la distinction subtile mais cruciale entre un élément "irréductible" et un élément "premier".
\end{objectif}

\begin{proposition}[Éléments Irréductibles vs. Premiers]
    Soit $a$ un élément non nul et non inversible d'un anneau intègre $A$.
    \begin{itemize}
        \item $a$ est \textbf{irréductible} s'il ne peut pas s'écrire comme un produit de deux non-inversibles. (On ne peut pas le "casser" davantage).
        \item $a$ est \textbf{premier} si $a | bc \implies a|b$ ou $a|c$. (C'est le lemme d'Euclide).
    \end{itemize}
\end{proposition}

\begin{proposition}
    Dans un anneau intègre, tout élément premier est irréductible. La réciproque est vraie si et seulement si l'anneau est factoriel.
\end{proposition}

\begin{remark}[La Clé de la Factorialité]
    Cette équivalence est le cœur de la factorialité. Dans $\mathbb{Z}[i\sqrt{5}]$, l'élément 2 est irréductible, mais il n'est pas premier car $2 | (1+i\sqrt{5})(1-i\sqrt{5})=6$, mais 2 ne divise aucun des deux facteurs. C'est cette défaillance du lemme d'Euclide qui cause la non-unicité de la décomposition.
\end{remark}

\section{Les Anneaux de Polynômes : L'Objet Central}

\begin{objectif}
    Étudier les propriétés arithmétiques du ring $A[X]$ en fonction de celles de $A$. Le résultat principal est un théorème de transfert de la factorialité, qui est un des piliers de l'algèbre commutative. On développera ensuite une boîte à outils pour tester l'irréductibilité, la "primalité" des polynômes.
\end{objectif}

\begin{lemma}[Lemme de Gauss]
    Soit $A$ un anneau factoriel de corps des fractions $K$. Un polynôme $P \in A[X]$ non constant est irréductible dans $A[X]$ si et seulement s'il est primitif et irréductible dans $K[X]$.
\end{lemma}

\begin{theorem}[Théorème Fondamental]
    Si $A$ est un anneau factoriel, alors l'anneau des polynômes $A[X]$ est aussi un anneau factoriel.
\end{theorem}

\begin{corollary}
    Par récurrence, les anneaux $\mathbb{Z}[X_1, \dots, X_n]$ et $K[X_1, \dots, X_n]$ (où $K$ est un corps) sont factoriels.
\end{corollary}

\begin{theorem}[Critères d'Irréductibilité sur Q[X] (ou sur le corps des fractions d'un UFD)]
    \begin{itemize}
        \item \textbf{Contenu :} Un polynôme de $\mathbb{Z}[X]$ est irréductible sur $\mathbb{Q}[X]$ ssi il l'est sur $\mathbb{Z}[X]$.
        \item \textbf{Eisenstein :} Soit $P(X) = a_n X^n + \dots + a_0 \in \mathbb{Z}[X]$. S'il existe un nombre premier $p$ tel que $p \nmid a_n$, $p | a_i$ pour $i<n$, et $p^2 \nmid a_0$, alors $P$ est irréductible sur $\mathbb{Q}$.
        \item \textbf{Réduction modulo $p$ :} Si la réduction $\bar{P}$ de $P$ modulo $p$ est irréductible dans $\mathbb{F}_p[X]$ (et $\deg(\bar{P})=\deg(P)$), alors $P$ est irréductible sur $\mathbb{Q}$.
    \end{itemize}
\end{theorem}

\begin{application}[Irréductibilité du polynôme cyclotomique $\Phi_p$]
    Le $p$-ième polynôme cyclotomique est $\Phi_p(X) = \frac{X^p-1}{X-1} = X^{p-1} + \dots + 1$. Il est irréductible sur $\mathbb{Q}$.
    \textit{Preuve :} On considère $\Phi_p(X+1) = \frac{(X+1)^p-1}{X} = \sum_{k=1}^p \binom{p}{k} X^{k-1}$. Tous les coefficients binomiaux sont divisibles par $p$, sauf le coefficient dominant. Le coefficient constant est $\binom{p}{1}=p$, qui n'est pas divisible par $p^2$. Par le critère d'Eisenstein (translaté), $\Phi_p(X+1)$ est irréductible, donc $\Phi_p(X)$ l'est aussi.
\end{application}

\section{Polynômes Symétriques, Résultant et Discriminant}

\begin{objectif}
    Développer des outils avancés pour étudier les relations entre les racines d'un polynôme sans avoir à les calculer. Les polynômes symétriques fournissent le langage, et le résultant/discriminant sont des outils de calcul effectifs.
\end{objectif}

\begin{definition}[Polynômes Symétriques Élémentaires]
    Les polynômes symétriques élémentaires en $n$ variables $X_1, \dots, X_n$ sont :
    $\sigma_1 = \sum X_i$, $\sigma_2 = \sum_{i<j} X_i X_j$, ..., $\sigma_n = X_1 \cdots X_n$.
\end{definition}

\begin{theorem}[Théorème Fondamental des Polynômes Symétriques]
    Tout polynôme symétrique à coefficients dans un anneau $A$ peut s'écrire de manière unique comme un polynôme en les polynômes symétriques élémentaires à coefficients dans $A$.
\end{theorem}

\begin{remark}[Le lien universel entre coefficients et racines]
    Si $P(X) = \prod (X-x_i) = X^n - c_1 X^{n-1} + \dots + (-1)^n c_n$, alors les coefficients $c_k$ sont précisément les polynômes symétriques élémentaires évalués aux racines $x_i$ : $c_k = \sigma_k(x_1, \dots, x_n)$. Le théorème précédent implique que toute expression symétrique des racines peut être réécrite en fonction des coefficients du polynôme, sans connaître les racines elles-mêmes.
\end{remark}

\begin{definition}[Résultant et Discriminant]
    Soient $P, Q$ deux polynômes. Le \textbf{résultant} $\mathrm{Res}(P,Q)$ est le déterminant de leur matrice de Sylvester. Il est nul si et seulement si $P$ et $Q$ ont une racine commune.
    Le \textbf{discriminant} de $P$ est (à un facteur près) $\mathrm{Res}(P, P')$. Il est nul si et seulement si $P$ a une racine multiple. Pour $P(X) = a_n \prod (X-x_i)$, on a $\Delta = a_n^{2n-2} \prod_{i<j} (x_i-x_j)^2$.
\end{definition}

\begin{application}[Calcul explicite en théorie de Galois]
    Le discriminant est un outil essentiel. Par exemple, le groupe de Galois d'un polynôme irréductible de degré 3 sur $\mathbb{Q}$ est soit $\mathfrak{S}_3$, soit le sous-groupe cyclique $\mathcal{A}_3$. Lequel des deux ? On calcule le discriminant $\Delta$. Si $\sqrt{\Delta}$ est dans $\mathbb{Q}$, le groupe est $\mathcal{A}_3$. Sinon, c'est $\mathfrak{S}_3$.
\end{application}
\chapter{Fiche de Synthèse : La Philosophie des Modules (hors programme) : Le Grand Unificateur de l'Algèbre}

\section{Qu'est-ce qu'un Module, Conceptuellement ?}

\begin{objectif}
    Comprendre le module comme l'ultime généralisation de la notion d'espace vectoriel. En remplaçant le corps des scalaires par un simple anneau, on perd des propriétés fortes (comme l'existence systématique d'une base), mais on gagne un pouvoir d'unification conceptuel immense.
\end{objectif}

\begin{definition}[Module sur un anneau]
    Soit $A$ un anneau. Un \textbf{$A$-module} (à gauche) est un groupe abélien $(M,+)$ muni d'une loi externe $A \times M \to M$ qui vérifie des axiomes de compatibilité (distributivité, associativité mixte, $1_A \cdot x = x$).
\end{definition}

\begin{remark}[L'Analogie Fondamentale]
    La relation entre ces structures est une simple substitution :
    \begin{center}
    \begin{tabular}{lcl}
         Action d'un \textbf{corps} $K$ sur un groupe abélien & $\implies$ & \textbf{$K$-Espace Vectoriel} \\
         Action d'un \textbf{anneau} $A$ sur un groupe abélien & $\implies$ & \textbf{$A$-Module}
    \end{tabular}
    \end{center}
    La théorie des modules contient donc l'algèbre linéaire comme un cas particulier. Mais elle contient bien plus...
\end{remark}

\section{Le Dictionnaire : Trois Exemples Fondamentaux pour l'Agrégation}

\begin{objectif}
    Établir le "dictionnaire" qui traduit des concepts familiers en langage des modules. C'est la clé pour comprendre la puissance de cette théorie.
\end{objectif}

\begin{example}[Les Groupes Abéliens sont des $\mathbb{Z}$-modules]
    Un groupe abélien $(G,+)$ est \textbf{canoniquement} un $\mathbb{Z}$-module.
    \begin{itemize}
        \item \textbf{L'action :} Pour $n \in \mathbb{Z}$ et $g \in G$, l'action $n \cdot g$ est simplement $g$ additionné à lui-même $n$ fois (ou $-g$, $|n|$ fois si $n<0$).
        \item \textbf{Le dictionnaire :}
        \begin{itemize}
            \item un sous-groupe de $G$ $\iff$ un sous-$\mathbb{Z}$-module de $G$.
            \item un morphisme de groupes $\iff$ un morphisme de $\mathbb{Z}$-modules.
            \item un élément de torsion de $G$ $\iff$ un élément de torsion du $\mathbb{Z}$-module $G$.
        \end{itemize}
    \end{itemize}
\end{example}

\begin{example}
	\textbf{Un Endomorphisme induit une structure de $K[X]$-module}.
    C'est l'idée la plus profonde et la plus utile. Soit $V$ un $K$-espace vectoriel et $u \in \mathcal{L}(V)$ un endomorphisme. On peut alors voir $V$ comme un $K[X]$-module.
    \begin{itemize}
        \item \textbf{L'action :} Pour un polynôme $P(X) = \sum a_k X^k \in K[X]$ et un vecteur $v \in V$, l'action est définie par $P(X) \cdot v = P(u)(v) = \sum a_k u^k(v)$. L'action de la variable $X$ est simplement l'application de l'endomorphisme $u$.
        \item \textbf{Le dictionnaire :}
        \begin{itemize}
            \item un sous-espace de $V$ \textbf{stable par $u$} $\iff$ un sous-$K[X]$-module de $V$.
            \item un morphisme d'espaces vectoriels qui commute avec $u$ $\iff$ un morphisme de $K[X]$-modules.
            \item le polynôme minimal de $u$ $\iff$ l'annulateur du $K[X]$-module $V$.
        \end{itemize}
    \end{itemize}
\end{example}

\section{Le Théorème de Structure : Le "Marteau" Unificateur}

\begin{objectif}
    Présenter le "théorème monstre" qui, appliqué à nos deux exemples principaux, va résoudre d'un seul coup deux problèmes de classification majeurs en algèbre.
\end{objectif}

\begin{theorem}[Théorème de Structure des Modules de Type Fini sur un Anneau Principal]
    Soit $A$ un anneau \textbf{principal} et $M$ un $A$-module de type fini. Alors $M$ est isomorphe à une somme directe unique de la forme :
    $$ M \cong A^r \oplus \frac{A}{(a_1)} \oplus \frac{A}{(a_2)} \oplus \dots \oplus \frac{A}{(a_k)} $$
    où $r \geq 0$ est le rang du module, et les $a_i \in A$ sont des éléments non-inversibles (uniques à association près) appelés \textbf{facteurs invariants}, vérifiant $a_1 | a_2 | \dots | a_k$.
\end{theorem}

\begin{application}[Le Marteau en Action : Deux Théorèmes pour le prix d'Un]
    Ce théorème unique a deux corollaires spectaculaires, qui sont des chapitres entiers du programme de l'agrégation.
    
    \paragraph{Cas 1 : L'anneau est $\mathbb{Z}$ (principal).}
    \begin{itemize}
        \item \textbf{Traduction :} Un $\mathbb{Z}$-module de type fini est un groupe abélien de type fini.
        \item \textbf{Résultat :} Le théorème de structure donne la \textbf{classification des groupes abéliens de type fini}.
        Tout groupe abélien de type fini est isomorphe à un produit $\mathbb{Z}^r \times \mathbb{Z}/n_1\mathbb{Z} \times \dots \times \mathbb{Z}/n_k\mathbb{Z}$ avec $n_1 | \dots | n_k$.
    \end{itemize}

    \paragraph{Cas 2 : L'anneau est $K[X]$ (principal).}
    \begin{itemize}
        \item \textbf{Traduction :} Un $K[X]$-module de type fini et de torsion est un $K$-espace vectoriel de dimension finie muni d'un endomorphisme $u$.
        \item \textbf{Résultat :} Le théorème de structure donne les \textbf{théorèmes de réduction des endomorphismes}.
            \begin{itemize}
                \item La décomposition en facteurs invariants correspond à la \textbf{décomposition de Frobenius}, où les $A/(a_i)$ sont des sous-espaces cycliques dont les matrices sont des \textbf{blocs compagnons} des polynômes invariants.
                \item En utilisant le lemme des restes chinois pour décomposer les $A/(a_i)$, on obtient la \textbf{décomposition de Jordan}, où les blocs correspondent aux diviseurs élémentaires (les facteurs primaires des polynômes invariants).
            \end{itemize}
    \end{itemize}
\end{application}

\section{Conclusion : Comment Utiliser cette Fiche à l'Agrégation ?}

\begin{objectif}
    Transformer cette connaissance "méta" en un avantage stratégique concret pour les épreuves.
\end{objectif}

\begin{remark}[Une Carte, pas un Territoire à Exposer]
    Cette fiche est votre carte mentale personnelle. Elle vous assure de ne jamais vous perdre, car vous voyez le paysage global de l'algèbre.
    \begin{itemize}
        \item \textbf{À NE PAS FAIRE :} Commencer une leçon sur la réduction en disant "Soit V un K[X]-module". C'est une erreur pédagogique à l'oral, sauf si la leçon porte spécifiquement sur les anneaux principaux.
        \item \textbf{À FAIRE :} Utiliser cette vision pour structurer votre pensée. Une fois une leçon sur la réduction parfaitement maîtrisée avec les outils de l'algèbre linéaire, vous pouvez conclure par une \textbf{remarque d'ouverture} :
        
        \textit{"Il est remarquable que cette théorie de la réduction puisse être vue comme un cas particulier d'un théorème de structure beaucoup plus général sur les modules sur un anneau principal, qui unifie également la classification des groupes abéliens."}
        
        Une telle phrase, placée judicieusement, signale au jury une profondeur de compréhension et une hauteur de vue exceptionnelles. C'est l'arme secrète d'un candidat qui ne se contente pas de connaître, mais qui \textbf{comprend}.
    \end{itemize}
\end{remark}

\part{Intégration}
\chapter{Intégration de Riemann : La Gloire et les Limites d'une Idée Intuitive}


\section{La Construction de l'Intégrale : L'Idée Géométrique}

\begin{objectif}
    Construire rigoureusement l'intégrale à partir de l'idée la plus intuitive qui soit : approcher l'aire sous une courbe par des rectangles. On va formaliser cette idée d'approximation par le bas et par le haut (sommes de Darboux) pour "piéger" la valeur de l'intégrale.
\end{objectif}

\begin{definition}[Subdivisions et Sommes de Darboux]
    Soit $f: [a,b] \to \mathbb{R}$ une fonction bornée. Soit $\sigma = (x_0, \dots, x_n)$ une subdivision de $[a,b]$. On pose $m_i = \inf_{[x_i, x_{i+1}]} f$ et $M_i = \sup_{[x_i, x_{i+1}]} f$.
    \begin{itemize}
        \item La \textbf{somme de Darboux inférieure} est $s(f, \sigma) = \sum_{i=0}^{n-1} m_i (x_{i+1}-x_i)$.
        \item La \textbf{somme de Darboux supérieure} est $S(f, \sigma) = \sum_{i=0}^{n-1} M_i (x_{i+1}-x_i)$.
    \end{itemize}
\end{definition}

\begin{remark}[L'Encadrement]
    L'idée de Darboux est de créer un encadrement de l'aire "vraie" qui est indépendant de tout choix de points intermédiaires (contrairement aux sommes de Riemann). Pour toute subdivision, on a $s(f, \sigma) \le \text{Aire} \le S(f, \sigma)$. Raffiner une subdivision resserre cet encadrement.
\end{remark}

\begin{definition}[Intégrabilité au sens de Riemann]
    Une fonction bornée $f$ est \textbf{Riemann-intégrable} sur $[a,b]$ si son intégrale inférieure et son intégrale supérieure coïncident :
    $$ \sup_{\sigma} s(f, \sigma) = \inf_{\sigma} S(f, \sigma) $$
    Cette valeur commune est alors notée $\int_a^b f(x) dx$.
    Un critère équivalent est que pour tout $\epsilon > 0$, il existe une subdivision $\sigma$ telle que $S(f,\sigma) - s(f,\sigma) < \epsilon$.
\end{definition}

\section{Le Champ d'Application : Quelles Fonctions sont Intégrables ?}

\begin{objectif}
    Identifier les classes de fonctions pour lesquelles cette construction fonctionne. On verra que l'intégrale de Riemann est très robuste pour les fonctions "régulières", mais qu'elle est très sensible aux discontinuités "abondantes".
\end{objectif}

\begin{theorem}[Classes de fonctions intégrables]
    \begin{itemize}
        \item Toute fonction \textbf{continue} sur $[a,b]$ est Riemann-intégrable.
        \item Toute fonction \textbf{monotone} sur $[a,b]$ est Riemann-intégrable.
        \item Toute fonction continue par morceaux sur $[a,b]$ est Riemann-intégrable.
    \end{itemize}
\end{theorem}

\begin{theorem}[Critère de Lebesgue pour l'intégrabilité de Riemann]
    Une fonction bornée $f: [a,b] \to \mathbb{R}$ est Riemann-intégrable si et seulement si l'ensemble de ses points de discontinuité est de \textbf{mesure de Lebesgue nulle}.
\end{theorem}

\begin{remark}[La Puissance et la Limite de l'Intégrale de Riemann]
    Ce théorème est extraordinaire. Il nous dit que l'intégrale de Riemann est "aveugle" aux pathologies qui se produisent sur des ensembles "petits" (de mesure nulle). On peut avoir une infinité de discontinuités, tant qu'elles sont négligeables.
    Cependant, ce théorème montre aussi la limite de la théorie : pour la comprendre pleinement, on a besoin du langage de la "mesure de Lebesgue", qui est l'objet de la théorie suivante. L'intégrale de Riemann n'est pas auto-contenue.
\end{remark}

\begin{example}[Des cas fascinants]
    \begin{itemize}
        \item La \textbf{fonction de Thomae} (ou "popcorn") est discontinue sur tous les rationnels et continue sur tous les irrationnels de $[0,1]$. L'ensemble de ses discontinuités, $\mathbb{Q} \cap [0,1]$, est dénombrable et donc de mesure nulle. Elle est donc Riemann-intégrable (et son intégrale vaut 0).
        \item La \textbf{fonction de Dirichlet}, $\mathbf{1}_{\mathbb{Q}}$, est discontinue en tout point de $[0,1]$. L'ensemble de ses discontinuités est $[0,1]$, qui n'est pas de mesure nulle. Elle n'est donc \textbf{pas} Riemann-intégrable.
    \end{itemize}
\end{example}

\section{Les Théorèmes Fondamentaux : Le Pont entre Intégrale et Dérivée}

\begin{objectif}
    Révéler la connexion miraculeuse entre deux concepts a priori distincts : l'intégrale (une notion "globale" d'aire) et la dérivée (une notion "locale" de pente). Ce lien est le cœur du calcul différentiel et intégral.
\end{objectif}

\begin{theorem}[Premier Théorème Fondamental de l'Analyse]
    Soit $f: [a,b] \to \mathbb{R}$ une fonction continue. La fonction "aire" $F(x) = \int_a^x f(t) dt$ est de classe $\mathcal{C}^1$ sur $[a,b]$ et vérifie $F'(x) = f(x)$. L'intégration est donc un processus qui "régularise" les fonctions.
\end{theorem}

\begin{theorem}[Second Théorème Fondamental de l'Analyse]
    Soit $F: [a,b] \to \mathbb{R}$ une fonction de classe $\mathcal{C}^1$. Alors :
    $$ \int_a^b F'(t) dt = F(b) - F(a) $$
\end{theorem}
\begin{remark}[La Dualité Intégration/Dérivation]
    Ces deux théorèmes montrent que la dérivation et l'intégration (à une constante près) sont des opérations inverses l'une de l'autre. C'est ce qui rend le calcul d'intégrales possible : au lieu de calculer des limites de sommes de Riemann, il suffit de trouver une primitive.
\end{remark}

\begin{application}[Formules de calcul]
    Les deux grandes techniques de calcul d'intégrales, l'\textbf{intégration par parties} et le \textbf{changement de variables}, sont des corollaires directs de ce lien et des formules de dérivation d'un produit et d'une composée.
\end{application}

\section{Extensions et Limites de la Théorie}

\begin{objectif}
    Étendre la notion d'intégrale aux domaines non compacts (intervalles infinis) et aux fonctions non bornées. Puis, identifier les faiblesses structurelles profondes de la théorie de Riemann qui motivent la construction d'une nouvelle théorie (celle de Lebesgue).
\end{objectif}

\begin{definition}[Intégrales Impropres (ou Généralisées)]
    On étend la notion d'intégrale en passant à la limite :
    \begin{itemize}
        \item Sur un intervalle non borné : $\int_a^\infty f(t) dt = \lim_{X \to \infty} \int_a^X f(t) dt$.
        \item Pour une fonction non bornée en $a$ : $\int_a^b f(t) dt = \lim_{\epsilon \to 0^+} \int_{a+\epsilon}^b f(t) dt$.
    \end{itemize}
\end{definition}

\begin{example}[La fonction Gamma d'Euler]
    La fonction $\Gamma(x) = \int_0^\infty t^{x-1}e^{-t}dt$ est définie par une intégrale impropre aux deux bornes. Elle généralise la factorielle aux nombres complexes.
\end{example}

\begin{remark}[Les Faiblesses Structurelles de l'Intégrale de Riemann]
    Malgré son succès, la théorie de Riemann souffre de deux défauts majeurs qui la rendent inadaptée à l'analyse moderne.
    \begin{enumerate}
        \item \textbf{Le problème de la complétude :} L'espace des fonctions Riemann-intégrables sur $[a,b]$, muni de la norme $L^1$ ou $L^2$, n'est \textbf{pas un espace de Banach}. Il y a des suites de Cauchy de fonctions Riemann-intégrables qui convergent vers des fonctions non-Riemann-intégrables. C'est un cadre trop "troué" pour l'analyse fonctionnelle.
        \item \textbf{Le problème de l'interversion des limites :} Les théorèmes permettant d'intervertir les limites et l'intégrale sont très faibles. La convergence simple ne suffit pas, et même la convergence dominée requiert des hypothèses très fortes (comme la convergence uniforme) qui sont souvent fausses en pratique.
    \end{enumerate}
    Ces défauts ne sont pas des problèmes techniques que l'on peut "réparer". Ils sont inhérents à la construction même de l'intégrale, basée sur le découpage du domaine de définition. Pour les surmonter, il faudra changer de philosophie : c'est l'objet de la théorie de la mesure et de l'intégration de Lebesgue.
\end{remark}
\chapter{Fourier et Espaces $L^p$ : Décomposer les Fonctions en Atomes}

\section{Espaces de Banach $L^p$ : Mesurer la "Taille" des Fonctions}

\begin{objectif}
    Construire les espaces fonctionnels fondamentaux de l'analyse moderne. L'idée est de généraliser la notion de norme vectorielle à des espaces de fonctions, en "sommant" (via une intégrale) la taille locale de la fonction. Chaque valeur de $p$ définit une manière différente de mesurer cette taille globale, donnant naissance à une famille d'espaces aux propriétés riches et variées.
\end{objectif}

\begin{definition}[Norme et Espace $L^p$]
    Soit $(\Omega, \mathcal{A}, \mu)$ un espace mesuré et $p \in [1, \infty[$. La \textbf{norme $L^p$} d'une fonction mesurable $f: \Omega \to \mathbb{C}$ est :
    $$ \|f\|_p = \left( \int_\Omega |f(x)|^p d\mu(x) \right)^{1/p} $$
    L'espace $L^p(\Omega)$ est l'ensemble des (classes d'équivalence p.p. de) fonctions mesurables pour lesquelles cette norme est finie.
    Pour $p=\infty$, $\|f\|_\infty = \inf \{ M \ge 0 \mid |f(x)| \le M \text{ p.p.}\}$ (le "sup essentiel").
\end{definition}

\begin{theorem}[Inégalités fondamentales]
    \begin{itemize}
        \item \textbf{Inégalité de Hölder :} Pour $p,q \in [1,\infty]$ tels que $\frac{1}{p}+\frac{1}{q}=1$, et pour $f \in L^p, g \in L^q$, on a $fg \in L^1$ et $\|fg\|_1 \le \|f\|_p \|g\|_q$.
        \item \textbf{Inégalité de Minkowski :} L'application $\|\cdot\|_p$ est une norme (vérifie l'inégalité triangulaire).
    \end{itemize}
\end{theorem}

\begin{theorem}[Théorème de Riesz-Fischer]
    Pour tout $p \in [1, \infty]$, l'espace $(L^p(\Omega), \|\cdot\|_p)$ est un espace de Banach.
\end{theorem}
\begin{remark}[La puissance de l'intégrale de Lebesgue]
    Ce théorème est l'une des raisons d'être de l'intégrale de Lebesgue. L'espace des fonctions Riemann-intégrables n'est pas complet. La complétude des espaces $L^p$ est ce qui en fait le cadre parfait pour l'analyse fonctionnelle, garantissant la convergence des processus d'approximation.
\end{remark}

\begin{theorem}[Dualité des espaces $L^p$]
    Pour $p \in [1, \infty[$, le dual topologique de $L^p(\Omega)$ s'identifie isométriquement à $L^q(\Omega)$ où $\frac{1}{p}+\frac{1}{q}=1$. En particulier, les espaces $L^p$ pour $p \in ]1,\infty[$ sont réflexifs. Les espaces $L^1$ et $L^\infty$ ne le sont pas en général.
\end{theorem}

\section{L'Espace de Hilbert $L^2$ : La Géométrie de l'Analyse}

\begin{objectif}
    Se concentrer sur le cas unique $p=2$, où la norme dérive d'un produit scalaire. Cela dote l'espace $L^2$ d'une structure euclidienne de dimension infinie, où l'intuition géométrique (orthogonalité, projection, Pythagore) devient un outil d'analyse extraordinairement puissant.
\end{objectif}

\begin{definition}[Structure Hilbertienne de $L^2$]
    L'espace $L^2(\Omega)$ est un espace de Hilbert pour le produit scalaire hermitien :
    $$ \langle f, g \rangle = \int_\Omega f(x) \overline{g(x)} d\mu(x) $$
    Deux fonctions sont "orthogonales" si leur produit scalaire est nul.
\end{definition}

\begin{theorem}[Projection sur un Convexe Fermé]
    C'est le théorème central de la géométrie hilbertienne, qui garantit l'existence et l'unicité de la meilleure approximation d'un élément par un point d'un convexe fermé.
\end{theorem}

\begin{application}[Moindres carrés et approximation]
    Chercher la meilleure approximation d'une fonction $f$ dans un sous-espace $F$ (par exemple, un espace de polynômes) au sens de l'énergie (norme $L^2$) revient à calculer la projection orthogonale de $f$ sur $F$. C'est le fondement de l'approximation polynomiale, des ondelettes, etc.
\end{application}

\section{Séries de Fourier : La Décomposition Harmonique sur le Cercle}

\begin{objectif}
    Appliquer la théorie des espaces de Hilbert au cas canonique de $L^2([0, 2\pi])$. On montre que les fonctions trigonométriques $\{e^{int}\}_{n \in \mathbb{Z}}$ forment une "base" de cet espace, permettant de décomposer n'importe quelle fonction périodique en une somme de ses "harmoniques".
\end{objectif}

\begin{theorem}[Base hilbertienne trigonométrique]
    La famille de fonctions $(e_n)_{n \in \mathbb{Z}}$ définie par $e_n(t) = \frac{1}{\sqrt{2\pi}}e^{int}$ forme une base orthonormale (base hilbertienne) de l'espace de Hilbert $L^2([0, 2\pi])$.
\end{theorem}

\begin{corollary}[Décomposition et Identité de Parseval]
    Toute fonction $f \in L^2([0, 2\pi])$ se décompose de manière unique $f(t) = \sum_{n=-\infty}^\infty c_n(f) e^{int}$, où la convergence a lieu au sens de la norme $L^2$. De plus, on a l'identité de Parseval (Pythagore en dimension infinie) :
    $$ \frac{1}{2\pi} \int_0^{2\pi} |f(t)|^2 dt = \sum_{n=-\infty}^\infty |c_n(f)|^2 $$
\end{corollary}

\begin{remark}[Les différents modes de convergence]
    La convergence de la série de Fourier est une question subtile.
    \begin{itemize}
        \item \textbf{Convergence $L^2$ :} Toujours vraie pour une fonction de carré intégrable.
        \item \textbf{Convergence ponctuelle :} Garantie si la fonction est $\mathcal{C}^1$ par morceaux (Théorème de Dirichlet).
        \item \textbf{Convergence uniforme :} Garantie si la fonction est continue et $\mathcal{C}^1$ par morceaux.
    \end{itemize}
    Il existe des fonctions continues dont la série de Fourier diverge en certains points (un résultat profond d'analyse fonctionnelle utilisant Banach-Steinhaus).
\end{remark}

\section{La Transformation de Fourier : L'Analyse sur la Droite Réelle}

\begin{objectif}
    Étendre l'analyse harmonique aux fonctions non périodiques définies sur $\mathbb{R}$. La somme discrète (série de Fourier) devient une somme continue (intégrale de Fourier), ce qui permet d'analyser le "contenu fréquentiel" de n'importe quel signal de durée finie.
\end{objectif}

\begin{definition}[Transformée de Fourier]
    Pour une fonction $f \in L^1(\mathbb{R})$, sa transformée de Fourier $\hat{f}$ (ou $\mathcal{F}(f)$) est la fonction définie sur $\mathbb{R}$ par :
    $$ \hat{f}(\xi) = \int_{-\infty}^\infty f(x) e^{-2i\pi x \xi} dx $$
\end{definition}

\begin{theorem}[Propriétés fondamentales]
    \begin{itemize}
        \item \textbf{Lemme de Riemann-Lebesgue :} Si $f \in L^1$, alors $\hat{f}$ est une fonction continue qui tend vers 0 à l'infini.
        \item \textbf{Théorème de convolution :} $\widehat{f*g} = \hat{f} \cdot \hat{g}$. La transformée de Fourier transforme la convolution (opération d'analyse) en un simple produit (opération algébrique).
        \item \textbf{Formule d'inversion de Fourier :} Sous de bonnes conditions, on peut retrouver $f$ à partir de $\hat{f}$ par $f(x) = \int_{-\infty}^\infty \hat{f}(\xi) e^{2i\pi x \xi} d\xi$.
    \end{itemize}
\end{theorem}

\begin{theorem}[Théorème de Plancherel]
    La transformation de Fourier, initialement définie sur $L^1 \cap L^2$, se prolonge de manière unique en un isomorphisme d'espaces de Hilbert de $L^2(\mathbb{R})$ dans lui-même. En particulier, elle préserve le produit scalaire et la norme (Identité de Plancherel) :
    $$ \int_{-\infty}^\infty |f(x)|^2 dx = \int_{-\infty}^\infty |\hat{f}(\xi)|^2 d\xi $$
\end{theorem}

\begin{application}[Principe d'incertitude de Heisenberg]
    Ce principe fondamental de la mécanique quantique et du traitement du signal est une conséquence directe des propriétés de la transformée de Fourier. Il stipule qu'une fonction et sa transformée de Fourier ne peuvent pas être simultanément très localisées. Plus un signal est concentré dans le temps, plus son spectre de fréquences est étalé, et vice-versa.
\end{application}

\section{Distributions : La Théorie des Fonctions Généralisées}

\begin{objectif}
    Créer un cadre théorique rigoureux pour manipuler des "fonctions" pathologiques comme le "pic de Dirac" $\delta_0$, qui sont infinies en un point et nulles ailleurs, mais dont l'intégrale vaut 1. L'idée de Laurent Schwartz est de définir ces objets non pas par leurs valeurs, mais par la manière dont ils agissent sur des fonctions très régulières.
\end{objectif}

\begin{definition}[Distribution]
    Une \textbf{distribution} sur un ouvert $\Omega \subset \mathbb{R}^n$ est une forme linéaire continue sur l'espace des fonctions test $\mathcal{D}(\Omega)$ (fonctions $\mathcal{C}^\infty$ à support compact).
\end{definition}

\begin{example}[Distributions usuelles]
    \begin{itemize}
        \item \textbf{Distributions régulières :} Toute fonction localement intégrable $f$ définit une distribution $T_f$ par $\langle T_f, \phi \rangle = \int f(x)\phi(x)dx$.
        \item \textbf{Distribution de Dirac :} $\langle \delta_a, \phi \rangle = \phi(a)$. Ce n'est pas une distribution régulière.
        \item \textbf{Distribution "valeur principale" :} Permet de donner un sens à l'intégrale de $1/x$.
    \end{itemize}
\end{example}

\begin{definition}[Dérivée d'une distribution]
    On définit la dérivée $T'$ d'une distribution $T$ par la formule (inspirée de l'intégration par parties) :
    $$ \langle T', \phi \rangle = - \langle T, \phi' \rangle $$
\end{definition}
\begin{remark}[La Révolution des Distributions]
    Cette définition est extraordinaire : toute distribution est infiniment dérivable. Les problèmes de régularité disparaissent. Par exemple, la fonction de Heaviside $H$ (non dérivable en 0) a pour dérivée au sens des distributions la distribution de Dirac $\delta_0$.
\end{remark}

\begin{proposition}[Transformée de Fourier des distributions]
    La transformée de Fourier s'étend aux distributions tempérées (duales de l'espace de Schwartz $\mathcal{S}(\mathbb{R})$).
\end{proposition}

\begin{example}
    \begin{itemize}
        \item $\mathcal{F}(1) = \delta_0$.
        \item $\mathcal{F}(\delta_0) = 1$.
        \item $\mathcal{F}(e^{2i\pi a x}) = \delta_a$.
    \end{itemize}
    Le peigne de Dirac $\sum_{n \in \mathbb{Z}} \delta_n$ a pour transformée de Fourier lui-même, c'est la base de la formule sommatoire de Poisson.
\end{example}

\chapter{Théorie de la Mesure : Refonder l'Intégrale sur des Bases Solides}

\section{Mesurer les Ensembles : Tribus et Mesures}

\begin{objectif}
    Construire une théorie rigoureuse de la "taille" (longueur, aire, volume, probabilité...) des ensembles. Confrontés à des ensembles très "exotiques", nous devons abandonner l'idée de pouvoir tout mesurer. L'objectif est de définir une classe d'ensembles "mesurables" (une tribu) suffisamment grande pour tous les besoins de l'analyse, et sur laquelle on peut définir une fonction "mesure" cohérente.
\end{objectif}

\begin{definition}[Tribu ou $\sigma$-algèbre]
    Une \textbf{tribu} (ou $\sigma$-algèbre) sur un ensemble $X$ est une collection $\mathcal{A}$ de parties de $X$ telle que :
    \begin{enumerate}
        \item $X \in \mathcal{A}$.
        \item $\mathcal{A}$ est stable par passage au complémentaire.
        \item $\mathcal{A}$ est stable par union \textbf{dénombrable}.
    \end{enumerate}
    Le couple $(X, \mathcal{A})$ est un \textbf{espace mesurable}.
\end{definition}
\begin{remark}[La Stabilité Dénombrable]
    L'hypothèse de stabilité par union \textit{dénombrable} (et non quelconque comme en topologie) est le cœur de la théorie. C'est le bon équilibre : assez forte pour permettre les passages à la limite, mais assez restrictive pour éviter les paradoxes de type Banach-Tarski qui émergent si on essaie de tout mesurer.
\end{remark}

\begin{definition}[Mesure]
    Une \textbf{mesure} sur un espace mesurable $(X, \mathcal{A})$ est une fonction $\mu: \mathcal{A} \to [0, \infty]$ telle que $\mu(\emptyset)=0$ et qui est \textbf{$\sigma$-additive} : pour toute suite $(A_n)$ d'ensembles de $\mathcal{A}$ deux à deux disjoints, $\mu(\cup_{n=1}^\infty A_n) = \sum_{n=1}^\infty \mu(A_n)$.
\end{definition}

\begin{theorem}[Construction de la mesure de Lebesgue (via Carathéodory)]
    Il existe une unique mesure $\lambda$ sur la tribu borélienne de $\mathbb{R}$ (la plus petite tribu contenant tous les ouverts) qui coïncide avec la longueur sur les intervalles. On peut ensuite compléter cette tribu pour obtenir la \textbf{tribu de Lebesgue}, qui contient tous les ensembles "utiles" à l'analyse, et bien plus.
\end{theorem}

\begin{example}[Ensembles de mesure nulle]
    \begin{itemize}
        \item Tout ensemble dénombrable (comme $\mathbb{Q}$) est de mesure de Lebesgue nulle.
        \item L'\textbf{ensemble de Cantor} est un exemple de compact non dénombrable, totalement discontinu, et de mesure de Lebesgue nulle. C'est un objet fractal qui montre que la notion de "taille" est subtile.
    \end{itemize}
\end{example}

\section{L'Intégrale de Lebesgue : Une Révolution Conceptuelle}

\begin{objectif}
    Construire une nouvelle intégrale, basée sur une philosophie radicalement différente de celle de Riemann. Au lieu de découper le domaine de définition (l'axe des abscisses), Lebesgue a l'idée de découper le domaine d'arrivée (l'axe des ordonnées).
\end{objectif}

\begin{definition}[Fonction mesurable]
    Une fonction $f: (X, \mathcal{A}) \to (Y, \mathcal{B})$ est mesurable si l'image réciproque de tout ensemble mesurable de $Y$ est un ensemble mesurable de $X$. C'est l'analogue de la continuité pour les espaces mesurables.
\end{definition}

\begin{proposition}[Construction de l'intégrale de Lebesgue]
    L'intégrale est construite en trois étapes :
    \begin{enumerate}
        \item \textbf{Pour les fonctions étagées positives} ($f = \sum a_i \mathbf{1}_{A_i}$) : $\int f d\mu = \sum a_i \mu(A_i)$.
        \item \textbf{Pour les fonctions mesurables positives} $f \ge 0$ : $\int f d\mu = \sup \{ \int s d\mu \mid s \text{ étagée, } 0 \le s \le f \}$.
        \item \textbf{Pour une fonction mesurable quelconque} $f$ : On la décompose en $f=f^+ - f^-$. Si $\int |f| d\mu < \infty$, on dit que $f$ est intégrable et on pose $\int f d\mu = \int f^+ d\mu - \int f^- d\mu$.
    \end{enumerate}
\end{proposition}

\begin{remark}[La Philosophie de Lebesgue vs. Riemann]
    Imaginez que vous devez payer une grande somme avec des pièces et des billets.
    \begin{itemize}
        \item \textbf{Riemann :} Il prend les pièces et billets dans l'ordre où ils sortent de votre portefeuille (découpage du domaine) et les additionne au fur et à mesure.
        \item \textbf{Lebesgue :} Il commence par trier toutes les pièces et tous les billets par valeur (découpage du codomaine). Il compte combien il a de pièces de 1€, de 2€, de billets de 5€, etc., puis il fait une seule grande somme : (nb de pièces de 1€)x1€ + (nb de billets de 5€)x5€...
    \end{itemize}
    L'approche de Lebesgue est plus sophistiquée, mais bien plus efficace pour des ensembles de valeurs "compliqués". [Image comparing Riemann and Lebesgue integration methods]
\end{remark}

\section{Les Théorèmes de Convergence : La Puissance Retrouvée}

\begin{objectif}
    Montrer le gain spectaculaire de cette nouvelle construction. Les problèmes d'interversion de limites et d'intégrales, qui étaient un cauchemar avec l'intégrale de Riemann, sont résolus par des théorèmes d'une élégance et d'une puissance remarquables.
\end{objectif}

\begin{theorem}[Théorème de Convergence Monotone (Beppo-Levi)]
    Soit $(f_n)$ une suite croissante de fonctions mesurables positives. Alors :
    $$ \int \lim_{n\to\infty} f_n d\mu = \lim_{n\to\infty} \int f_n d\mu $$
    L'interversion est toujours possible, même si la limite est infinie.
\end{theorem}

\begin{lemma}[Lemme de Fatou]
    Soit $(f_n)$ une suite de fonctions mesurables positives. Alors :
    $$ \int \liminf_{n\to\infty} f_n d\mu \le \liminf_{n\to\infty} \int f_n d\mu $$
\end{lemma}

\begin{theorem}[Théorème de Convergence Dominée de Lebesgue]
    Soit $(f_n)$ une suite de fonctions mesurables qui converge presque partout vers une fonction $f$. S'il existe une fonction \textbf{intégrable} $g$ (la "dominatrice") telle que $|f_n(x)| \le g(x)$ pour tout $n$ et presque tout $x$, alors $f$ est intégrable et :
    $$ \int f d\mu = \lim_{n\to\infty} \int f_n d\mu $$
\end{theorem}

\begin{remark}[La Clé de l'Analyse Moderne]
    Le théorème de convergence dominée est le "couteau suisse" de l'analyste. L'hypothèse de domination est souvent facile à vérifier et permet de justifier rigoureusement des passages à la limite qui seraient impossibles autrement. C'est l'un des théorèmes les plus utiles de toutes les mathématiques.
\end{remark}

\section{Le Lien avec l'Analyse Fonctionnelle et Fubini}

\begin{objectif}
    Connecter la théorie de la mesure à ses deux grandes applications : la construction des espaces fonctionnels modernes ($L^p$) et le calcul d'intégrales multiples (Fubini).
\end{objectif}

\begin{theorem}[Théorème de Riesz-Fischer]
    Les espaces de fonctions $L^p(\Omega, \mu)$ sont des espaces de Banach pour tout $p \in [1, \infty]$.
\end{theorem}
\begin{remark}[La Résolution du Problème de Complétude]
    C'est le résultat qui justifie a posteriori toute la construction. En changeant de théorie de l'intégration, on a non seulement gagné en généralité et en puissance, mais on a aussi "réparé" la structure des espaces de fonctions, les rendant complets et donc aptes à l'usage des outils de l'analyse fonctionnelle.
\end{remark}

\begin{theorem}[Théorèmes de Fubini]
    Soient $(X, \mathcal{A}, \mu)$ et $(Y, \mathcal{B}, \nu)$ deux espaces mesurés $\sigma$-finis.
    \begin{itemize}
        \item \textbf{(Fubini-Tonelli)} Si $f: X \times Y \to [0, \infty]$ est mesurable positive, alors les intégrales itérées et l'intégrale double sont égales (éventuellement infinies). On peut toujours intervertir l'ordre d'intégration.
        \item \textbf{(Fubini-Lebesgue)} Si $f: X \times Y \to \mathbb{C}$ est \textbf{intégrable} pour la mesure produit, alors les intégrales itérées existent et sont égales à l'intégrale double.
    \end{itemize}
\end{theorem}

\begin{application}[Calcul de l'intégrale de Gauss]
    On veut calculer $I = \int_{-\infty}^\infty e^{-x^2} dx$. On calcule $I^2$ :
    $$ I^2 = \left( \int_{-\infty}^\infty e^{-x^2} dx \right) \left( \int_{-\infty}^\infty e^{-y^2} dy \right) = \int_{\mathbb{R}^2} e^{-(x^2+y^2)} dx dy $$
    La fonction $e^{-(x^2+y^2)}$ est positive, donc on peut appliquer Fubini-Tonelli et passer en coordonnées polaires :
    $$ I^2 = \int_0^{2\pi} \int_0^\infty e^{-r^2} r dr d\theta = 2\pi \left[ -\frac{1}{2} e^{-r^2} \right]_0^\infty = \pi $$
    D'où $I = \sqrt{\pi}$. Le passage en polaires est rigoureusement justifié par Fubini.
\end{application}

\part{Analyse}
\chapter{Séries : L'Art de Dompter l'Infini}

\section{Séries Numériques : Le Sens d'une Somme Infinie}

\begin{objectif}
    Donner un sens mathématique rigoureux à l'idée intuitive de "somme infinie". On développe une boîte à outils de critères de convergence pour déterminer si une telle somme est un nombre bien défini ou n'a pas de sens. La distinction entre convergence simple et convergence absolue est la première subtilité fondamentale.
\end{objectif}

\begin{definition}[Convergence et Nature d'une série]
    Une série $\sum u_n$ converge si la suite de ses sommes partielles $S_N = \sum_{n=0}^N u_n$ converge. Elle converge \textbf{absolument} si la série $\sum |u_n|$ converge.
\end{definition}

\begin{proposition}[Convergence absolue $\implies$ Convergence]
    La réciproque est fausse, ce qui donne lieu aux séries semi-convergentes.
\end{proposition}

\begin{example}[La série harmonique alternée]
    La série $\sum_{n=1}^\infty \frac{(-1)^{n+1}}{n} = 1 - \frac{1}{2} + \frac{1}{3} - \dots$ converge (vers $\ln(2)$), mais ne converge pas absolument (la série harmonique $\sum \frac{1}{n}$ diverge).
\end{example}

\begin{theorem}[Critères de convergence pour les séries à termes positifs]
    \begin{itemize}
        \item \textbf{Comparaison :} Comparaison à une série de référence (Riemann, géométrique).
        \item \textbf{Règle de d'Alembert :} Basée sur la limite du rapport $|u_{n+1}/u_n|$.
        \item \textbf{Règle de Cauchy :} Basée sur la limite de $\sqrt[n]{|u_n|}$.
        \item \textbf{Comparaison série-intégrale :} Permet d'étudier les séries de Riemann $\sum 1/n^\alpha$.
    \end{itemize}
\end{theorem}

\begin{application}[Constante d'Euler-Mascheroni]
    En étudiant la différence entre la série harmonique et l'intégrale de $1/t$, on montre l'existence de la constante d'Euler-Mascheroni $\gamma = \lim_{N \to \infty} \left( \sum_{n=1}^N \frac{1}{n} - \ln(N) \right)$.
\end{application}

\section{Séries de Fonctions : Construire de Nouveaux Objets}

\begin{objectif}
    Passer de la somme de nombres à la somme de fonctions. La question centrale devient : si une suite de fonctions $f_n$ possède une certaine propriété (continuité, dérivabilité, intégrabilité), la fonction somme $F = \sum f_n$ hérite-t-elle de cette propriété ? Cela dépend crucialement du \textbf{mode de convergence}.
\end{objectif}

\begin{definition}[Modes de convergence]
    Soit une suite de fonctions $(f_n)$ sur un ensemble $X$.
    \begin{itemize}
        \item \textbf{Convergence simple :} Pour chaque $x \in X$, la suite numérique $(f_n(x))$ converge.
        \item \textbf{Convergence uniforme :} $\sup_{x \in X} |f_n(x) - f(x)| \to 0$. C'est une convergence globale.
        \item \textbf{Convergence normale (pour les séries) :} La série numérique $\sum \sup_X |f_n(x)|$ converge. C'est un critère pratique qui implique la convergence uniforme.
    \end{itemize}
\end{definition}

\begin{remark}[La force de la convergence uniforme]
    La convergence simple est une notion faible et souvent pathologique. La convergence uniforme est la "bonne" notion qui permet de transférer les propriétés des termes de la série à la fonction somme. Elle assure que les graphes des fonctions $S_N$ se rapprochent "globalement" du graphe de $S$.
\end{remark}

\begin{theorem}[Théorèmes de transfert]
    Si la série de fonctions $\sum f_n$ converge \textbf{uniformément} vers $S$ :
    \begin{itemize}
        \item Si les $f_n$ sont continues, alors $S$ est continue.
        \item On peut intervertir série et intégrale : $\int_a^b S(x) dx = \sum_{n=0}^\infty \int_a^b f_n(x) dx$.
        \item Si de plus les $f_n$ sont $\mathcal{C}^1$ et que la série des dérivées $\sum f'_n$ converge uniformément, on peut dériver terme à terme : $S' = \sum f'_n$.
    \end{itemize}
\end{theorem}

\begin{example}[Construction de fonctions pathologiques]
    On peut utiliser des séries de fonctions pour construire des objets contre-intuitifs. L'exemple le plus célèbre est la \textbf{fonction de Weierstrass}, $f(x) = \sum_{n=0}^\infty a^n \cos(b^n \pi x)$, qui est continue sur $\mathbb{R}$ mais dérivable en aucun point.
\end{example}

\section{Séries Entières : Le Pont entre Algèbre et Analyse}

\begin{objectif}
    Étudier les "polynômes de degré infini". Les séries entières sont des objets remarquables qui jouissent d'une régularité exceptionnelle à l'intérieur de leur disque de convergence. Elles sont le pont entre les fonctions analytiques et l'algèbre des polynômes.
\end{objectif}

\begin{definition}[Série entière et Rayon de convergence]
    Une série entière est une série de fonctions de la forme $\sum a_n z^n$. Il existe un $R \in [0, \infty]$, le \textbf{rayon de convergence}, tel que la série converge absolument pour $|z|<R$ et diverge pour $|z|>R$.
\end{definition}

\begin{theorem}[Régularité de la somme]
    La somme d'une série entière est une fonction \textbf{continue} et même \textbf{infiniment dérivable} (ou holomorphe) à l'intérieur de son disque de convergence. On peut dériver et intégrer terme à terme.
\end{theorem}

\begin{remark}[La rigidité analytique]
    Ce théorème est spectaculaire. Une série entière se comporte comme un polynôme : on peut la manipuler terme à terme. La connaissance des coefficients $\{a_n\}$ (une information "algébrique") détermine entièrement la fonction somme et toutes ses dérivées (une information "analytique").
\end{remark}

\begin{example}[Développements en série entière usuels]
    \begin{itemize}
        \item $e^z = \sum_{n=0}^\infty \frac{z^n}{n!}$ pour $z \in \mathbb{C}$. ($R=\infty$)
        \item $\frac{1}{1-z} = \sum_{n=0}^\infty z^n$ pour $|z|<1$. ($R=1$)
        \item $\ln(1+x) = \sum_{n=1}^\infty \frac{(-1)^{n+1}x^n}{n}$ pour $|x|<1$. ($R=1$)
    \end{itemize}
\end{example}

\begin{application}[Résolution d'équations différentielles]
    On peut chercher des solutions d'EDO sous forme de séries entières. Par exemple, pour l'équation d'Airy $y'' - xy = 0$, on suppose $y(x) = \sum a_n x^n$. On injecte dans l'équation, et on trouve une relation de récurrence sur les coefficients $a_n$, ce qui permet de définir les fonctions d'Airy.
\end{application}

\section{Séries de Fourier : La Décomposition Harmonique}

\begin{objectif}
    Changer de point de vue : au lieu de "construire" une fonction, on cherche à la "décomposer" en une somme de fonctions de base simples et oscillantes (les sinus et cosinus). C'est le passage de la base polynomiale $\{x^n\}$ (pour les séries entières) à la base trigonométrique $\{e^{inx}\}$.
\end{objectif}

\begin{definition}[Coefficients et Série de Fourier]
    Pour une fonction $f$ $2\pi$-périodique et intégrable, ses coefficients de Fourier sont $c_n(f) = \frac{1}{2\pi}\int_0^{2\pi} f(t)e^{-int}dt$. Sa série de Fourier est $S(f)(x) = \sum_{n=-\infty}^\infty c_n(f) e^{inx}$.
\end{definition}

\begin{theorem}[Théorème de Convergence de Dirichlet]
    Si $f$ est $2\pi$-périodique et $\mathcal{C}^1$ par morceaux, alors sa série de Fourier converge en tout point $x$ vers $\frac{1}{2}(f(x^+) + f(x^-))$.
\end{theorem}

\begin{theorem}[Identité de Parseval]
    Si $f$ est de carré intégrable, alors sa série de Fourier converge vers $f$ au sens de la norme $L^2$, et on a :
    $$ \frac{1}{2\pi} \int_0^{2\pi} |f(t)|^2 dt = \sum_{n=-\infty}^\infty |c_n(f)|^2 $$
\end{theorem}

\begin{remark}[La vision Hilbertienne]
    Ce théorème est la traduction du théorème de Pythagore dans l'espace de Hilbert $L^2([0,2\pi])$. Il signifie que la famille des $\{e^{inx}\}$ (normalisée) forme une base orthonormale de cet espace. L'énergie totale du signal ($L^2$ norm) est la somme des énergies de ses composantes harmoniques.
\end{remark}

\begin{example}[Décomposition de signaux simples]
    \begin{itemize}
        \item Une fonction "créneau" (égale à 1 sur $[0, \pi]$ et -1 sur $[\pi, 2\pi]$) se décompose en une somme infinie de sinus d'harmoniques impaires.
        \item Une fonction "dents de scie" se décompose en une somme de sinus.
    \end{itemize}
\end{example}

\begin{application}[Résolution de l'équation de la chaleur]
    La méthode de séparation des variables pour l'équation de la chaleur $\frac{\partial u}{\partial t} = \frac{\partial^2 u}{\partial x^2}$ sur un segment mène naturellement à une décomposition de la solution en série de Fourier. Les conditions aux limites sélectionnent la base (sinus ou cosinus), et l'équation différentielle transforme les coefficients de Fourier de la condition initiale $u(0,x)$ en $c_n(t) = c_n(0)e^{-n^2 t}$.
\end{application}

\begin{application}[Calcul de sommes de séries numériques]
    En appliquant la formule de Parseval à la série de Fourier de fonctions simples, on peut calculer des sommes célèbres. Par exemple, avec la fonction $f(x)=x$ sur $[-\pi, \pi]$, Parseval donne :
    $$ \sum_{n=1}^\infty \frac{1}{n^2} = \frac{\pi^2}{6} $$
    C'est le problème de Bâle, résolu par Euler.
\end{application}
\chapter{Analyse Fonctionnelle : Une Perspective Géométrique}

\section{Préambule : Fondamentaux de Topologie et des Espaces Métriques}

\begin{objectif}
    Établir le langage commun et les concepts fondamentaux sur lesquels repose toute l'analyse fonctionnelle. L'idée est de partir du cadre le plus général (la topologie) pour arriver progressivement à la structure riche et concrète des espaces vectoriels normés. Ce préambule sert de fondation et de référence pour toutes les notions ultérieures.
\end{objectif}

\subsection{Topologie Générale : Le Langage de la Proximité}

\begin{definition}[Espace topologique]
    Un espace topologique est un couple $(X, \mathcal{T})$ où $X$ est un ensemble et $\mathcal{T}$ est une collection de parties de $X$ (appelées \textbf{ouverts}) telle que :
    \begin{enumerate}
        \item $\emptyset \in \mathcal{T}$ et $X \in \mathcal{T}$.
        \item Toute union (finie ou infinie) d'ouverts est un ouvert.
        \item Toute intersection finie d'ouverts est un ouvert.
    \end{enumerate}
    Une partie $F \subset X$ est dite \textbf{fermée} si son complémentaire $X \setminus F$ est un ouvert.
\end{definition}

\begin{definition}[Continuité]
    Une application $f: (X, \mathcal{T}_X) \to (Y, \mathcal{T}_Y)$ entre deux espaces topologiques est \textbf{continue} si l'image réciproque de tout ouvert de $Y$ est un ouvert de $X$.
\end{definition}

\begin{definition}[Compacité]
    Une partie $K$ d'un espace topologique $X$ est \textbf{compacte} si de tout recouvrement de $K$ par une famille d'ouverts, on peut extraire un sous-recouvrement fini.
\end{definition}

\begin{remark}[La Propriété Fondamentale]
    La compacité est la notion la plus importante de ce préambule. C'est l'analogue topologique de la "dimension finie". Elle garantit l'existence de limites pour les suites (ou suites généralisées) et de points extrémaux pour les fonctions continues.
    \begin{itemize}
        \item L'image d'un compact par une application continue est compacte.
        \item Dans un espace séparé (Hausdorff), un compact est toujours fermé.
        \item Un fermé dans un compact est compact.
    \end{itemize}
\end{remark}

\begin{definition}[Connexité]
    Un espace topologique $X$ est \textbf{connexe} s'il n'est pas la réunion de deux ouverts non vides disjoints. Intuitivement, un espace "d'un seul tenant".
\end{definition}

\subsection{Espaces Métriques : La Notion de Distance}

\begin{objectif}
    Spécialiser le cadre topologique en introduisant une fonction "distance". Cela permet de quantifier la proximité et de définir des notions plus concrètes comme les suites de Cauchy et la complétude, qui sont la clé de voûte de l'analyse.
\end{objectif}

\begin{definition}[Espace métrique]
    Un espace métrique est un couple $(X, d)$ où $X$ est un ensemble et $d: X \times X \to \mathbb{R}_+$ est une distance, vérifiant la séparation, la symétrie et l'inégalité triangulaire.
\end{definition}

\begin{remark}[Topologie Induite]
    Toute distance $d$ induit une topologie sur $X$ dont les ouverts sont les réunions de boules ouvertes $B(x, r) = \{y \in X \mid d(x,y) < r\}$. C'est le cadre naturel de l'analyse "classique".
\end{remark}

\begin{definition}[Suite de Cauchy]
    Une suite $(x_n)_{n \in \mathbb{N}}$ dans un espace métrique $(X,d)$ est une \textbf{suite de Cauchy} si pour tout $\epsilon > 0$, il existe un rang $N \in \mathbb{N}$ tel que pour tous $p, q \geq N$, on a $d(x_p, x_q) < \epsilon$.
\end{definition}

\begin{remark}[L'Idée d'une "Convergence Intrinsèque"]
    Une suite de Cauchy est une suite dont les termes se rapprochent indéfiniment les uns des autres. C'est une suite qui "veut converger", sans préjuger de l'existence d'une limite dans l'espace. La notion est intrinsèque à la suite, contrairement à la convergence qui dépend de l'espace ambiant.
\end{remark}

\begin{definition}[Espace complet]
    Un espace métrique $(X,d)$ est \textbf{complet} si toute suite de Cauchy de $X$ converge vers un élément de $X$.
\end{definition}

\begin{remark}[La Propriété Clé de l'Analyse]
    La complétude est la propriété qui nous autorise à passer à la limite. C'est la garantie que l'espace n'a pas de "trous". $\mathbb{Q}$ n'est pas complet, mais $\mathbb{R}$ l'est. Cette propriété est le pont conceptuel entre les espaces métriques généraux et les espaces de Banach.
\end{remark}

\begin{theorem}[Caractérisation des compacts dans un métrique]
    Une partie d'un espace métrique est compacte si et seulement si elle est précompacte (ou totalement bornée) et complète. Dans $\mathbb{R}^n$, cela équivaut à "fermé et borné" (Théorème de Borel-Lebesgue).
\end{theorem}

\subsection{Espaces Vectoriels Normés : La Synthèse}

\begin{objectif}
    Combiner la structure algébrique d'un espace vectoriel avec la structure topologique d'un espace métrique, en s'assurant que les deux sont compatibles. C'est le point de départ effectif de l'analyse fonctionnelle.
\end{objectif}

\begin{definition}[Espace vectoriel normé]
    Un espace vectoriel normé $(E, \|\cdot\|)$ est un espace vectoriel $E$ muni d'une norme $\|\cdot\|$, qui induit une distance $d(x,y) = \|x-y\|$. Cette distance est compatible avec la structure vectorielle (i.e., l'addition et la multiplication par un scalaire sont continues).
\end{definition}

\begin{remark}[Le Lien Final]
    Un \textbf{espace de Banach} est un espace vectoriel normé qui est \textbf{complet} en tant qu'espace métrique.
    Un \textbf{espace de Hilbert} est un espace de Banach dont la norme dérive d'un \textbf{produit scalaire}.
    Ce préambule a donc posé toutes les briques logiques pour comprendre ces définitions fondamentales.
\end{remark}


\section{Espaces de Banach : La Scène de l'Analyse}

\begin{objectif}
    Poser le décor. L'analyse a besoin de limites pour exister. Le cadre minimal pour cela est un espace vectoriel normé qui soit \textbf{complet} : un espace de Banach. Nous verrons immédiatement que ce gain analytique (la complétude) se paie par une perte géométrique majeure : la compacité des ensembles bornés, qui est la propriété fondamentale des espaces de dimension finie. Toute la suite de ce cours peut être vue comme une quête pour retrouver cette compacité perdue.
\end{objectif}

\begin{definition}[Espace de Banach]
    Un espace vectoriel normé $(E, \|\cdot\|)$ est un espace de Banach s'il est complet pour la métrique induite par la norme, c'est-à-dire si toute suite de Cauchy y est convergente.
\end{definition}

\begin{theorem}[Lemme de Riesz]
    Soit $E$ un e.v.n. et $F$ un sous-espace vectoriel fermé strict de $E$. Alors pour tout $\epsilon \in ]0,1[$, il existe $x_\epsilon \in E$ tel que $\|x_\epsilon\|=1$ et $d(x_\epsilon, F) \geq 1-\epsilon$.
\end{theorem}

\begin{remark}[La Compacité, Analogue de la "Dimension Finie"]
    Ce lemme, d'apparence technique, est le théorème des "mauvaises nouvelles". Il est l'outil qui prouve que la boule unité fermée d'un e.v.n. de dimension infinie n'est \textbf{jamais} compacte.
    Pourquoi est-ce si grave ? Parce que la compacité est la propriété qui garantit qu'on peut extraire des sous-suites convergentes de toute suite bornée. C'est le moteur de tous les théorèmes d'existence en dimension finie. En analyse fonctionnelle, la compacité devient donc une propriété rare et précieuse. La traquer, c'est essayer de se ramener, par un moyen détourné, à une situation aussi "confortable" et "rigide" que la dimension finie.
\end{remark}

\begin{definition}[Norme subordonnée]
    Soient $(E, \|\cdot\|_E)$ et $(F, \|\cdot\|_F)$ deux e.v.n. La norme subordonnée d'une application linéaire continue $u \in \mathcal{L}(E,F)$ est :
    $$ \|u\|_{\mathcal{L}(E,F)} = \sup_{\|x\|_E=1} \|u(x)\|_F $$
    Muni de cette norme, si $F$ est un espace de Banach, alors $\mathcal{L}(E,F)$ l'est aussi.
\end{definition}

\section{Les Piliers : La Puissance de la Complétude}

\begin{objectif}
    Démontrer que si l'on paie le prix de la non-compacité, l'hypothèse de complétude (via le lemme de Baire) nous offre en retour des outils d'une puissance extraordinaire. Ces théorèmes sont les "machines" de l'analyse fonctionnelle, permettant de construire des objets, de prouver des convergences et de simplifier des problèmes.
\end{objectif}

\begin{theorem}[Hahn-Banach, forme analytique et géométrique]
    \textbf{Analytique :} Toute forme linéaire continue sur un s.e.v. peut être prolongée à l'espace entier en préservant sa norme.
    \textbf{Géométrique :} Deux convexes disjoints, dont l'un est ouvert, peuvent être séparés par un hyperplan affine fermé.
\end{theorem}

\begin{application}[Le dual "voit" toute la géométrie]
    La conséquence la plus importante de Hahn-Banach est que le dual $E^*$ est "riche" : il contient suffisamment de formes linéaires pour distinguer tous les points de $E$. Géométriquement, cela signifie que tout ensemble convexe fermé peut être vu comme une intersection d'demi-espaces. Le dual encode donc toute la structure convexe de l'espace.
\end{application}

\begin{theorem}[Banach-Steinhaus et le principe de la borne uniforme]
    Une famille d'opérateurs linéaires continus entre un Banach et un e.v.n. qui est simplement bornée est uniformément bornée.
\end{theorem}

\begin{theorem}[Théorème de l'application ouverte et de l'isomorphisme de Banach]
    Toute application linéaire continue surjective entre deux espaces de Banach est ouverte. En conséquence, toute application linéaire continue bijective entre deux Banach est un homéomorphisme.
\end{theorem}

\begin{theorem}[Théorème du graphe fermé]
    Une application linéaire entre deux espaces de Banach est continue si et seulement si son graphe est fermé.
\end{theorem}



\section{Dualité et Topologies Faibles : La Quête de la Compacité}

\begin{objectif}
    Ici, nous abordons frontalement le problème de la non-compacité. L'idée est un compromis fondamental : si l'on ne peut pas avoir la compacité pour la topologie "forte" (celle de la norme), peut-on l'obtenir en affaiblissant la notion de convergence ? C'est le rôle des topologies faibles, qui sont le cadre naturel pour de nombreux problèmes variationnels.
\end{objectif}

\begin{definition}[Dual, Bidual et Réflexivité]
    Le dual topologique de $E$ est $E^* = \mathcal{L}(E, \mathbb{K})$. Un espace de Banach $E$ est dit réflexif si l'injection canonique $J: E \to E^{**}$ est un isomorphisme.
\end{definition}

\begin{definition}[Topologies Faible et Faible-*]
    La topologie faible $\sigma(E, E^*)$ sur $E$ est la topologie initiale associée à la famille des formes linéaires de $E^*$.
    La topologie faible-* $\sigma(E^*, E)$ sur $E^*$ est la topologie initiale associée à la famille des évaluations $\{J(x) \mid x \in E\}$.
\end{definition}

\begin{theorem}[Théorème de Banach-Alaoglu]
    La boule unité fermée du dual $E^*$ d'un e.v.n. $E$ est compacte pour la topologie faible-*.
\end{theorem}

\begin{remark}[La Compacité Retrouvée]
    C'est le résultat central qui répond à notre quête. Il nous offre la compacité sur un plateau, mais à un prix : celui de la topologie. On sacrifie la convergence forte (en norme) pour une convergence plus subtile, mais on regagne l'outil essentiel d'extraction de sous-suites. Pour un espace réflexif (comme un Hilbert), la boule unité est elle-même faiblement compacte. C'est ce qui rend ces espaces si "confortables".
\end{remark}



\section{Espaces de Hilbert : La Géométrie Triomphante}

\begin{objectif}
    Étudier le cadre "parfait" où l'analyse et la géométrie fusionnent. La présence d'un produit scalaire induit une notion d'orthogonalité qui rigidifie l'espace et simplifie drastiquement la théorie. Les projections deviennent des outils d'approximation, le dual s'identifie à l'espace lui-même, et les opérateurs les plus importants deviennent "diagonalisables".
\end{objectif}

\begin{theorem}[Théorème de projection sur un convexe fermé]
    Dans un Hilbert, tout convexe fermé non vide est un "ensemble de meilleure approximation" : pour tout point de l'espace, il existe un unique point du convexe qui minimise la distance.
\end{theorem}

\begin{application}[Optimisation Convexe et Machine Learning]
    Ce théorème est le fondement de nombreux algorithmes. Par exemple, les "Support Vector Machines" (SVM) en classification consistent à trouver l'hyperplan qui sépare au mieux deux ensembles de points. Ce problème se ramène à la projection d'un point sur un convexe fermé.
\end{application}

\begin{theorem}[Théorème de représentation de Riesz]
    Tout espace de Hilbert s'identifie canoniquement (via une isométrie antilinéaire) à son dual. C'est l'archétype d'un espace réflexif.
\end{theorem}

\begin{theorem}[Alternative de Fredholm]
    Soit $T$ un opérateur compact sur un Hilbert $H$. Alors l'équation $(I-T)x=y$ se comporte exactement comme un système linéaire en dimension finie : elle a une solution unique pour tout $y$ si et seulement si l'équation homogène $(I-T)x=0$ n'a que la solution triviale.
\end{theorem}

\begin{theorem}[Théorème Spectral pour les opérateurs compacts auto-adjoints]
    Soit $T$ un opérateur compact auto-adjoint sur un Hilbert $H$. Alors il existe une base hilbertienne de $H$ formée de vecteurs propres de $T$.
\end{theorem}

\begin{remark}[La "Diagonalisation" en Dimension Infinie]
    C'est l'aboutissement de notre quête. Dans la "bonne" base, un opérateur complexe se comporte comme une simple multiplication par des scalaires. Cela permet de définir des fonctions d'opérateurs (comme $e^T$) et de résoudre des équations d'évolution (comme l'équation de la chaleur ou de Schrödinger) par décomposition sur cette base. L'opérateur différentiel (non borné) est "dompté" par un opérateur intégral inverse (compact), dont on peut appliquer le théorème spectral. C'est le triomphe de l'analyse fonctionnelle.
\end{remark}
 % Banach, Hilbert, convergence
\chapter{Analyse Complexe : La Rigidité Miraculeuse du Plan Complexe}

\section{Le Monde Holomorphe : Une Condition Extraordinaire}

\begin{objectif}
    Introduire la notion de dérivabilité complexe (holomorphie) et montrer qu'elle est infiniment plus contraignante que la dérivabilité réelle. C'est une condition locale d'une rigidité géométrique spectaculaire : elle force une fonction à être localement une similitude. Cette rigidité est la source de tous les "miracles" de l'analyse complexe.
\end{objectif}

\begin{definition}[Dérivabilité Complexe (Holomorphie)]
    Une fonction $f: U \to \mathbb{C}$ (où $U$ est un ouvert de $\mathbb{C}$) est \textbf{holomorphe} en $z_0 \in U$ si la limite du taux d'accroissement $\frac{f(z)-f(z_0)}{z-z_0}$ existe quand $z \to z_0$.
\end{definition}

\begin{theorem}[Équations de Cauchy-Riemann]
    Soit $f(z) = f(x+iy) = P(x,y) + iQ(x,y)$. La fonction $f$ est holomorphe sur $U$ si et seulement si ses parties réelle $P$ et imaginaire $Q$ sont de classe $\mathcal{C}^1$ et vérifient les équations de Cauchy-Riemann :
    $$ \frac{\partial P}{\partial x} = \frac{\partial Q}{\partial y} \quad \text{et} \quad \frac{\partial P}{\partial y} = -\frac{\partial Q}{\partial x} $$
\end{theorem}

\begin{remark}[La Contrainte Géométrique]
    Ces équations ne sont pas anodines. Elles impliquent que le Jacobien de $f$ (vue comme une application de $\mathbb{R}^2$ dans $\mathbb{R}^2$) est la matrice d'une similitude directe. Cela signifie qu'une fonction holomorphe préserve les angles en tout point où sa dérivée n'est pas nulle. C'est une transformation conforme.
\end{remark}

\begin{example}[Fonctions holomorphes et non-holomorphes]
    \begin{itemize}
        \item $f(z) = z^n$ est holomorphe sur $\mathbb{C}$ (entière).
        \item $f(z) = e^z = e^x(\cos y + i\sin y)$ est entière.
        \item $f(z) = 1/z$ est holomorphe sur $\mathbb{C}^*$.
        \item $f(z) = \bar{z} = x - iy$ n'est holomorphe en aucun point. Elle est $\mathbb{R}$-différentiable mais pas $\mathbb{C}$-différentiable.
        \item $f(z) = |z|^2 = z\bar{z}$ n'est holomorphe qu'en $z=0$.
    \end{itemize}
\end{example}

\section{L'Intégration Complexe et la Formule de Cauchy}

\begin{objectif}
    Développer l'outil principal de la théorie : l'intégration le long de chemins. Le théorème de Cauchy va révéler un premier miracle : l'intégrale d'une fonction holomorphe sur un lacet ne dépend que de la "topologie" du lacet. La formule de Cauchy en sera la conséquence spectaculaire, reliant les valeurs à l'intérieur d'un domaine à celles sur sa frontière.
\end{objectif}

\begin{definition}[Intégrale le long d'un chemin]
    Soit $\gamma: [a,b] \to U$ un chemin $\mathcal{C}^1$ et $f: U \to \mathbb{C}$ une fonction continue. On définit $\int_\gamma f(z) dz = \int_a^b f(\gamma(t))\gamma'(t) dt$.
\end{definition}

\begin{theorem}[Théorème intégral de Cauchy]
    Soit $U$ un ouvert étoilé de $\mathbb{C}$ et $f: U \to \mathbb{C}$ une fonction holomorphe. Alors, pour tout lacet (chemin fermé) $\gamma$ dans $U$, on a :
    $$ \oint_\gamma f(z) dz = 0 $$
\end{theorem}

\begin{theorem}[Formule intégrale de Cauchy]
    Sous les mêmes hypothèses, pour tout $z_0 \in U$ et pour tout lacet $\gamma$ simple entourant $z_0$, on a :
    $$ f(z_0) = \frac{1}{2i\pi} \oint_\gamma \frac{f(z)}{z-z_0} dz $$
\end{theorem}

\begin{remark}[Le Principe "Local-Global"]
    Cette formule est le cœur de la théorie. Elle est extraordinaire : la valeur d'une fonction holomorphe en un point est entièrement déterminée par ses valeurs sur un lacet qui l'entoure, aussi loin soit-il. L'information locale (en $z_0$) est encodée par une information globale (sur $\gamma$). C'est la première manifestation de la rigidité holomorphe.
\end{remark}

\section{Les Conséquences de la Rigidité Holomorphe}

\begin{objectif}
    Explorer les conséquences incroyables de la formule de Cauchy. On va voir qu'être une seule fois dérivable au sens complexe implique d'être infiniment dérivable, développable en série entière, et d'obéir à des principes très forts comme celui du maximum ou de Liouville.
\end{objectif}

\begin{corollary}[Développabilité en série entière]
    Toute fonction holomorphe sur un disque ouvert est développable en série entière sur ce disque. Une fonction holomorphe est donc analytique.
\end{corollary}

\begin{theorem}[Inégalités de Cauchy et Théorème de Liouville]
    La formule de Cauchy permet de majorer les dérivées successives. Une conséquence est le \textbf{théorème de Liouville} : toute fonction entière (holomorphe sur $\mathbb{C}$) et bornée est constante.
\end{theorem}

\begin{application}[Théorème Fondamental de l'Algèbre (d'Alembert-Gauss)]
    Soit $P$ un polynôme non constant. Si $P$ n'a pas de racine dans $\mathbb{C}$, alors $1/P$ est une fonction entière. Comme $|P(z)| \to \infty$ quand $|z| \to \infty$, $1/P$ est bornée. D'après Liouville, $1/P$ est constante, donc $P$ est constant, ce qui est une contradiction.
\end{application}

\begin{theorem}[Principe du Maximum]
    Le module d'une fonction holomorphe non constante sur un domaine ne peut pas atteindre son maximum à l'intérieur de ce domaine. Le maximum est nécessairement sur la frontière.
\end{theorem}

\begin{theorem}[Principe des zéros isolés]
    Les zéros d'une fonction holomorphe non nulle sont isolés.
\end{theorem}

\begin{corollary}[Principe du prolongement analytique]
    Si deux fonctions holomorphes sur un domaine coïncident sur un ensemble de points ayant un point d'accumulation dans ce domaine, alors elles sont égales sur tout le domaine.
\end{corollary}

\section{Analyse des Singularités et Théorème des Résidus}

\begin{objectif}
    Classifier les points où une fonction cesse d'être holomorphe et utiliser cette classification pour développer un outil de calcul d'intégrales d'une puissance phénoménale : le théorème des résidus.
\end{objectif}

\begin{theorem}[Séries de Laurent]
    Toute fonction holomorphe sur une couronne $C(z_0, r, R)$ admet un unique développement en série de Laurent : $f(z) = \sum_{n=-\infty}^\infty a_n (z-z_0)^n$.
\end{theorem}

\begin{definition}[Classification des singularités isolées]
    Soit $z_0$ une singularité isolée de $f$.
    \begin{itemize}
        \item \textbf{Singularité apparente (effaçable) :} La partie principale de la série de Laurent est nulle. $f$ est prolongeable par continuité.
        \item \textbf{Pôle d'ordre $k$ :} La partie principale est finie et le premier coefficient non nul est $a_{-k}$. $|f(z)| \to \infty$ quand $z \to z_0$.
        \item \textbf{Singularité essentielle :} La partie principale a une infinité de termes non nuls. Le comportement de $f$ au voisinage de $z_0$ est "chaotique".
    \end{itemize}
\end{definition}

\begin{example}[Types de singularités]
    \begin{itemize}
        \item $f(z) = \frac{\sin(z)}{z}$ a une singularité apparente en 0.
        \item $f(z) = \frac{e^z}{(z-1)^3}$ a un pôle d'ordre 3 en 1.
        \item $f(z) = e^{1/z}$ a une singularité essentielle en 0. D'après le théorème de Casorati-Weierstrass, l'image de toute boule épointée de centre 0 est dense dans $\mathbb{C}$.
    \end{itemize}
\end{example}

\begin{definition}[Résidu]
    Le \textbf{résidu} de $f$ en une singularité isolée $z_0$, noté $\mathrm{Res}(f, z_0)$, est le coefficient $a_{-1}$ de son développement en série de Laurent.
\end{definition}

\begin{theorem}[Théorème des Résidus]
    Soit $f$ une fonction holomorphe sur un domaine $U$ sauf en un nombre fini de singularités isolées $z_k$. Pour tout lacet simple $\gamma$ dans $U$ n'entourant que ces singularités, on a :
    $$ \oint_\gamma f(z) dz = 2i\pi \sum_k \mathrm{Res}(f, z_k) $$
\end{theorem}

\begin{application}[Calcul d'intégrales réelles]
    C'est l'application la plus célèbre.
    \begin{itemize}
        \item \textbf{Fractions rationnelles :} $\int_{-\infty}^{+\infty} \frac{dx}{1+x^4}$. On intègre sur un demi-cercle dans le demi-plan supérieur et on fait tendre le rayon vers l'infini.
        \item \textbf{Intégrales trigonométriques :} $\int_0^{2\pi} \frac{d\theta}{2+\cos\theta}$. On pose $z=e^{i\theta}$, l'intégrale devient une intégrale sur le cercle unité dans le plan complexe.
        \item \textbf{Transformées de Fourier :} $\int_{-\infty}^{+\infty} \frac{\cos(x)}{1+x^2} dx$. On utilise le lemme de Jordan.
        \item \textbf{Sommation de séries :} On peut montrer que $\sum_{n=1}^\infty \frac{1}{n^2} = \frac{\pi^2}{6}$ en intégrant la fonction $f(z) = \frac{\pi \cot(\pi z)}{z^2}$ sur un grand carré.
    \end{itemize}
\end{application}
\chapter{Analyse Convexe : La Géométrie de l'Optimisation}

\section{Le Langage de la Convexité}

\begin{objectif}
    Introduire les objets fondamentaux de l'analyse convexe. La convexité est une propriété géométrique extraordinairement puissante qui garantit que les minima locaux sont des minima globaux, et qui assure l'existence et l'unicité de solutions à de nombreux problèmes d'approximation.
\end{objectif}

\begin{definition}[Ensemble Convexe]
    Une partie $C$ d'un espace vectoriel $E$ est \textbf{convexe} si pour tous $x,y \in C$, le segment $[x,y] = \{tx + (1-t)y \mid t \in [0,1]\}$ est inclus dans $C$.
\end{definition}

\begin{definition}[Enveloppe Convexe]
    L'enveloppe convexe d'une partie $A$, notée $\mathrm{conv}(A)$, est le plus petit ensemble convexe contenant $A$.
\end{definition}

\begin{theorem}[Théorème de Carathéodory]
    Dans un espace de dimension $n$, tout point de l'enveloppe convexe d'un ensemble $A$ peut s'écrire comme une combinaison convexe de au plus $n+1$ points de $A$.
\end{theorem}

\begin{definition}[Fonction Convexe]
    Soit $C$ un convexe. Une fonction $f: C \to \mathbb{R}$ est \textbf{convexe} si pour tous $x,y \in C$ et $t \in [0,1]$ :
    $$ f(tx + (1-t)y) \le t f(x) + (1-t)f(y) $$
    Géométriquement, la corde est au-dessus de la courbe. L'épigraphe de $f$ est un ensemble convexe.
\end{definition}

\begin{proposition}[Caractérisation différentielle]
    Si $f$ est de classe $\mathcal{C}^1$, $f$ est convexe si et seulement si $\forall x,y, f(y) \ge f(x) + \langle \nabla f(x), y-x \rangle$ (la courbe est au-dessus de ses tangentes).
    Si $f$ est de classe $\mathcal{C}^2$, $f$ est convexe si et seulement si sa matrice Hessienne est semi-définie positive en tout point.
\end{proposition}

\section{Théorèmes de Séparation et Points Extrémaux}

\begin{objectif}
    Montrer comment les ensembles convexes peuvent être étudiés "de l'extérieur" (par les hyperplans qui les séparent) et "de l'intérieur" (par leurs briques de base, les points extrémaux).
\end{objectif}

\begin{theorem}[Théorèmes de Séparation de Hahn-Banach]
    C'est la version géométrique du théorème de Hahn-Banach.
    \begin{itemize}
        \item \textbf{Première forme :} Deux convexes non vides et disjoints, dont l'un est ouvert, peuvent être séparés par un hyperplan affine.
        \item \textbf{Seconde forme :} Deux convexes compacts non vides et disjoints peuvent être séparés \textbf{strictement}.
    \end{itemize}
\end{theorem}
\begin{remark}[La Dualité Géométrique]
    Ces théorèmes sont fondamentaux. Ils impliquent qu'un ensemble convexe fermé est l'intersection de tous les demi-espaces fermés qui le contiennent. Cela permet de ramener l'étude d'un objet convexe complexe à l'étude d'objets beaucoup plus simples, les demi-espaces.
\end{remark}

\begin{definition}[Point extrémal]
    Un point $x$ d'un convexe $C$ est \textbf{extrémal} s'il n'est le milieu d'aucun segment non trivial inclus dans $C$. Ce sont les "coins" ou les "sommets" du convexe.
\end{definition}

\begin{theorem}[Théorème de Krein-Milman]
    Tout convexe compact d'un espace localement convexe (en particulier, un Banach) est l'enveloppe convexe fermée de ses points extrémaux.
\end{theorem}
\begin{remark}[La Reconstruction par les Atomes]
    Ce théorème est magnifique. Il dit qu'un objet convexe compact, potentiellement très complexe, est entièrement déterminé par ses "atomes", les points extrémaux.
\end{remark}

\begin{example}
    \begin{itemize}
        \item Les points extrémaux d'un polygone sont ses sommets.
        \item Les points extrémaux d'un disque sont les points de son cercle frontière.
        \item La boule unité de l'espace $L^1([0,1])$ n'a aucun point extrémal.
    \end{itemize}
\end{example}

\section{Autres Théorèmes Géométriques Fondamentaux}

\begin{objectif}
    Présenter quelques autres résultats classiques de la géométrie convexe en dimension finie.
\end{objectif}

\begin{theorem}[Théorème de Helly]
    Soit une famille finie d'au moins $n+1$ ensembles convexes dans $\mathbb{R}^n$. Si toute sous-famille de $n+1$ de ces convexes a une intersection non vide, alors l'intersection de tous les convexes de la famille est non vide.
\end{theorem}

\begin{theorem}[Théorème du Point Fixe de Brouwer]
    Toute application continue d'une boule fermée de $\mathbb{R}^n$ dans elle-même admet au moins un point fixe.
\end{theorem}
\begin{remark}[Un Résultat Non Constructif]
    Ce théorème garantit l'existence d'un point fixe, mais ne donne aucune méthode pour le trouver. Sa preuve repose sur des arguments de topologie algébrique (homologie) et n'est pas au programme de l'agrégation, mais le résultat doit être connu.
\end{remark}

\begin{application}[Équilibre de Nash en théorie des jeux]
    Le théorème de Brouwer peut être utilisé pour prouver l'existence d'un équilibre de Nash dans un jeu à $n$ joueurs. L'ensemble des stratégies mixtes est un convexe compact, et on construit une fonction continue sur cet ensemble dont les points fixes correspondent aux équilibres.
\end{application}
\input{./chapters/analyse/equations_différentielles.tex}
\input{./chapters/analyse/calcul_différentiel_infinie.tex}

\part{Topologie}
\input{./chapters/topologie/espaces_métriques.tex}

\part{Probabilités}
\chapter{Combinatoire et Graphes : L'Art de Dénombrer et de Connecter}

\section{Principes Fondamentaux du Dénombrement}

\begin{objectif}
    Établir les principes de base du comptage et introduire l'outil le plus puissant de la combinatoire énumérative : les séries génératrices, qui encodent une suite infinie de nombres dans un objet analytique (une série formelle ou une fonction).
\end{objectif}

\begin{definition}[Coefficients Binomiaux et Formule du Binôme]
    $\binom{n}{k}$ est le nombre de façons de choisir $k$ objets parmi $n$. Formule du binôme : $(x+y)^n = \sum_{k=0}^n \binom{n}{k} x^k y^{n-k}$.
\end{definition}

\begin{proposition}[Formule du Crible de Poincaré (Principe d'Inclusion-Exclusion)]
    Pour des ensembles finis $A_1, \dots, A_n$ :
    $$ |\cup_{i=1}^n A_i| = \sum_i |A_i| - \sum_{i<j} |A_i \cap A_j| + \sum_{i<j<k} |A_i \cap A_j \cap A_k| - \dots $$
\end{proposition}
\begin{application}[Nombre de dérangements]
    La formule du crible permet de calculer le nombre de permutations d'un ensemble à $n$ éléments qui n'ont aucun point fixe.
\end{application}

\begin{definition}[Séries Génératrices]
    La série génératrice (ordinaire) d'une suite $(a_n)_{n \in \mathbb{N}}$ est la série formelle $A(x) = \sum_{n=0}^\infty a_n x^n$.
\end{definition}
\begin{remark}[Un Dictionnaire Combinatoire-Analyse]
    Les séries génératrices sont un "dictionnaire" qui traduit des opérations combinatoires sur les suites en opérations algébriques sur les séries.
    \begin{itemize}
        \item Somme de suites $\leftrightarrow$ Somme des séries.
        \item Produit de convolution $\leftrightarrow$ Produit des séries.
        \item Décalage de suite $\leftrightarrow$ Multiplication/Division par $x$.
    \end{itemize}
\end{remark}

\begin{example}[Nombres de Fibonacci]
    La suite de Fibonacci est définie par $F_{n+2} = F_{n+1} + F_n$. Cette relation de récurrence se traduit par une équation sur la série génératrice $F(x)$, ce qui permet de trouver sa fraction rationnelle : $F(x) = \frac{x}{1-x-x^2}$. En décomposant cette fraction en éléments simples, on retrouve la formule de Binet pour $F_n$.
\end{example}

\begin{example}[Nombres de Catalan]
    Les nombres de Catalan $C_n$ comptent de très nombreux objets (chemins de Dyck, triangulations de polygones...). Leur série génératrice vérifie une équation quadratique, ce qui permet de trouver une formule explicite pour $C_n$.
\end{example}

\section{Le Langage des Graphes}

\begin{objectif}
    Introduire le vocabulaire de base de la théorie des graphes, qui est le langage mathématique pour modéliser des relations entre des objets.
\end{objectif}

\begin{definition}[Graphe]
    Un graphe (simple) $G=(V,E)$ est la donnée d'un ensemble de sommets $V$ et d'un ensemble d'arêtes $E \subset \mathcal{P}_2(V)$ (paires de sommets).
    On définit le degré d'un sommet, la notion de chemin, de cycle, de connexité.
\end{definition}

\begin{proposition}[Lemme des poignées de main]
    La somme des degrés des sommets d'un graphe est égale à deux fois le nombre d'arêtes. En particulier, le nombre de sommets de degré impair est pair.
\end{proposition}

\begin{definition}[Types de Graphes]
    \begin{itemize}
        \item \textbf{Graphe complet $K_n$} : Tous les sommets sont reliés.
        \item \textbf{Graphe biparti} : L'ensemble des sommets peut être partitionné en deux ensembles $X,Y$ tels que toute arête relie un sommet de $X$ à un sommet de $Y$.
        \item \textbf{Arbre} : Graphe connexe et sans cycle.
    \end{itemize}
\end{definition}
\begin{theorem}[Caractérisation des arbres]
    Un graphe à $n$ sommets est un arbre si et seulement s'il est connexe et a $n-1$ arêtes.
\end{theorem}

\section{Théorèmes Fondamentaux de la Théorie des Graphes}

\begin{objectif}
    Présenter quelques résultats classiques et puissants qui illustrent les différents types de questions que l'on se pose en théorie des graphes.
\end{objectif}

\begin{theorem}[Cycles Eulériens et Hamiltoniens]
    \begin{itemize}
        \item \textbf{(Euler)} Un graphe connexe admet un cycle eulérien (qui passe par chaque arête exactement une fois) si et seulement si tous ses sommets sont de degré pair.
        \item \textbf{(Hamiltonien)} Trouver une condition nécessaire et suffisante pour l'existence d'un cycle hamiltonien (qui passe par chaque sommet exactement une fois) est un problème NP-complet. C'est le problème du voyageur de commerce.
    \end{itemize}
\end{theorem}

\begin{theorem}[Théorème de Hall sur les Mariages]
    Dans un graphe biparti $G=(X \cup Y, E)$, il existe un couplage parfait (qui couvre tous les sommets de $X$) si et seulement si pour tout sous-ensemble $A \subset X$, le nombre de ses voisins $|N(A)|$ est au moins égal à $|A|$.
\end{theorem}
\begin{remark}[La Condition du Voisinage]
    Ce théorème est le prototype des résultats de "combinatoire extrémale". Il donne une condition locale (sur les sous-ensembles) qui garantit une propriété globale (l'existence d'un couplage).
\end{remark}

\begin{definition}[Coloration de Graphe]
    Une $k$-coloration d'un graphe est une attribution d'une couleur (parmi $k$) à chaque sommet de sorte que deux sommets adjacents aient des couleurs différentes. Le nombre chromatique $\chi(G)$ est le plus petit $k$ pour lequel une telle coloration existe.
\end{definition}

\begin{theorem}[Théorème des Quatre Couleurs (admis)]
    Tout graphe planaire (qui peut être dessiné sur un plan sans que les arêtes ne se croisent) a un nombre chromatique inférieur ou égal à 4.
\end{theorem}

\begin{theorem}[Théorème de Turan (cas simple)]
    Le nombre maximal d'arêtes dans un graphe à $n$ sommets qui ne contient aucun triangle ($K_3$) est $\lfloor n^2/4 \rfloor$. Ce maximum est atteint par le graphe biparti complet équilibré.
\end{theorem}
\begin{remark}[Théorie Extrémale]
    Ce résultat est le point de départ de la théorie des graphes extrémale, qui cherche à déterminer le nombre maximal d'arêtes qu'un graphe peut avoir sans contenir un certain sous-graphe interdit.
\end{remark}
\chapter{Espaces Probabilisés : La Mesure de l'Incertitude}

\section{Les Axiomes de Kolmogorov : Le Cadre de la Théorie}

\begin{objectif}
    Établir le socle axiomatique de la théorie des probabilités moderne. L'idée géniale de Kolmogorov est de réaliser que le langage de la théorie de la mesure est l'outil parfait pour modéliser l'incertitude. On va donc "traduire" le vocabulaire de l'aléatoire dans celui, rigoureux et puissant, de la mesure.
\end{objectif}

\begin{definition}[Espace probabilisé]
    Un espace probabilisé est un triplet $(\Omega, \mathcal{F}, \mathbb{P})$ où :
    \begin{itemize}
        \item $\Omega$ est un ensemble, l'univers des possibles (ou "issues").
        \item $\mathcal{F}$ est une tribu (ou $\sigma$-algèbre) sur $\Omega$. Ses éléments sont les \textbf{événements}.
        \item $\mathbb{P}$ est une \textbf{mesure de probabilité} sur $(X, \mathcal{F})$, c'est-à-dire une mesure telle que $\mathbb{P}(\Omega)=1$.
    \end{itemize}
\end{definition}

\begin{remark}[La Probabilité est une Théorie de la Mesure]
    C'est le point de vue fondamental qui unifie tout. Un espace probabilisé N'EST RIEN D'AUTRE qu'un espace mesuré de masse totale 1. Cette vision fait disparaître le "mystère" des probabilités : tous les théorèmes de la théorie de la mesure (convergence, Fubini...) s'appliqueront.
\end{remark}

\begin{definition}[Probabilité conditionnelle et Indépendance]
    Soit $A, B$ deux événements avec $\mathbb{P}(B)>0$. La \textbf{probabilité conditionnelle} de $A$ sachant $B$ est $\mathbb{P}(A|B) = \frac{\mathbb{P}(A \cap B)}{\mathbb{P}(B)}$.
    Deux événements $A$ et $B$ sont \textbf{indépendants} si $\mathbb{P}(A \cap B) = \mathbb{P}(A)\mathbb{P}(B)$.
\end{definition}

\begin{theorem}[Formule des probabilités totales et Formule de Bayes]
    \textbf{FPT :} Si $(B_i)$ est une partition de $\Omega$, alors $\mathbb{P}(A) = \sum_i \mathbb{P}(A|B_i)\mathbb{P}(B_i)$.
    \textbf{Bayes :} $\mathbb{P}(B_i|A) = \frac{\mathbb{P}(A|B_i)\mathbb{P}(B_i)}{\sum_j \mathbb{P}(A|B_j)\mathbb{P}(B_j)}$.
\end{theorem}
\begin{remark}[Inférence et Apprentissage]
    La formule de Bayes est le moteur de l'inférence statistique bayésienne. Elle nous dit comment mettre à jour nos croyances (la probabilité d'une cause $B_i$) à la lumière de nouvelles données (l'observation d'un effet $A$).
\end{remark}

\section{Variables Aléatoires : Les Fonctions de l'Aléatoire}

\begin{objectif}
    Démystifier le concept de "variable aléatoire". Ce n'est pas une "variable" au sens usuel, mais une \textbf{fonction} qui associe une valeur numérique (ou vectorielle) à chaque issue de l'univers. C'est l'objet mathématique qui nous permet de passer de la description des événements à la quantification des résultats.
\end{objectif}

\begin{definition}[Variable Aléatoire]
    Une variable aléatoire (v.a.) réelle est une fonction \textbf{mesurable} $X: (\Omega, \mathcal{F}) \to (\mathbb{R}, \mathcal{B}(\mathbb{R}))$.
\end{definition}

\begin{definition}[Loi d'une Variable Aléatoire]
    La \textbf{loi} de la v.a. $X$ est la mesure de probabilité $P_X$ sur $(\mathbb{R}, \mathcal{B}(\mathbb{R}))$ définie par $P_X(A) = \mathbb{P}(X \in A) = \mathbb{P}(X^{-1}(A))$.
    C'est la \textbf{mesure image} de $\mathbb{P}$ par $X$.
\end{definition}
\begin{remark}[Séparer l'Aléatoire de l'Objet]
    Cette définition est cruciale. Elle sépare l'espace abstrait de départ $(\Omega, \mathcal{F}, \mathbb{P})$ de l'espace d'arrivée concret $(\mathbb{R})$ où l'on fait nos calculs. Toute l'information "aléatoire" de $X$ est contenue dans sa loi $P_X$.
\end{remark}

\begin{definition}[Fonction de Répartition]
    La fonction de répartition de $X$ est $F_X(x) = \mathbb{P}(X \le x) = P_X(]-\infty, x])$. Elle caractérise la loi $P_X$.
    \begin{itemize}
        \item Si $F_X$ est une fonction en escalier, la loi est \textbf{discrète}.
        \item Si $F_X$ est absolument continue, i.e. $F_X(x) = \int_{-\infty}^x f_X(t)dt$, la loi est à \textbf{densité} $f_X$.
    \end{itemize}
\end{definition}

\section{Espérance et Moments : L'Intégrale sur l'Univers}

\begin{objectif}
    Définir la "valeur moyenne" d'une variable aléatoire, l'espérance, comme une intégrale de Lebesgue. C'est l'application directe de la théorie de la mesure, qui donne un sens rigoureux à cette notion.
\end{objectif}

\begin{definition}[Espérance]
    L'espérance d'une variable aléatoire $X$ est son intégrale par rapport à la mesure de probabilité $\mathbb{P}$ :
    $$ E[X] = \int_\Omega X(\omega) d\mathbb{P}(\omega) $$
    L'espérance existe si $X$ est intégrable au sens de Lebesgue.
\end{definition}

\begin{theorem}[Théorème de Transfert]
    Le calcul de l'espérance peut être "transféré" de l'espace abstrait $\Omega$ à l'espace d'arrivée $\mathbb{R}$ :
    $$ E[X] = \int_\mathbb{R} x dP_X(x) $$
    Ce qui devient $\sum_i x_i \mathbb{P}(X=x_i)$ dans le cas discret et $\int_\mathbb{R} x f_X(x)dx$ dans le cas à densité.
\end{theorem}

\begin{definition}[Variance, Covariance et Moments]
    \textbf{Variance :} $V(X) = E[(X-E[X])^2] = E[X^2] - (E[X])^2$. C'est une mesure de la dispersion de la loi.
    \textbf{Covariance :} $\mathrm{Cov}(X,Y) = E[(X-E[X])(Y-E[Y])]$. Mesure le lien linéaire entre $X$ et $Y$.
    \textbf{Moments :} Le moment d'ordre $k$ est $m_k=E[X^k]$.
\end{definition}

\begin{proposition}[Inégalités de Concentration]
    \begin{itemize}
        \item \textbf{Markov :} Pour $X \ge 0$, $\mathbb{P}(X \ge a) \le \frac{E[X]}{a}$.
        \item \textbf{Bienaymé-Tchebychev :} $\mathbb{P}(|X-E[X]| \ge a) \le \frac{V(X)}{a^2}$.
    \end{itemize}
\end{proposition}

\section{Les Théorèmes Limites : L'Émergence de l'Ordre}

\begin{objectif}
    Présenter les deux résultats les plus importants et les plus profonds de la théorie. Ils décrivent comment la somme d'un grand nombre de variables aléatoires indépendantes et identiquement distribuées (i.i.d.) cesse d'être aléatoire pour faire émerger un comportement déterministe (Loi des Grands Nombres) et une forme universelle (Théorème Central Limite).
\end{objectif}

\begin{definition}[Modes de Convergence des v.a.]
    Soit $(X_n)$ une suite de v.a.
    \begin{itemize}
        \item \textbf{Presque sûre ($p.s.$) :} $X_n \to X$ si $\mathbb{P}(\{\omega \mid X_n(\omega) \to X(\omega)\}) = 1$.
        \item \textbf{En probabilité ($P$) :} $X_n \to X$ si $\forall \epsilon>0, \mathbb{P}(|X_n-X|>\epsilon) \to 0$.
        \item \textbf{En loi ($\mathcal{L}$) :} $X_n \to X$ si $F_{X_n}(x) \to F_X(x)$ en tout point de continuité de $F_X$.
    \end{itemize}
\end{definition}

\begin{theorem}[Loi des Grands Nombres]
    Soit $(X_n)$ une suite de v.a. i.i.d. d'espérance $\mu$. Soit $\bar{X}_n = \frac{1}{n}\sum_{i=1}^n X_i$ la moyenne empirique.
    \begin{itemize}
        \item \textbf{Loi Faible :} Si la variance est finie, $\bar{X}_n \xrightarrow{P} \mu$.
        \item \textbf{Loi Forte (Kolmogorov) :} Si l'espérance est finie, $\bar{X}_n \xrightarrow{p.s.} \mu$.
    \end{itemize}
\end{theorem}
\begin{remark}[La Stabilité des Moyennes]
    Ce théorème est le fondement de l'assurance, des casinos, et de la méthode de Monte-Carlo. Il garantit que sur le long terme, les moyennes des résultats observés convergent vers la moyenne théorique. L'aléa se "moyenne" et disparaît à la limite.
\end{remark}

\begin{theorem}[Théorème Central Limite (Lindeberg-Lévy)]
    Soit $(X_n)$ une suite de v.a. i.i.d. d'espérance $\mu$ et de variance finie $\sigma^2$. Alors la moyenne empirique normalisée converge en loi vers une loi normale centrée réduite :
    $$ \frac{\bar{X}_n - \mu}{\sigma/\sqrt{n}} \xrightarrow{\mathcal{L}} \mathcal{N}(0,1) $$
\end{theorem}

\begin{remark}[L'Universalité de la Courbe de Gauss]
    C'est un résultat miraculeux. Il dit que, quelles que soient les particularités de la loi de départ des $X_n$ (loi de Bernoulli, loi uniforme, etc.), la somme d'un grand nombre de ces variables aura toujours une distribution qui ressemble à une courbe de Gauss. La loi normale est un "attracteur" dans le monde des lois de probabilité. C'est pourquoi on la retrouve partout en nature et en sciences.
\end{remark}
\chapter{Statistiques et Méthodes : L'Art de l'Inférence face à l'Incertitude}

\section{Le Cadre de la Statistique Inférentielle}

\begin{objectif}
    Poser le problème fondamental de la statistique, qui est le "problème inverse" des probabilités. En probabilités, on connaît le modèle et on déduit le comportement des données. En statistiques, on observe les données et on cherche à en \textbf{inférer} le modèle sous-jacent. C'est l'art de remonter de l'échantillon à la population.
\end{objectif}

\begin{definition}[Modèle Statistique et Échantillon]
    Un \textbf{modèle statistique} est un triplet $(\mathcal{X}, \mathcal{A}, \{P_\theta\}_{\theta \in \Theta})$ où :
    \begin{itemize}
        \item $(\mathcal{X}, \mathcal{A})$ est un espace mesurable (l'espace des observations).
        \item $\{P_\theta\}_{\theta \in \Theta}$ est une famille de lois de probabilité sur cet espace, indexée par un paramètre $\theta$ inconnu appartenant à l'espace des paramètres $\Theta$.
    \end{itemize}
    Un \textbf{échantillon} de taille $n$ est une suite $(X_1, \dots, X_n)$ de variables aléatoires i.i.d. (indépendantes et identiquement distribuées) selon une loi $P_\theta$.
\end{definition}

\begin{remark}[Les Deux Grandes Tâches de la Statistique]
    Face à un échantillon, le statisticien se pose principalement deux questions :
    \begin{enumerate}
        \item \textbf{Estimation :} Quelle est la "meilleure" valeur possible pour le paramètre inconnu $\theta$ ? (e.g., estimer la proportion de votants pour un candidat).
        \item \textbf{Test d'hypothèses :} Les données sont-elles compatibles avec une certaine affirmation sur $\theta$ ? (e.g., ce nouveau médicament est-il plus efficace que le placebo ?).
    \end{enumerate}
\end{remark}

\section{Estimation Ponctuelle : Deviner le Bon Paramètre}

\begin{objectif}
    Construire des "estimateurs", c'est-à-dire des fonctions de l'échantillon qui nous donnent une valeur approchée du paramètre inconnu. On développera des critères pour juger de la "qualité" d'un estimateur.
\end{objectif}

\begin{definition}[Estimateur et ses Qualités]
    Un \textbf{estimateur} de $\theta$ est une statistique $\hat{\theta}_n = T(X_1, \dots, X_n)$.
    \begin{itemize}
        \item Le \textbf{biais} est $b(\hat{\theta}_n) = E[\hat{\theta}_n] - \theta$. Un estimateur est \textbf{sans biais} si son biais est nul.
        \item L'\textbf{erreur quadratique moyenne} est $\mathrm{MSE}(\hat{\theta}_n) = E[(\hat{\theta}_n - \theta)^2] = V(\hat{\theta}_n) + b(\hat{\theta}_n)^2$.
        \item Un estimateur est \textbf{convergent} si $\hat{\theta}_n \to \theta$ (en probabilité ou presque sûrement).
    \end{itemize}
\end{definition}

\begin{definition}[Méthode du Maximum de Vraisemblance]
    La fonction de \textbf{vraisemblance} est la densité de l'échantillon, vue comme une fonction du paramètre $\theta$: $\mathcal{L}(\theta; x_1, \dots, x_n) = \prod_{i=1}^n f_\theta(x_i)$.
    L'estimateur du maximum de vraisemblance (EMV) est la valeur $\hat{\theta}_{MV}$ qui maximise cette fonction. C'est la valeur du paramètre qui rend les observations observées "les plus probables".
\end{definition}

\begin{theorem}[Borne de Cramér-Rao]
    Sous des conditions de régularité, la variance de tout estimateur sans biais $\hat{\theta}_n$ est minorée par l'inverse de l'information de Fisher : $V(\hat{\theta}_n) \ge \frac{1}{I(\theta)}$. Un estimateur qui atteint cette borne est dit \textbf{efficace}.
\end{theorem}

\begin{example}[Estimation pour une loi de Bernoulli]
    On observe $n$ lancers d'une pièce (0 ou 1) de paramètre inconnu $p$. La moyenne empirique $\bar{X}_n$ est un estimateur de $p$. Il est sans biais, convergent (par la LGN), et c'est aussi l'estimateur du maximum de vraisemblance.
\end{example}

\section{Estimation par Intervalle de Confiance}

\begin{objectif}
    Reconnaître l'incertitude. Une estimation ponctuelle est presque sûrement fausse. Un intervalle de confiance fournit une plage de valeurs plausibles pour le paramètre, associée à un niveau de confiance.
\end{objectif}

\begin{definition}[Intervalle de Confiance]
    Un intervalle de confiance pour $\theta$ au niveau de confiance $1-\alpha$ est un intervalle aléatoire $[A_n, B_n]$ (où $A_n$ et $B_n$ sont des statistiques) tel que :
    $$ \mathbb{P}_\theta( \theta \in [A_n, B_n] ) = 1-\alpha $$
\end{definition}
\begin{remark}[L'Interprétation Fréquentiste]
    Attention à l'interprétation ! Ce n'est pas le paramètre $\theta$ qui est aléatoire, mais l'intervalle. Un niveau de confiance de 95\% signifie que si l'on répétait l'expérience un très grand nombre de fois, 95\% des intervalles ainsi construits contiendraient la vraie valeur (inconnue) du paramètre.
\end{remark}

\begin{application}[Intervalle de confiance pour une moyenne]
    Si l'on a un échantillon d'une loi normale de moyenne $\mu$ inconnue et de variance $\sigma^2$ connue, un intervalle de confiance pour $\mu$ au niveau $1-\alpha$ est :
    $$ \left[ \bar{X}_n - z_{\alpha/2} \frac{\sigma}{\sqrt{n}}, \bar{X}_n + z_{\alpha/2} \frac{\sigma}{\sqrt{n}} \right] $$
    où $z_{\alpha/2}$ est le quantile de la loi normale centrée réduite. Si $\sigma^2$ est inconnue, on l'estime par la variance empirique et on utilise les quantiles de la loi de Student.
\end{application}

\section{Tests d'Hypothèses : La Prise de Décision}

\begin{objectif}
    Formaliser la démarche scientifique de prise de décision face à des données. On pose une hypothèse par défaut ($H_0$) et on ne la rejette que si les données observées sont "extrêmement improbables" sous cette hypothèse.
\end{objectif}

\begin{definition}[Formalisme de Neyman-Pearson]
    Un test oppose une \textbf{hypothèse nulle} $H_0: \theta \in \Theta_0$ à une \textbf{hypothèse alternative} $H_1: \theta \in \Theta_1$.
    \begin{itemize}
        \item \textbf{Erreur de type I :} Rejeter $H_0$ alors qu'elle est vraie. Sa probabilité est notée $\alpha$ (le "risque").
        \item \textbf{Erreur de type II :} Ne pas rejeter $H_0$ alors qu'elle est fausse. Sa probabilité est notée $\beta$.
    \end{itemize}
    La \textbf{puissance} du test est $1-\beta$.
\end{definition}

\begin{lemma}[Neyman-Pearson]
    Pour tester $H_0: \theta=\theta_0$ contre $H_1: \theta=\theta_1$, le test le plus puissant (qui maximise $1-\beta$ pour un $\alpha$ donné) est le test du rapport de vraisemblance.
\end{lemma}

\begin{definition}[p-valeur]
    La \textbf{p-valeur} est la probabilité, sous l'hypothèse $H_0$, d'observer une valeur de la statistique de test au moins aussi "extrême" que celle qui a été effectivement observée. On rejette $H_0$ si la p-valeur est inférieure au risque $\alpha$ que l'on s'est fixé.
\end{definition}

\begin{application}[Tests statistiques classiques]
    \begin{itemize}
        \item \textbf{Test de Student :} Teste l'égalité d'une moyenne à une valeur de référence. (Ex: "Le poids moyen des baguettes dans cette boulangerie est-il bien de 250g ?").
        \item \textbf{Test du $\chi^2$ d'adéquation :} Teste si un échantillon suit une loi de probabilité donnée. (Ex: "Ce dé est-il équilibré ?").
        \item \textbf{Test du $\chi^2$ d'indépendance :} Teste l'indépendance de deux variables qualitatives. (Ex: "La couleur des yeux et la couleur des cheveux sont-elles des variables indépendantes ?").
    \end{itemize}
\end{application}

\part{Méthodes numériques}
\chapter{Équations Différentielles Numériques : Approximer le Continu par le Discret}

\section{Principes Fondamentaux de la Discrétisation}

\begin{objectif}
    Poser les bases de la résolution numérique des EDO. L'idée est de remplacer le problème continu $y'=f(t,y)$ par un schéma de récurrence discret $y_{n+1} = \Phi(h, t_n, y_n)$ qui approche la solution à des instants $t_n = t_0+nh$. Le défi est de s'assurer que notre approximation discrète reste "fidèle" à la véritable solution continue. Trois notions clés gouvernent cette fidélité : la consistance, la stabilité et la convergence.
\end{objectif}

\begin{definition}[Erreur, Ordre et Consistance]
    Soit $y(t)$ la solution exacte. L'erreur de consistance locale est $\epsilon_n(h) = y(t_{n+1}) - \Phi(h, t_n, y(t_n))$. C'est l'erreur que le schéma commet en une seule étape si on part de la solution exacte.
    \begin{itemize}
        \item Un schéma est \textbf{consistant} si $\epsilon_n(h) = o(h)$.
        \item Un schéma est d'**ordre** $p$ si $\epsilon_n(h) = \mathcal{O}(h^{p+1})$.
    \end{itemize}
\end{definition}

\begin{definition}[Stabilité]
    Un schéma est \textbf{stable} si de petites perturbations (erreurs d'arrondi, erreurs sur la condition initiale) ne sont pas amplifiées de manière explosive au fil des itérations. C'est une notion de robustesse et de bon conditionnement du schéma numérique.
\end{definition}

\begin{definition}[Convergence]
    Un schéma est \textbf{convergent} si l'erreur globale (la différence entre la solution numérique et la solution exacte en un temps donné) tend vers zéro lorsque le pas $h$ tend vers zéro.
\end{definition}

\begin{theorem}[Théorème d'Équivalence de Lax-Richtmyer]
    Pour un schéma numérique consistant qui résout un problème bien posé :
    $$ \textbf{Stabilité} \iff \textbf{Convergence} $$
\end{theorem}
\begin{remark}[Le Théorème Fondamental]
    Ce théorème est la pierre angulaire de l'analyse numérique des EDO. Il nous dit que pour garantir la convergence (le but ultime), il "suffit" de vérifier deux choses : la consistance (le schéma approxime bien l'équation localelement, ce qui se fait par des développements de Taylor) et la stabilité (le schéma ne fait pas exploser les erreurs).
\end{remark}

\section{Les Méthodes à un Pas : Le Futur ne dépend que du Présent}

\begin{objectif}
    Étudier la famille la plus simple de schémas, où le calcul de $y_{n+1}$ ne dépend que de l'information en $t_n$. On explorera le compromis entre simplicité (Euler explicite), stabilité (Euler implicite) et précision (Runge-Kutta).
\end{objectif}

\begin{definition}[Schéma d'Euler Explicite]
    Basé sur l'approximation de Taylor à l'ordre 1 : $y(t+h) \approx y(t) + h y'(t)$.
    Le schéma est : $y_{n+1} = y_n + h f(t_n, y_n)$.
    Il est d'ordre 1.
\end{definition}

\begin{definition}[Schéma d'Euler Implicite]
    Basé sur une approximation "à l'arrivée" : $y(t+h) \approx y(t) + h y'(t+h)$.
    Le schéma est : $y_{n+1} = y_n + h f(t_{n+1}, y_{n+1})$.
    Il est d'ordre 1. Il nécessite de résoudre une équation (souvent non-linéaire) à chaque étape pour trouver $y_{n+1}$.
\end{definition}

\begin{remark}[Le Coût de la Stabilité]
    L'Euler implicite est plus coûteux, mais il est beaucoup plus stable. Il est \textbf{A-stable}, ce qui le rend indispensable pour les problèmes dits "raides".
\end{remark}

\begin{definition}[Problèmes Raides (Stiff Problems)]
    Un système différentiel est raide s'il combine des dynamiques à des échelles de temps très différentes (certaines composantes évoluent très vite, d'autres très lentement).
\end{definition}

\begin{application}[Chimie et Biologie]
    La cinétique chimique et les réseaux de régulation biologique sont des sources majeures de problèmes raides. Les schémas explicites, contraints par la dynamique la plus rapide, nécessiteraient un pas $h$ ridiculement petit pour rester stables, même si l'on s'intéresse à l'évolution lente. Les schémas implicites sont alors la seule solution viable.
\end{application}

\begin{definition}[Méthodes de Runge-Kutta]
    L'idée est d'améliorer l'ordre de la méthode en évaluant $f$ en des points intermédiaires de l'intervalle $[t_n, t_{n+1}]$ pour obtenir une meilleure estimation de la "pente moyenne".
    Le schéma de Runge-Kutta classique d'ordre 4 (RK4) est le plus célèbre :
    $$ y_{n+1} = y_n + \frac{h}{6}(k_1 + 2k_2 + 2k_3 + k_4) $$
    où les $k_i$ sont des évaluations successives de $f$.
\end{definition}

\section{Les Méthodes Multipas : La Mémoire du Passé}

\begin{objectif}
    Construire des schémas d'ordre plus élevé en utilisant non seulement le point précédent $y_n$, mais aussi plusieurs points du passé ($y_{n-1}, y_{n-2}, \dots$).
\end{objectif}

\begin{definition}[Schémas d'Adams-Bashforth (explicites)]
    L'idée est d'approcher $y(t_{n+1}) = y(t_n) + \int_{t_n}^{t_{n+1}} f(t, y(t)) dt$ en remplaçant la fonction $f$ par le polynôme qui interpole les valeurs $f_k = f(t_k, y_k)$ aux points précédents $t_n, t_{n-1}, \dots$.
\end{definition}

\begin{definition}[Schémas d'Adams-Moulton (implicites)]
    Même idée, mais le polynôme d'interpolation utilise aussi le point (inconnu) $t_{n+1}$. Ces méthodes sont plus stables et plus précises à nombre de pas égal.
\end{definition}

\begin{definition}[Méthodes Prédicteur-Correcteur]
    On combine les deux approches pour obtenir le meilleur des deux mondes :
    \begin{enumerate}
        \item \textbf{Prédiction :} On calcule une première estimation $\tilde{y}_{n+1}$ avec une méthode explicite (e.g. Adams-Bashforth).
        \item \textbf{Correction :} On utilise cette estimation pour évaluer le terme implicite $f(t_{n+1}, \tilde{y}_{n+1})$ et on réinjecte dans une méthode implicite (e.g. Adams-Moulton) pour obtenir une valeur $y_{n+1}$ plus précise.
    \end{enumerate}
\end{definition}

\begin{theorem}[Barrières de Dahlquist]
    Ces deux résultats théoriques fondamentaux limitent l'optimisme dans la conception de schémas multipas.
    \begin{itemize}
        \item \textbf{Première barrière :} Un schéma multipas linéaire stable est d'ordre au plus $k+2$ (si $k$ est pair) ou $k+1$ (si $k$ est impair), où $k$ est le nombre de pas.
        \item \textbf{Seconde barrière :} Il n'existe pas de schéma multipas linéaire A-stable d'ordre supérieur à 2.
    \end{itemize}
\end{theorem}
\chapter{Résolution Numérique : L'Art de l'Approximation Efficace}

\section{Résolution d'Équations Non Linéaires $f(x)=0$}

\begin{objectif}
    Développer des algorithmes itératifs pour trouver les racines d'une fonction, c'est-à-dire les zéros de $f(x)=0$. La philosophie est de partir d'une estimation et de la raffiner à chaque étape. La qualité d'une méthode se juge à sa vitesse de convergence et à la robustesse de cette convergence.
\end{objectif}

\begin{definition}[Ordre de convergence]
    Soit $(x_n)$ une suite qui converge vers $x^*$. La convergence est d'ordre $p$ s'il existe $C>0$ tel que $|x_{n+1}-x^*| \le C |x_n - x^*|^p$.
    \begin{itemize}
        \item $p=1$ : Convergence \textbf{linéaire} (l'erreur est réduite d'un facteur constant à chaque étape).
        \item $p=2$ : Convergence \textbf{quadratique} (le nombre de chiffres significatifs double à chaque étape).
    \end{itemize}
\end{definition}

\begin{proposition}[Méthode de la bissection (dichotomie)]
    Si $f$ est continue sur $[a,b]$ avec $f(a)f(b)<0$, l'algorithme consiste à couper l'intervalle en deux à chaque étape et à conserver la moitié où le changement de signe persiste.
    La convergence est \textbf{lente} (linéaire, avec un facteur 1/2) mais \textbf{toujours garantie}.
\end{proposition}

\begin{theorem}[Méthode de Newton]
    Partant d'une estimation $x_n$, on approxime $f$ par sa tangente en $x_n$. La nouvelle estimation $x_{n+1}$ est le zéro de cette tangente :
    $$ x_{n+1} = x_n - \frac{f(x_n)}{f'(x_n)} $$
    Si $f$ est $\mathcal{C}^2$ et que $f'(x^*) \neq 0$, la convergence est (localement) \textbf{quadratique}.
\end{theorem}
\begin{remark}[La Ferrari des algorithmes]
    La méthode de Newton est extraordinairement rapide lorsqu'elle converge. Cependant, elle a des défauts : elle nécessite le calcul de la dérivée, et sa convergence n'est que locale. Un mauvais point de départ peut mener à une divergence ou à un comportement chaotique.
\end{remark}

\begin{application}[Calcul de $\sqrt{a}$]
    Pour trouver la racine de $f(x)=x^2-a=0$, la méthode de Newton donne l'itération $x_{n+1} = x_n - \frac{x_n^2-a}{2x_n} = \frac{1}{2}(x_n + \frac{a}{x_n})$. C'est l'algorithme de Héron, utilisé par les Babyloniens.
\end{application}

\section{Interpolation et Approximation Polynomiale}

\begin{objectif}
    Étant donné un ensemble de points (des données), construire une fonction (souvent un polynôme) qui passe par ces points (interpolation) ou qui les "approche au mieux" (approximation).
\end{objectif}

\begin{theorem}[Interpolation de Lagrange]
    Étant donnés $n+1$ points $(x_i, y_i)$ avec des abscisses distinctes, il existe un \textbf{unique} polynôme $P$ de degré au plus $n$ tel que $P(x_i)=y_i$ pour tout $i$.
    Les polynômes de base de Lagrange $L_i(X) = \prod_{j \neq i} \frac{X-x_j}{x_i-x_j}$ fournissent une construction explicite : $P(X) = \sum y_i L_i(X)$.
\end{theorem}

\begin{remark}[Phénomène de Runge]
    L'interpolation polynomiale de haut degré peut être un désastre. Si on interpole la fonction $f(x)=\frac{1}{1+25x^2}$ sur $[-1,1]$ avec des points équidistants, les polynômes d'interpolation oscillent de manière sauvage près des bords lorsque le degré augmente. Cela montre que "plus de points" ne signifie pas toujours "meilleure approximation globale".
\end{remark}

\begin{proposition}[Nœuds de Tchebychev]
    Pour minimiser l'erreur d'interpolation sur un intervalle, il faut choisir des nœuds non pas équidistants, mais resserrés sur les bords. Les racines des polynômes de Tchebychev sont un choix quasi-optimal.
\end{proposition}

\begin{application}[Approximation par les Moindres Carrés]
    Quand les données $(x_i, y_i)$ sont bruitées, on ne cherche plus un polynôme qui passe exactement par les points, mais un polynôme de bas degré $P$ qui minimise la somme des carrés des erreurs $\sum_i (y_i - P(x_i))^2$. Ce problème d'optimisation se ramène à la résolution d'un système linéaire (les équations normales), ce qui est un problème de projection orthogonale.
\end{application}

\section{Intégration Numérique (Quadrature)}

\begin{objectif}
    Calculer une valeur approchée de l'intégrale définie $I = \int_a^b f(x) dx$, souvent parce que l'on ne connaît pas de primitive de $f$.
\end{objectif}

\begin{definition}[Formules de Newton-Cotes]
    L'idée est d'approcher $\int_a^b f(x) dx$ par $\int_a^b P(x) dx$, où $P$ est un polynôme d'interpolation de $f$ en des points équidistants.
    \begin{itemize}
        \item \textbf{Méthode des rectangles} (degré 0).
        \item \textbf{Méthode des trapèzes} (degré 1).
        \item \textbf{Méthode de Simpson} (degré 2), qui est d'ordre 3, étonnamment élevé.
    \end{itemize}
\end{definition}

\begin{theorem}[Formules de Quadrature de Gauss-Legendre]
    Pour calculer $\int_{-1}^1 f(x) dx$, il est possible de faire beaucoup mieux que Newton-Cotes. Une formule de quadrature de Gauss à $n$ points est de la forme $\sum_{i=1}^n w_i f(x_i)$. En choisissant \textbf{judicieusement} les poids $w_i$ et les nœuds $x_i$ (qui sont les racines du $n$-ième polynôme de Legendre), on peut obtenir une méthode d'ordre $2n-1$, ce qui est extraordinairement efficace.
\end{theorem}

\section{Algèbre Linéaire Numérique}

\begin{objectif}
    Développer des algorithmes efficaces et numériquement stables pour résoudre les deux problèmes centraux de l'algèbre linéaire : la résolution de systèmes linéaires $Ax=b$ et le calcul de valeurs propres.
\end{objectif}

\begin{definition}[Conditionnement]
    Le \textbf{conditionnement} d'une matrice inversible $A$ est $\mathrm{cond}(A) = \|A\| \|A^{-1}\|$. Il mesure la sensibilité de la solution $x$ aux perturbations des données $A$ et $b$. Si $\mathrm{cond}(A)$ est grand, le problème est \textbf{mal conditionné} et difficile à résoudre numériquement.
\end{definition}

\begin{proposition}[Méthodes Directes pour $Ax=b$]
    \begin{itemize}
        \item \textbf{Élimination de Gauss et Décomposition LU :} C'est l'algorithme de base, qui consiste à transformer $A$ en une matrice triangulaire supérieure $U$ par des opérations élémentaires sur les lignes, ce qui revient à factoriser $A=LU$. La \textbf{stratégie du pivot partiel} est essentielle pour garantir la stabilité numérique.
        \item \textbf{Décomposition de Cholesky :} Si $A$ est symétrique définie positive, on peut la factoriser sous la forme $A=B B^T$, où $B$ est triangulaire inférieure. C'est deux fois plus rapide que LU.
    \end{itemize}
\end{proposition}

\begin{theorem}[Méthodes Itératives pour $Ax=b$]
    Pour les systèmes très grands et creux (issus des EDP), les méthodes directes sont trop coûteuses. On utilise des méthodes itératives qui construisent une suite $x_k$ convergeant vers la solution. L'idée est de décomposer $A=M-N$ et d'itérer $M x_{k+1} = N x_k + b$.
    \begin{itemize}
        \item \textbf{Méthode de Jacobi :} $M$ est la diagonale de $A$.
        \item \textbf{Méthode de Gauss-Seidel :} $M$ est la partie triangulaire inférieure de $A$.
    \end{itemize}
\end{theorem}

\begin{theorem}[Convergence des méthodes itératives]
    La suite $(x_k)$ converge vers la solution de $Ax=b$ pour tout $x_0$ si et seulement si le rayon spectral de la matrice d'itération $J=M^{-1}N$ est strictement inférieur à 1.
\end{theorem}

\begin{application}[Calcul de valeurs propres : Méthode de la puissance]
    Pour trouver la valeur propre de plus grand module d'une matrice $A$, on peut utiliser l'algorithme simple de la \textbf{méthode de la puissance} : on part d'un vecteur $x_0$ et on calcule la suite $x_{k+1} = \frac{A x_k}{\|A x_k\|}$. Sous de bonnes conditions, cette suite converge vers le vecteur propre associé à la valeur propre dominante.
\end{application}
\chapter{Option Calcul Scientifique : La Boîte à Outils du Modélisateur}

\section{La Philosophie de l'Épreuve de Modélisation}

\begin{objectif}
    Comprendre que cet oral n'est pas un test de connaissances brutes, mais une évaluation de la \textbf{maturité scientifique}. Le jury veut voir votre capacité à :
    \begin{enumerate}
        \item \textbf{Analyser et Synthétiser :} Lire un texte scientifique, en extraire le modèle mathématique principal et les questions pertinentes.
        \item \textbf{Proposer et Critiquer :} Suggérer des méthodes numériques adaptées au modèle, et surtout, être capable de discuter leurs avantages, leurs inconvénients, et leur coût.
        \item \textbf{Implémenter et Simuler :} Écrire un code (simple) qui met en œuvre l'algorithme choisi et interpréter les résultats obtenus.
    \end{enumerate}
    Cette fiche est un guide pour les deux premières étapes, qui sont les plus importantes.
\end{objectif}

\begin{remark}[La Question Clé]
    Face à un texte, la première question à se poser est toujours : \textbf{Quel est le modèle mathématique sous-jacent ?} Est-ce une EDO ? Un système linéaire ? Un problème d'optimisation ? Un problème aux valeurs propres ? L'identification correcte du modèle dicte toute la suite de l'analyse.
\end{remark}

\section{La Boîte à Outils Algorithmique par Type de Problème}

\begin{objectif}
    Organiser sa pensée non pas par méthode, mais par \textbf{type de problème}. C'est la clé pour proposer rapidement une solution pertinente le jour de l'épreuve.
\end{objectif}

\subsection{Si le modèle est une Équation Différentielle Ordinaire (EDO)}

\begin{proposition}[Checklist pour EDO $y'=f(t,y)$]
    \begin{enumerate}
        \item \textbf{Théorie :} Le problème de Cauchy est-il bien posé ? (théorème de Cauchy-Lipschitz, $f$ est-elle lipschitzienne ?)
        \item \textbf{Stabilité :} Le système est-il \textbf{raide (stiff)} ? Y a-t-il des échelles de temps très différentes ?
            \begin{itemize}
                \item \textbf{Si NON (non-raide) :} Un schéma \textbf{explicite} est généralement préférable (moins coûteux). RK4 est le cheval de bataille par défaut : simple à coder, d'ordre élevé, stable pour des pas raisonnables.
                \item \textbf{Si OUI (raide) :} Un schéma \textbf{implicite} est \textbf{obligatoire}. Le plus simple est l'Euler implicite. Il est A-stable et robuste, mais d'ordre 1 et nécessite de résoudre une équation à chaque pas (par ex. avec la méthode de Newton).
            \end{itemize}
        \item \textbf{Conservation :} Le système physique a-t-il des invariants (énergie, moment...) ? Si oui, il peut être judicieux de choisir un \textbf{schéma symplectique} (comme Euler-Cromer ou Verlet pour la mécanique) qui préserve ces quantités au lieu d'un schéma qui les dissipe.
    \end{enumerate}
\end{proposition}

\subsection{Si le modèle est un Système Linéaire $Ax=b$}

\begin{proposition}[Checklist pour $Ax=b$]
    \begin{enumerate}
        \item \textbf{Analyse de la Matrice $A$ :} C'est l'étape la plus importante.
            \begin{itemize}
                \item \textbf{Taille :} Petite ($n < 1000$) ou Grande ($n > 10^4$) ?
                \item \textbf{Structure :} Pleine, Creuse (beaucoup de zéros), Symétrique, Définie Positive ?
                \item \textbf{Conditionnement :} $\mathrm{cond}(A) = \|A\| \|A^{-1}\|$. Est-il grand ? (si oui, le problème est sensible aux erreurs).
            \end{itemize}
        \item \textbf{Choix de la Méthode :}
            \begin{itemize}
                \item \textbf{Si $A$ est petite et pleine :} Une \textbf{méthode directe} est la meilleure solution. La décomposition LU avec pivot partiel est la méthode universelle. Si $A$ est symétrique définie positive, la décomposition de Cholesky est deux fois plus rapide.
                \item \textbf{Si $A$ est grande et creuse :} Une \textbf{méthode itérative} est indispensable.
                    \begin{itemize}
                        \item Jacobi ou Gauss-Seidel sont des méthodes de base. Elles convergent si $A$ est à diagonale strictement dominante.
                        \item Pour les problèmes plus difficiles, la méthode du \textbf{gradient conjugué} (si $A$ est SDP) est la référence absolue. Elle est optimale au sens d'un certain critère.
                    \end{itemize}
            \end{itemize}
    \end{enumerate}
\end{proposition}

\subsection{Si le modèle est un Problème aux Valeurs Propres $Ax=\lambda x$}

\begin{proposition}[Checklist pour $Ax=\lambda x$]
    \begin{enumerate}
        \item \textbf{objectif :} Cherche-t-on toutes les valeurs propres ou seulement quelques-unes ?
            \begin{itemize}
                \item \textbf{Pour quelques v.p. (souvent la plus grande/petite) :} Des \textbf{méthodes itératives} sont efficaces. La \textbf{méthode de la puissance} (pour la v.p. de plus grand module) et la méthode de la puissance inverse (pour la v.p. la plus proche d'un scalaire donné) sont les plus simples.
                \item \textbf{Pour toutes les v.p. :} Des méthodes plus complexes sont nécessaires. L'algorithme \textbf{QR} est la méthode de référence. Il consiste à construire une suite de matrices orthogonalement semblables qui converge vers une forme triangulaire (ou quasi-triangulaire), révélant les valeurs propres sur la diagonale.
            \end{itemize}
    \end{enumerate}
\end{proposition}

\section{Thèmes Transversaux et Algorithmes Essentiels}

\begin{objectif}
    Ajouter à notre boîte à outils des méthodes plus avancées qui apparaissent fréquemment dans les textes de modélisation.
\end{objectif}

\subsection{Optimisation Numérique : Trouver le Minimum}

\begin{remark}[Le Lien avec la Résolution d'Équations]
    Trouver le minimum d'une fonction $F(x)$ revient à trouver un zéro de son gradient $\nabla F(x)=0$. Les méthodes d'optimisation sont donc des "méthodes de Newton" pour le gradient.
\end{remark}

\begin{definition}[Méthodes de Descente]
    L'idée est de construire une suite $x_{k+1} = x_k + \alpha_k d_k$, où $d_k$ est une direction de descente et $\alpha_k$ est le pas.
    \begin{itemize}
        \item \textbf{Méthode du Gradient :} On choisit la direction de plus grande pente, $d_k = -\nabla F(x_k)$. Simple, mais peut être très lente (zigzags).
        \item \textbf{Méthode de Newton :} On minimise le modèle quadratique de $F$ en $x_k$. La direction est $d_k = -[H_F(x_k)]^{-1} \nabla F(x_k)$, où $H_F$ est la matrice Hessienne. Convergence quadratique, mais très coûteuse.
        \item \textbf{Méthode du Gradient Conjugué :} Pour les problèmes quadratiques, c'est une méthode "intermédiaire" qui converge en au plus $n$ itérations. C'est une méthode de choix pour les grands systèmes linéaires SDP.
    \end{itemize}
\end{definition}

\subsection{Équations aux Dérivées Partielles (EDP) : Le Strict Minimum}

\begin{remark}[La Philosophie de la Discrétisation]
    L'idée est de remplacer un problème en dimension infinie (trouver une fonction) par un très grand système algébrique en dimension finie (trouver les valeurs de la fonction sur une grille).
\end{remark}

\begin{definition}[Méthode des Différences Finies]
    On remplace les dérivées par des approximations par quotients différentiels sur une grille.
    $$ u'(x) \approx \frac{u(x+h)-u(x)}{h}, \quad u''(x) \approx \frac{u(x+h)-2u(x)+u(x-h)}{h^2} $$
    L'EDP devient un grand système linéaire (ou non-linéaire) dont les inconnues sont les valeurs de la fonction aux nœuds de la grille. La matrice de ce système est souvent très grande, creuse et structurée (bande, tridiagonale...).
\end{definition}

\begin{definition}[Méthode des Éléments Finis]
    Approche plus sophistiquée. On cherche une solution approchée comme une combinaison linéaire de "fonctions de base" simples (souvent des polynômes par morceaux). La méthode consiste à projeter l'EDP sur le sous-espace engendré par ces fonctions de base, via une formulation variationnelle. C'est la méthode de choix pour les géométries complexes.
\end{definition}

\subsection{La Transformée de Fourier Rapide (FFT)}

\begin{theorem}[Algorithme de Cooley-Tukey]
    La Transformée de Fourier Discrète (TFD) d'un signal de taille $N$ peut être calculée en $\mathcal{O}(N \log N)$ opérations au lieu de $\mathcal{O}(N^2)$ pour un calcul naïf.
\end{theorem}
\begin{remark}[L'Algorithme le plus Important du XXe Siècle]
    La FFT est partout : traitement du son et de l'image (JPEG, MP3), télécommunications (WiFi, 4G), résolution d'EDP (méthodes spectrales), multiplication rapide de grands polynômes... Sa basse complexité a rendu possibles des applications qui étaient inimaginables auparavant. Il faut connaître son existence et son coût.
\end{remark}


% \printbibliography[heading=bibintoc]

\restoregeometry

\end{document}
