\pdfobjcompresslevel 0
\documentclass[10pt, a4paper, parskip=full, twoside, twocolumn]{report}

% --- PREAMBULE ---
\usepackage[utf8]{inputenc}
\usepackage[T1]{fontenc}
\usepackage[french]{babel}
\usepackage{amsmath, amssymb, amsthm, amsfonts}
\usepackage[top=1cm,bottom=2cm,left=1cm,right=1cm]{geometry}
\usepackage{graphicx}
\usepackage{stmaryrd}
\usepackage{xcolor}
\usepackage{framed}
\usepackage{enumitem}
\usepackage{titlesec}
\usepackage{aliascnt}
\usepackage{tcolorbox} % The main package for creating the colored box
\tcbuselibrary{breakable}

% Define a custom color (optional, but good practice)
\definecolor{mygreen}{rgb}{0.2, 0.7, 0.2}
\definecolor{myblue}{rgb}{0.0, 0.2, 0.7}
\definecolor{myred}{rgb}{0.7, 0.2, 0.2}
\definecolor{developpement}{RGB}{255, 255, 224}

% Apply the color to the subsection title
\titleformat{\section}
  {\normalfont\large\bfseries\color{myred}} % The format for the whole line
  {\thesection}                           % The subsection number
  {1em}                                      % Separation between number and title
  {}                                         % Code before the title text (empty for now)

% Apply the color to the subsection title
\titleformat{\subsection}
  {\normalfont\large\bfseries\color{mygreen}} % The format for the whole line
  {\thesubsection}                           % The subsection number
  {1em}                                      % Separation between number and title
  {}                                         % Code before the title text (empty for now)
  
\newtheorem{definition}{Définition}
\newtheorem{theorem}[definition]{Théorème}
\newtheorem{theorem_def}[definition]{Théorème et définition}
\newtheorem{proposition}[definition]{Proposition}
\newtheorem{properties}[definition]{Propriétés}
\newtheorem{property}[definition]{Propriété}
\newtheorem{lemma}[definition]{Lemme}
\newtheorem{corollary}[definition]{Corollaire}
\newtheorem{example}[definition]{Exemple}
\newtheorem{cexample}[definition]{Contre-exemple}
\newtheorem{remark}[definition]{Remarque}
\newtheorem{application}[definition]{Application}
\newtheorem{reference}[definition]{Référence}
\newtheorem{algorithm}[definition]{Algorithme}
\newtheorem{conjecture}[definition]{Conjecture}
\newtheorem{notation}[definition]{Notation}
\newtheorem*{notation*}{Notation}

\newcommand{\IN}{\mathbb{N}}
\newcommand{\IZ}{\mathbb{Z}}
\newcommand{\IU}{\mathbb{U}}
\newcommand{\IK}{\mathbb{K}}
\newcommand{\IZnZ}{\mathbb{Z}/n\mathbb{Z}}
\newcommand{\IZpZ}{\mathbb{Z}/p\mathbb{Z}}
\newcommand{\IQ}{\mathbb{Q}}
\newcommand{\IC}{\mathbb{C}}
\newcommand{\IF}{\mathbb{F}}
\newcommand{\IR}{\mathbb{R}}
\newcommand{\IRn}{\mathbb{R}^n}
\newcommand{\IRd}{\mathbb{R}^d}
\newcommand{\IRm}{\mathbb{R}^m}
\newcommand{\IRp}{\mathbb{R}^p}
\newcommand{\IRnm}{\mathbb{R}^{n\times m}}
\newcommand{\IRmn}{\mathbb{R}^{m\times n}}
\newcommand{\IRpn}{\mathbb{R}^{p\times n}}
\newcommand{\IRnp}{\mathbb{R}^{n\times p}}
\newcommand{\IRpm}{\mathbb{R}^{p\times m}}
\newcommand{\IRmp}{\mathbb{R}^{m\times p}}
\newcommand{\actson}{\circlearrowleft}

\DeclareMathOperator{\im}{Im}
% \DeclareMathOperator{\Im}{Im}
\DeclareMathOperator{\pgcd}{pgcd}
\DeclareMathOperator{\ppcm}{ppcm}
\DeclareMathOperator{\card}{Card}
\DeclareMathOperator{\Is}{Is}
\DeclareMathOperator{\Orb}{Orb}
\DeclareMathOperator{\Stab}{Stab}
\DeclareMathOperator{\Fix}{Fix}
\DeclareMathOperator{\Supp}{Supp}
\DeclareMathOperator{\ord}{ord}
\DeclareMathOperator{\Ker}{Ker}
\DeclareMathOperator{\Syl}{Syl}
\DeclareMathOperator{\Mat}{Mat}
\DeclareMathOperator{\id}{id}
\DeclareMathOperator{\diag}{diag}
\DeclareMathOperator{\rg}{rg}
\DeclareMathOperator{\car}{car}
\DeclareMathOperator{\Hom}{Hom}
\DeclareMathOperator{\Aut}{Aut}
\DeclareMathOperator{\End}{End}
\DeclareMathOperator{\Div}{Div}

\titleformat{\chapter}[display]
  {\normalfont\bfseries}{}{0pt}{\LARGE}
\titlespacing*{\chapter}{0pt}{0pt}{\baselineskip}

\newcommand{\vertiii}[1]{{\left\vert\kern-0.25ex\left\vert\kern-0.25ex\left\vert #1 
    \right\vert\kern-0.25ex\right\vert\kern-0.25ex\right\vert}}

\title{Leçons d'oral de l'Agrégation}
\author{Gautier Laisné}
\date{}


\begin{document}
% \maketitle
% \tableofcontents

\chapter{101 : Groupe opérant sur un ensemble. Exemples d'applications.}
Dans cette leçon, $G$ désigne un groupe de neutre $1$, et $X$ désigne un ensemble.
\section*{I. Action d'un groupe sur un ensemble}
\subsection*{A. Définitions et premiers exemples}
\begin{definition}[\textnormal{[R] 19, [U] 27}]
	Une \emph{action} de $G$ sur $X$ est une application $G\times X\to X$ définie par 
	$(g,x)\mapsto g\cdot x$ vérifiant
	\begin{enumerate}
		\item $\forall\, (g,g')\in G^2,\, \forall\, x\in X,\, g'\cdot (g\cdot x) = (g'g)\cdot x$
		\item $\forall\, x\in X,\, 1\cdot x = x$
	\end{enumerate}
	Pour signigier que $G$ agit sur $X$, on note $G\actson X$.
\end{definition}

\begin{example}[\textnormal{[R] 19, [U] 28}]
	\begin{itemize}
		\item $\mathfrak{S}(X) \actson X$ par $\sigma\cdot x = \sigma(x)$
		\item Si $E$ est un espace vectoriel, alors $GL(E)\actson E$ par $\varphi\cdot x = \varphi(x)$
		\item $(g,x)\mapsto x$ est une action de $G$ sur $X$, appelée \emph{action triviale}.
	\end{itemize}
\end{example}

\begin{proposition}[\textnormal{[R] 19, [U] 28}]
	La donnée d'une action $(g,x)\mapsto g\cdot x$ de $G$ sur $X$ équivaut à la donnée d'un morphisme $\varphi\,\colon G\to \mathfrak{S}(X)$, $g\mapsto \left[x\mapsto g\cdot x\right]$, appelé \emph{morphisme associé à l'action de $G$ sur $X$}.
\end{proposition}

\begin{definition}[\textnormal{[R] 19/21, [U] 29}]
	Soit $x\in X$. Alors :
	\begin{itemize}
		\item L'\emph{orbite} de $x$ est l'ensemble $\Orb(x) = \left\{g\cdot x \mid g\in G\right\}$ (aussi noté $G\cdot x$) ;
		\item Le \emph{stabilisateur} de $x$ est l'ensemble $\Stab(x) = \left\{g\in G \mid g\cdot x = x\right\}$.
	\end{itemize}
\end{definition}

\begin{proposition}[\textnormal{[U] 34/37}]
	\begin{enumerate}
		\item $G\actson G$ par $g\cdot h=ghg^{-1}$ (on l'appelle \emph{action par conjugaison}). Le stabilisateur de $h\in G$ est appelé \emph{centralisateur} de $h$, et est noté $C(h)$.
		\item $G$ agit sur l'ensemble de ses sous-groupes par $g\cdot H = gHg^{-1}$ (action par conjugaison). Le stabilisateur de $H\leq G$ est appelé \emph{normalisateur} de $H$, et est noté $N(H)$.
	\end{enumerate}
\end{proposition}

\begin{definition}[\textnormal{[R] 20, [U] 29/31}]
	On dit que l'action de $G$ sur $X$ est \emph{transitive} si elle n'a qu'une seule orbite, \emph{i.e.} si $\forall\,(x,y)\in X^2,\, \exists\, g\in G\,\colon g=g\cdot x$.

	On dit que l'action de $G$ sur $X$ est \emph{fidèle} si $\varphi$ est injective.
\end{definition}

\begin{example}[\textnormal{[U] 31}]
	\begin{itemize}
		\item $\mathfrak{S}_n\actson \llbracket 1,n\rrbracket$ transitivement par $\sigma\cdot i=\sigma(i)$
		\item $G\actson G$ fidèlement par $g\cdot h = gh$ (on l'appelle \emph{action par translation à gauche})
		\item Soit $H$ un sous-groupe de $G$. L"'action de $G$ sur $G/H$ définie par $g\cdot xH = gxH$, appelée \emph{action par translation à gauche}, est transitive.
	\end{itemize}
\end{example}

\begin{proposition}[\textnormal{[R] 21}]
	Pour tout $x\in X$, $\Stab(x)$ est un sous-groupe de $G$.
\end{proposition}

\begin{proposition}[\textnormal{[U] 30}]
	$x\mathcal{R}y \iff \exists\, g\in G\,\colon g=g\cdot x$ définit 
	une relation d'équivalence sur $X$ dont les classes sont les orbites de l'action de $G$ sur $X$.
\end{proposition}

\begin{corollary}[\textnormal{[U] 30}]
	Les orbites partitionnent $X$.
\end{corollary}

\begin{example}[\textnormal{[U] 41}]
	Soit $\sigma\in\mathfrak{S}_n$. Le groupe $\langle\sigma\rangle$ agit sur $\llbracket 1,n\rrbracket$ par $\sigma^k\cdot i = \sigma^k(i)$.
	Les orbites non ponctuelles sont les supports des cylckes dans la décomposition en produit de cycles à supports disjoints de $\sigma$.
\end{example}

\subsection*{B. Cas d'un groupe et d'un ensemble finis}
Dans ce paragraphe, on suppose $G$ et $X$ finis. On pose $n = \card(G)$.
\begin{theorem}[de Caylay - \textnormal{[R] 21, [U] 31}]
	$G$ s'identifie à un sous-groupe de $\mathfrak{S}_n$.
\end{theorem}
\begin{proposition}[\textnormal{[R] ?, [U] ?}]
	$\forall\, (x,y)\in X^2,\, y\in \Orb(x)\implies \exists\, g\in G\,\colon \Stab(y) = g\Stab(x)g^{-1}$.
\end{proposition}

\begin{theorem}[Relation orbite-stabilisateur - \textnormal{[R] 21}]
	Pour tout $x\in X$, $G/\Stab(x)$ et $\Orb(x)$ sont équipotents (cela reste vrai si $G$ est infini).
	Par conséquent,
	$$\card(G) = \card(\Stab(x))\card(\Orb(x))$$
\end{theorem}

\begin{theorem}[Équation aux classes - \textnormal{[R] 21}]Soit $\left\{x_1,\dots,x_r\right\}$ un système de représentants pour les orbites. Alors,
	$$\card X = \sum_{i=1}^{r} \card(\Orb(x_i)) = \sum_{i=1}^{r} \frac{\card G}{\card(\Stab(x_i))}$$
\end{theorem}

\begin{example}[\textnormal{[R] 22}]
	Si $\card G$ est une puissance d'un nombre premier, alors son centre $Z(G) := \left\{g\in G\mid \forall\, h\in G,\,ghg^{-1}=h\right\}$ n'est pas réduit à $\left\{1\right\}$.

	Corrolaire \textnormal{([R] 23)}: tout groupe d'ordre $p^2$ avec $p$ premier est abélien.
\end{example}


\begin{theorem}[Formule de Burnside - \textnormal{[R] 35}]
	L'action de $G$ sur $X$ possède $\frac{1}{\card G}\sum_{g\in G} \card(\Fix(g))$
	orbites, où $\Fix(g) = \left\{x\in X \mid g\cdot x = x\right\}$.
\end{theorem}

\begin{example}[\textnormal{[C] 132}]
	En moyenne, une permutation de $\llbracket 1,n\rrbracket$ tirée aléatoirement a $1$ point fixe.
\end{example}
\begin{example}[\textnormal{[C] 132}]
	Si $G$ n'est pas abélien, alors la probabilité de tirer simultanément deux éléments qui commutent vaut $\frac{k}{n}$, avec $k$ le nombre de classes de conjugaison de $G$.
\end{example}

\begin{theorem}[de Cauchy - \textnormal{[R] 23}]
	Soit $p$ un nombre premier. Si $p\mid \card G$, alors $G$ admet un élément d'ordre $p$.
\end{theorem}

\section*{II. Applications}
\subsection*{A. En géométrie : les isométries des polytopes}
\begin{theorem}[\textnormal{[R] 94}]
	L'ensemble des isométries du plan conservant un triangle équilatéral est un groupe isomorphe à $\mathfrak{S}_3$.
\end{theorem}

\begin{proposition}[\textnormal{[R] 82}]
	Soit $\mathcal{C}$ un cube. L'ensemble des isométries de l'espace conservant $\mathcal{C}$ est un groupe, noté $\text{Is}(\mathcal{C})$.
	On note $\text{Is}^+(\mathcal{C})$ le sous-groupe de $\mathcal{C}$ formé de rotations.
\end{proposition}

\begin{tcolorbox}[
    breakable, % Allows the theorem to split across pages
    colback=developpement, % The background color
    colframe=gray!0!black, % The frame color
    boxrule=0pt, % The frame thickness
    arc=1mm, % Sharp corners
	boxsep=0pt,
	left=0pt, right=0pt, top=0pt, bottom=0pt
]
\begin{theorem}[\textnormal{[R] 85}]
	\label{dev1}
	$\text{Is}^+(\mathcal{C})\cong \mathfrak{S}_4$ et $\text{Is}(\mathcal{C})\cong \mathfrak{S}_4\times \IZ/2\IZ$.
\end{theorem}
\end{tcolorbox}

\begin{theorem}[\textnormal{[R] 95}]
	En notant $\mathcal{T}$ le tétraèdre régulier, on a $\text{Is}^+(\mathcal{T})\cong \mathcal{A}_4$ et $\text{Is}(\mathcal{T})\cong \mathfrak{S}_4$.
\end{theorem}

\subsection*{B. Du côté des matrices}
Dans ce paragraphe, $K$ désigne un corps. On fixe $(n,m)\in \left(\IN^*\right)^2$.

\begin{proposition}[\textnormal{[R] 184/185/199/195/206}]
	Les applications suivantes sont des actions :
	\begin{enumerate}
		\item Translation à gauche : $GL_n(K)\times \mathcal{M}_{n,m}(K)\to\mathcal{M}_{n,m}(K)$, $(P,A)\mapsto PA$
		\item Translation à droite : $GL_n(K)\times \mathcal{M}_{n,m}(K)\to\mathcal{M}_{n,m}(K)$, $(P,A)\mapsto AP^{-1}$
		\item Similitude (ou conjugaison) : $GL_n(K)\times \mathcal{M}_{n}(K)\to\mathcal{M}_{n}(K)$, $(P,A)\mapsto PAP^{-1}$
		\item Équivalence (ou \emph{action de Steiniz}) : $\left(GL_n(K)\times GL_m(K)\right)\times \mathcal{M}_{n,m}(K)\to\mathcal{M}_{n,m}(K)$, $\left(\left(P, Q\right), A\right) \mapsto PAQ^{-1}$
		\item Congruence : $GL_n(K)\times \mathcal{M}_{n}(K)\to\mathcal{M}_{n}(K)$, $(P,A)\to {}^tPAP$
	\end{enumerate}
\end{proposition}

\begin{proposition}[\textnormal{[R] 184/185/?/195/207}]
	Dans l'ordre de la proposition précédente, les orbites sont caractérisées par :
	\begin{enumerate}
		\item le noyau de $A$
		\item l'image de $A$
		\item les molynômes minimal et caractéristique de $A$
		\item Ça dépend de $K$...
	\end{enumerate}
\end{proposition}

\begin{example}
	$\text{Diag}(1,2,2)$ et $\text{Diag}(1,1,2)$ ont même polynôme minimal mais ne sont pas semblables : il faut donc bien les deux informations !
\end{example}

\subsection*{C. Théorèmes de Sylow}
Dans ce paragraphe, on se donne $p$ premier, et on note $\card G = p^{\alpha}m$, $m\wedge p = 1$.

\begin{definition}[\textnormal{[U] 85}]
	Un $p$-Sylow de $G$ est un sous-groupe de $G$ de cardinal $p^{\alpha}$.

	$\Syl_p(G)$ désigne l'ensemble des $p$-Sylow de $G$, et $n_p := \card(\Syl_p(G))$.
\end{definition}

\begin{theorem}[de Sylow - \textnormal{[U] 87}]Soit $G$ un groupe d'ordre $p^{\alpha}m$, $m\wedge p = 1$. Alors,
	\begin{enumerate}
		\item $\Syl_p(G)\neq \empty$
		\item $G$ agit transitivement sur $\Syl_p(G)$ par conjugaison
		\item $n_p\equiv 1\, [p]$
	\end{enumerate}
\end{theorem}

\begin{definition}
	On dit que $G$ est \emph{simple} si les seuls sous-groupes de $G$ distingués (\emph{i.e.} fixe par l'action par conjugaison de $G$)
	sont $\left\{1\right\}$ et $G$.
\end{definition}

\begin{tcolorbox}[
    breakable, % Allows the theorem to split across pages
    colback=developpement, % The background color
    colframe=gray!0!black, % The frame color
    boxrule=0pt, % The frame thickness
    arc=1mm, % Sharp corners
	boxsep=0pt,
	left=0pt, right=0pt, top=0pt, bottom=0pt
]
\begin{theorem}[\textnormal{[S] 277}]\label{dev2}
	Si $G$ est simple et d'ordre $60$, alors 
	$G\cong \mathcal{A}_5$.
\end{theorem}
\end{tcolorbox}

\section*{Développements}
\begin{itemize}
	\item Développement 1 : Théorème \ref{dev1}
	\item Développement 2 : Théorème \ref{dev2}
\end{itemize}

\section*{Références}
\begin{itemize}
	\item[U] \emph{Théorie des groupes}, Félix Ulmer
	\item[R] \emph{Mathématiques pour l'agrégation - Algèbre et géométrie}, Jean-Étienne Rombaldi, 2e édition
	\item[S] \emph{Algèbre pour la licence 3}, Szpirglas
	\item[C] \emph{Carnets de voyage en Algébrie}, Caldero
\end{itemize}

\begin{figure}[!htb]
	\centering
	\includegraphics[trim={0 0 0 0},clip,width=1\linewidth]{img/101.pdf}
\end{figure}


\chapter*{105 : Groupe des permutations d'un ensemble fini. Applications.}
\setcounter{definition}{0}
\section*{I. Permutations d'un ensemble fini}
\subsection*{A. Introduction}
\begin{definition}[\textnormal{[R] 37}]
	Soit $E$ un ensemble. On note $\mathfrak{S}(E)$
	l'ensemble des bijections de $E$ dans $E$. On l'appelle \emph{groupe symétrique} de $E$.
	On notera plus simplement $\mathfrak{S}_n = \mathfrak{S}(\llbracket 1, n\rrbracket)$.
	On appelle \emph{permutation} de $E$ un élément de $\mathfrak{S}(E)$.
\end{definition}

\begin{proposition}
	$\mathfrak{S}(E)$ est un groupe pour la composition, de neutre l'identité de $E$.
\end{proposition}

\begin{proposition}[\textnormal{[R] 39}]
	Si $E$ et $F$ sont deux ensembles équipotents, alors $\mathfrak{S}(E)$ et $\mathfrak{S}(F)$ sont isomorphes (en tant que groupes).
\end{proposition}

\begin{proposition}[\textnormal{[R] 39}]
	Pour $n\geq 3$, $\mathfrak{S}_3$ n'est pas commutatif.
\end{proposition}

Dans toute la suite, on étudiera $\mathfrak{S}_n$ pour $n\geq 3$.

\begin{proposition}[\textnormal{[R] 40}]
	$\#\mathfrak{S}_n = n!$
\end{proposition}

\begin{notation*}[\textnormal{[U] 41}]
	Soit $\sigma\in\mathfrak{S}_n$. On représentera $\sigma$ par la matrice $2\times n$ :
	\begin{align*}
		\sigma = \left(\begin{smallmatrix} 
			1 & 2 & \cdots & n\\
			\sigma(1) & \sigma(2) & \cdots & \sigma(n)
		\end{smallmatrix}\right)
	\end{align*}
\end{notation*}

\subsection*{B. Action naturelle de $\mathfrak{S}_n$ sur $\llbracket 1,n\rrbracket$, conséquences}

\begin{proposition}[\textnormal{[U] 41}]
		$\mathfrak{S}_n$ agit naturellement sur $\llbracket 1,n\rrbracket$ par $\sigma \cdot i = \sigma(i)$.
		Le morphisme associé est l'identité de $\mathfrak{S}_n$.
\end{proposition}

\begin{definition}[\textnormal{[U] 42}]
	On note $\Fix(\sigma)$ l'ensemble des points fixes de $\sigma\in\mathfrak{S}_n$.
	Son complémentaire dans $\llbracket 1,n\rrbracket$ est appelé \emph{support} de $\sigma$, et est noté $\Supp(\sigma)$.
\end{definition}

\begin{proposition}[\textnormal{[U] 43}]
	Soit $\sigma \in \mathfrak{S}_n$. Le sous-groupe $\langle\sigma\rangle$ agit sur $\llbracket 1,n\rrbracket$ par restriction
	de l'action de $\mathfrak{S}_n$. Les orbites de cette action sont appelées \emph{$\sigma$-orbites}.
	La réunion des $\sigma$-orbites ponctuelles est $\Fix(\sigma)$. Les $\sigma$-orbites non ponctuelles partitionnent $\Supp(\sigma)$.
\end{proposition}

\begin{example}
	Soit $\sigma = \left(\begin{smallmatrix} 
			1 & 2 & 3 & 4 & 5 \\
			2 & 1 & 3 & 5 & 4
		\end{smallmatrix}\right)$.
	On a $\Supp(\sigma)=\left\{1,2\right\}\sqcup \left\{4,5\right\}=\langle\sigma\rangle\cdot \left\{1\right\}\sqcup \langle\sigma\rangle\cdot\left\{4\right\}$.
\end{example}

\begin{definition}[\textnormal{[U] 43}]
	Un \emph{$k$-cycle} ($2\leq k \leq n$) est une permutation n'ayant qu'une seule $\sigma$-orbite non ponctuelle $\left\{i_1,\dots,i_k\right\}$.
	On la note $\sigma = (i_1,\dots,i_k)$ pour signifier que 
	$\forall j\notin \left\{i_1,\dots,i_k\right\}$, $\sigma(j)=j$ et $\sigma(i_j)=i_{j+1}$ en regardant les indices modulo $k$.

	Un $2$-cycle est appelé \emph{transposition}.
\end{definition}

\begin{proposition}[\textnormal{[U] 43}]
	$(i_1,i_2,\dots,i_k) = (i_2,i_3,\dots,i_k,i_1)=\dots=(i_k,i_1,i_2,\dots,i_{k-1})$
\end{proposition}

\begin{proposition}
	Un $k$-cycle est d'ordre $k$.
\end{proposition}

\subsection*{C. Décomposition d'une permutation, conséquences}

\begin{proposition}[\textnormal{[U] 42}]
	Deux permutations à supports disjoints commutent.
\end{proposition}

\begin{theorem}[\textnormal{[U] 43}]
	Toute permutation se décompose de manière unique (à l'ordre des facteurs près) comme produit de cycles à supports disjoints.
\end{theorem}

\begin{algorithm}[\textnormal{[U] 43}]
	Pour trouver une telle décomposition, il suffit de trouver les $r$-orbites.
	\begin{enumerate}
		\item On calcule $\sigma(1), \sigma^2(1),\dots$ justqu'à trouver $\sigma^{k_1}(1)=1$ (NB : $k_1\leq n$) ;
		\item On pose $i_2 = \min \llbracket 1,n\rrbracket \setminus (\langle\sigma\rangle\cdot\left\{1\right\})$, et de même on calcule $\sigma(i_2),\sigma^2(i_2),\dots$ jusqu'à trouver $\sigma^{k_2}(i_2)=i_2$ ;
		\item On itère jusqu'à épuiser $\llbracket 1,n\rrbracket$.
	\end{enumerate}
	On a alors $\sigma = (1,\sigma(1),\dots,\sigma^{k_1-1}(1))\circ (i_2, \sigma(i_2),\dots,\sigma^{k_2-1}(i_2))\circ \dots \circ(i_j, \sigma(i_j),\dots,\sigma^{k_j-1}(i_j))$
\end{algorithm}

\begin{example}
	$\sigma = \left(\begin{smallmatrix} 
			1 & 2 & 3 & 4 & 5 & 6 \\
			3 & 2 & 4 & 1 & 6 & 5
		\end{smallmatrix}\right) = (1,3,4)(5,6)$
\end{example}

\begin{proposition}[\textnormal{[R] 44}]
		$(i_1,\dots,i_k) = (i_1,i_2)(i_2,i_3)\dots(i_{k-1},i_k)$
\end{proposition}
\begin{corollary}[\textnormal{[R] 44}]
	Les transpositions engendrent $\mathfrak{S}_n$.
\end{corollary}
\begin{proposition}[\textnormal{[R] 45}]
	$\mathfrak{S}_n = \langle(i,i+1),\, 1\leq i\leq n\rangle = \langle (1,i), \, 2\leq i \leq n\rangle = \langle(1,2),\, (1,2,\dots, n) \rangle$
\end{proposition}

\begin{definition}[\textnormal{[U] 45}]
	On appelle \emph{type} de $\sigma\in\mathfrak{S}_n$ la liste croissante des cardinaux des $\sigma$-orbites.
\end{definition}

\begin{example}
	Le type de $(1,2,5)(3,4)(7,8)\in\mathfrak{S}_8$ est la liste $\left[1,2,2,3\right]$.
\end{example}

\begin{proposition}[\textnormal{[U] 46}]
	Deux permutations sont conjuguées dans $\mathfrak{S}_n$ si, et seulement si, elles ont le même type.
	Cela décrit donc les classes de conjugaison de $\mathfrak{S}_n$.
\end{proposition}

\begin{proposition}[\textnormal{[U] 45}]
	Si $\sigma$ est du type $\left[l_1,\dots,l_k\right]$, alors $\ord(\sigma) = l_1 \vee \dots \vee l_k$.
\end{proposition}

\subsection*{D. Signature dune permutation, groupe alterné}

\begin{proposition}[\textnormal{[R] 47}]
	Il existe un unique morphisme $\varepsilon\,\colon \mathfrak{S}_n \to \left\{\pm 1\right\}$ qui envoie
	les transpositions sur $-1$. On appelle \emph{signature} de $\sigma$ la quantité $\varepsilon(\sigma)$.
\end{proposition}

\begin{corollary}
	La signature d'un $k$-cycle est $(-1)^{k+1}$.
\end{corollary}

\begin{proposition}[\textnormal{[R] 48}]
	$\forall\sigma\in\mathfrak{S}_n$, 
	$$\varepsilon(\sigma) = \prod_{1\leq i\leq j\leq n}\frac{\sigma(j) - \sigma(i)}{j-i}$$
	En particulier, la signature mesure le nombre d'inversions.
\end{proposition}

\begin{definition}[\textnormal{[R] 48}]
	On appelle \emph{$n$-ième groupe alterné} le sous-groupe $\mathcal{A}_n = \Ker(\varepsilon)$.
	C'est l'ensemble des permutations dîtes \emph{paires}.
\end{definition}

\begin{example}
	$\mathcal{A}_3 = \left\{\textnormal{id},\, (1,2,3),\, (1,3,2)\right\}$.
\end{example}

\begin{proposition}
	$\#\mathcal{A}_n = \frac{n!}{2}$
\end{proposition}

\begin{theorem}[\textnormal{[R] 49}]
	Pour $n\geq 3$, les $3$-cycles engendrent $\mathcal{A}_n$, et y sont conjugués.
\end{theorem}

\begin{theorem}[\textnormal{[R] 50}]
	Pour $n\geq 5$, $\mathcal{A}_n$ n'admet pas de sous-groupe distingué non trivial.
\end{theorem}

\section*{II. Quelques applications du groupe symétrique}
\subsection*{A. En géométrie : les isométries des polytopes}

\begin{theorem}[\textnormal{[R] 94}]
	L'ensemble des isométries du plan conservant un triangle équilatéral est un groupe isomorphe à $\mathfrak{S}_3$.
\end{theorem}

\begin{proposition}[\textnormal{[R] 82}]
	Soit $\mathcal{C}$ un cube. L'ensemble des isométries de l'espace conservant $\mathcal{C}$ est un groupe, noté $\Is(\mathcal{C})$.
	On note $\Is^+(\mathcal{C})$ le sous-groupe de $\Is(\mathcal{C})$ formé des rotations.
\end{proposition}

\begin{tcolorbox}[
    breakable, % Allows the theorem to split across pages
    colback=developpement, % The background color
    colframe=gray!0!black, % The frame color
    boxrule=0pt, % The frame thickness
    arc=1mm, % Sharp corners
	boxsep=0pt,
	left=0pt, right=0pt, top=0pt, bottom=0pt
]
\begin{theorem}[\textnormal{[R] 85}]
	\label{105dev1}
	$\Is^+(\mathcal{C}) \cong \mathfrak{S}_4$ et $\Is(\mathcal{C})\cong \mathfrak{S}_4\times \IZ/2\IZ$.
\end{theorem}
\end{tcolorbox}

\begin{theorem}[\textnormal{[R] 95}]
	En notant $\mathcal{T}$ le tétraèdre régulier, on a :
	$\Is(\mathcal{T})\cong \mathfrak{S}_4$ et $\Is^+(\mathcal{T})\cong \mathcal{A}_4$.
\end{theorem}

\subsection*{Chez les (actions de) groupes}
\begin{theorem}[de Cayley - \textnormal{[R] 53}]
	Tout groupe fini d'ordre $n$ est isomorphe à un sous-groupe de $\mathfrak{S}_n$.
\end{theorem}
\begin{proposition}
	Comme pout tout corps (commutatif) $K$, $\mathfrak{S}_n\actson GL_n(K)$, tout groupe de garde $n$ 
	est isomorphe à un sous-groupe de $GL_n(K)$.
\end{proposition}

\begin{example}
	Soit $D_{2\times 4}$ le groupe des isométries du carré. Comme $\#D_{2\times 4} = 8$, $D_{2\times 4}$ est isomorphe à un sous-groupe de $\mathfrak{S}_8$. Noton $\varphi$ un tel isomorphisme.
	Comme $D_{2\times 4} = \langle r, s\rangle$ où $\ord(r) = 4$, $\ord(s) = 2$ et $\ord(rs) = 2$, on a $\varepsilon\circ \varphi(s)=\varepsilon\circ \varphi(rs) = -1$, donc $\varepsilon\circ \varphi(r) = 1$.
\end{example}

\subsection*{C. Polynômes symétriques}
\begin{definition}[\textnormal{[R] 55}]
	Un \emph{polynôme symétrique} est un polynôme $P\in K[X_1,\dots,X_n]$
	tel que $\forall\sigma\in\mathfrak{S}_n$, $P(X_{\sigma(1)},\dots,X_{\sigma(n)}) = P(X_1,\dots,X_n)$.
\end{definition}

\begin{definition}[\textnormal{[R] 55}]
	Les \emph{polynômes symétriques élémentaires} sont les 
	\begin{align*}
		\Sigma_{k,n} = \sum_{1\leq i_1\leq\dots\leq i_k\leq n} X_{i_1}\dots X_{i_k}\in K[X_1,\dots,X_n]
	\end{align*}
\end{definition}

\begin{theorem}[ADMIS - \textnormal{[R] 55}]
	Pour tout polynôme symétrique $P\in K[X_1,\dots,X_n]$, il
	existe un unique polynôme $Q\in K[X_1,\dots,X_n]$ tel que 
	$P(X_1,\dots,X_n) = Q(\Sigma_{1,n},\dots,\Sigma_{n,n})$.
\end{theorem}

\subsection*{D. En algèbre (multi-)linéaire}
Dans ce paragraphe, $E$ est un $\mathbb{K}$-espace vectoriel de dimension finie $n$.
On fixe une base $\mathcal{B} = (e_1,\dots,e_n)$ de $E$.
\begin{definition}[\textnormal{[R] 545}]
	Une \emph{forme $k$-linéaire} sur $E$ est une application $\varphi \,\colon E^k\to \mathbb{K}$ telle que pour tout $i\in\llbracket 1,n\rrbracket$, pour tout $(x_1,\dots, x_k)\in E^k$,
	$\varphi(x_1,\dots,x_{i-1}, \cdot, x_{i+1}, \dots, x_k)$ est linéaire.

	On note $\bigotimes^k E^*$ l'ensemble des formes $k$-linéaires sur $E$.
\end{definition}

\begin{proposition}[\textnormal{[R] 546}]
	$\left(e_{i_1}^*\otimes\dots\otimes e_{i_k}^*\right)_{1\leq i_1<\dots < i_k\leq n}$ est une base de $\bigotimes^kE^*$,
	où pour $(x_1,\dots, x_k)\in E^k$, $e_{i_1}^*\otimes\dots\otimes e_{i_k}^*(x_1,\dots, x_k) = e_{i_1}^*(x_1)\dots e_{i_k}^*(x_k)$.
\end{proposition}

\begin{definition}[\textnormal{[R] 546}]
	Une forme $k$-linéaire \emph{alternée} est une forme $k$-linéaire $\varphi\in\bigotimes^kE^*$
	telle que $\forall\sigma\in\mathfrak{S}_k$, $\forall(x_1,\dots, x_k)\in E^k$, $\varphi(x_{\sigma(1)},\dots, x_{\sigma(k)}) = \varepsilon(\sigma)\varphi(x_1,\dots, x_k)$.

	On note $\bigwedge^k E^*$ l'espace des formes $k$-linéaires alternées sur $E$.
\end{definition}

\begin{proposition}
	$\left(e_{i_1}^*\wedge \dots \wedge e_{i_k}^*\right)_{1\leq i_1 < \dots < i_k\leq n}$ est une base de $\bigwedge^kE^*$, où pour $(x_1,\dots,x_k)\in E^k$,
	$e_{i_1}^*\wedge \dots \wedge e_{i_k}^*(x_1,\dots,x_k) = \sum_{\sigma\in\mathfrak{S}_k} \varepsilon(\sigma) e_{i_1}^*(x_{\sigma(1)}) \dots e_{i_k}^*(x_{\sigma(k)})$.
\end{proposition}

\begin{corollary}
	On a $\dim\left(\bigwedge^kE^*\right) = {n \choose k}$.
\end{corollary}

\begin{definition}
	On appelle \emph{déterminant dans la base $\mathcal{B}$} l'unique forme $n$-linéaire alternée $\det_{\mathcal{B}}$ sur $E$ vérifiant
	$\det_{\mathcal{B}}(\mathcal{B}) = 1$. (La fammille $\left(\det_{\mathcal{B}}\right)$ est une base de $\bigwedge^n E^*$.)
\end{definition}

\begin{proposition}[\textnormal{[R] 547}]
	$\forall(x_1,\dots, x_n)\in E^n$, $\det_{\mathcal{B}}(x_1,\dots,x_n) = \sum_{\sigma\in\mathfrak{S}_n} \varepsilon(\sigma)e_1^*(x_{\sigma(1)})\dots e_n^*(x_{\sigma(n)})$.
\end{proposition}

\subsection*{E. Résultats en probabilités}

\begin{definition}[\textnormal{[R] 51}]
	On appelle \emph{dérangement} une permutation sans point fixes.
\end{definition}

\begin{proposition}
	Notons $d_n$ le nombre de dérangements de $\llbracket 1,n\rrbracket$.
	Alors $d_n = n!\sum_{k=0}^{n}\frac{(-1)^k}{k!}$. En particulier, la probabilité
	de choisir un dérangement en tiant au hasard une permutation de $\llbracket 1,n\rrbracket$ tend vers $\frac{1}{e}$ quand $n\to +\infty$.
\end{proposition}

\begin{proposition}[\textnormal{[C]}]
	Soit $X$ la variable aléatoire qui compte le nombre de points fixes d'une permutation aléatoirement choisie dans $\mathfrak{S}_n$.
	Alors $\mathbb{E}[X] = \mathbb{V}[X] = 1$.
\end{proposition}

\subsection*{F. Groupes simples d'ordre 60}
Dans ce paragraphe, on se donne $p$ premier, et on note $\#G = p^{\alpha}m$, $m\wedge p = 1$.

\begin{definition}[\textnormal{[U] 85}]
	Un \emph{$p$-Sylow} de $G$ est un sous-groupe de $G$ de cardinal $p^{\alpha}$.
\end{definition}

\begin{notation*}
	$\Syl_p(G)$ désigne l'ensemble des $p$-Sylow de $G$, et $n_p =\#\Syl_p(G)$.
\end{notation*}

\begin{theorem}[de Sylow - \textnormal{[U] 87}]
	Soit $G$ un groupe d'ordre $p^{\alpha}m$, $p$ premier et $m\wedge p = 1$.
	\begin{enumerate}
		\item $\Syl_p(G)\neq \emptyset$
		\item $G$ agit transitivement sur $\Syl_p(G)$ par conjugaison
		\item $n_p \equiv 1\,[p]$ (donc $n_p\mid m$).
	\end{enumerate}
\end{theorem}

\begin{definition}
	On dit que $G$ est \emph{simple} si les seuls sous-groupes de $G$ distingués (\emph{i.e.} fixe par l'action par conjugaison de G) sont $\left\{1\right\}$ et $G$.
\end{definition}

\begin{tcolorbox}[
    breakable, % Allows the theorem to split across pages
    colback=developpement, % The background color
    colframe=gray!0!black, % The frame color
    boxrule=0pt, % The frame thickness
    arc=1mm, % Sharp corners
	boxsep=0pt,
	left=0pt, right=0pt, top=0pt, bottom=0pt
]
\begin{theorem}[\textnormal{[S] - 277}]
	\label{105dev2}
	Si $G$ est simple et d'ordre $60$, alors $G\cong \mathcal{A}_5$.
\end{theorem}
\end{tcolorbox}

\section*{Développements}
\begin{itemize}
	\item Développement 1 : Théorème \ref{105dev1}
	\item Développement 2 : Théorème \ref{105dev2}
\end{itemize}

\section*{Références}
\begin{itemize}
	\item[R] \emph{Mathématiques pour l'agrégation - Algèbre et géométrie}, Jean-Étienne Rombaldi, 2e édition
	\item[U] \emph{Théorie des groupes}, Félix Ulmer
	\item[S] \emph{Algèbre pour la licence 3}, Szpirglas
	\item[C] \emph{Carnets de voyage en Algébrie}, Caldero
\end{itemize}

\begin{figure}[!htb]
	\centering
	\includegraphics[trim={0 0 0 0},clip,width=1\linewidth]{img/101.pdf}
	\caption{Isométries du cube}
\end{figure}

\chapter*{106 : Groupe linéaire d'un espace vectoriel de dimension finie E, sous-groupes de $GL(E)$. Applications}
\setcounter{definition}{0}

Dans cette leçon, $K$ est un corps commutatif, et $E$ est un $K$-espace vectoriel de dimension finie $n\geq 1$.
\section*{I. Endomorphismes inversibles d'un espace vectoriel}
\subsection*{A. Introduction au groupe linéaire}
\begin{theorem}[\textnormal{[Rb] 139}]
	\begin{itemize}
		\item L'ensemble $\mathcal{L}(E)$ des endomorphismes de $E$ est un anneau pour $+$ et $\circ$, dont le groupe des inversibles est noté $GL(E)$, et est appelé \emph{groupe linéaire} de $E$.
		\item Similairement, l'ensemble $\mathcal{M}_n(K)$ des matrices carrées de taille $n\times n$ est un anneau pour $+$ et $\times$, dont le groupe des inversibles est noté $GL_n(K)$, appelé \emph{groupe linéaire d'ordre $n$ sur $K$}.
	\end{itemize}
\end{theorem}

\begin{remark}[\textnormal{[Rb] 140}]
	Étant donnée une base $\mathcal{B}$, l'application $u\mapsto \Mat_{\mathcal{B}}(u)$ induit un isomorphisme entre $GL(E)$ et $GL_n(K)$.
\end{remark}

\begin{definition}[\textnormal{[Rb] 141}]
	On note $SL(E)$ (resp. $SL_n(K)$) le noyau du morphisme $\det$ de $GL(E)$ (resp. $GL_n(K)$) dans $K^{\times}$.
	On l'appelle \emph{groupe spécial linéaire de $E$} (resp. \emph{groupe spécial linéaire d'ordre $n$ sur $K$}).
\end{definition}

\begin{theorem}[\textnormal{[Rb] 140}]
	Soit $u\in\mathcal{L}(E)$. Comme $\dim E < +\infty$, sont équivalentes :
	\begin{enumerate}
		\item $u\in GL(E)$
		\item {\begin{enumerate}
			\item $u$ est injectif
			\item $\Ker u = \left\{0\right\}$
			\item $\exists v\in \mathcal{L}(E)\,\colon v\circ u = \id_E$
		\end{enumerate}}
		\item {\begin{enumerate}
			\item $u$ est surjectif
			\item $\im u = E$
			\item $\exists v\in \mathcal{L}(E)\,\colon u\circ v = \id_E$
		\end{enumerate}}
		\item L'image par $u$ d'une base de $E$ est une base de $E$
		\item $\det(u) \neq 0$
	\end{enumerate}
\end{theorem}

\begin{remark}
	Un matrice $A$ est inversible si, et seulement si, ses colonnes forment une base de $K^n$, et si, et seulement si, ses lignes forment une base de $K^n$.
\end{remark}

\begin{definition}
	On dit que $u\in\mathcal{L}(E)$ est une \emph{homothétie de rapport $\lambda\in K^{\times}$} si $\forall x\in E$, $u(x) = \lambda x$.
\end{definition}

\begin{proposition}
	Une homothétie de rapport $\lambda\in K^{\times}$ est inversible, d'inverse l'homothétie de rapport $1/\lambda$.
\end{proposition}

\begin{proposition}[\textnormal{[Rb] e168}]
	Les homothéties sont les seuls endomorphismes à stabiliser toute droite.
\end{proposition}

\subsection*{B. Opérations élémentaires}
Soit $A\in\mathcal{M}_n(K)$. On note $L_1, \dots, L_p$ les lignes de $A$, et $C_1,\dots,C_n$ ses colonnes.
\begin{definition}[\textnormal{[Bu] 315-317}]
	Soient $\alpha \in K^{\times}$, $(i,j)\in\llbracket 1,n\rrbracket^2$ tel que $i\neq j$ et $\sigma\in\mathfrak{S}_n$. On définit les matrices suivantes :
	\begin{itemize}
		\item Matrice de \emph{dilatation} $D_i(\alpha) = \diag(1,\dots, 1, \alpha,1,\dots, 1)\in GL_n(K)$ ($\alpha$ est à la $i$-ième position)
		\item Matrice de \emph{transvection} $T_{i,j}(\alpha) = I_n + \alpha E_{i,j}\in GL_n(K)$
		\item Matrice de \emph{permutation} $P_{\sigma} = \left(\delta_{i,\sigma(j)}\right)_{1\leq i,j\leq n}\in GL_n(K)$
	\end{itemize}
\end{definition}

\begin{definition}
	On définit les \emph{opérations élémentaires} sur les colonnes :
	\begin{itemize}
		\item $C_i \longleftarrow \alpha C_i$ : on remplace $C_i$ par $\alpha C_i$
		\item $C_i \longleftarrow C_i + \alpha C_j$ : on remplace $C_i$ par $\alpha C_i+\alpha C_j$
		\item $C_i \longleftrightarrow C_j$ : on échange $C_i$ et $C_j$
	\end{itemize}
\end{definition}

\begin{theorem}[\textnormal{[Bu] 315-318}]
	On a les correspondances suivantes entre opérations élémentaires et multiplication matricielle :
	\begin{itemize}
		\item $D_i(\alpha)A\iff L_i \longleftarrow \alpha L_i$
		\item $T_{i,j}(\alpha)A\iff L_i \longleftarrow L_i \alpha L_j$
		\item $P_{(i,j)}A\iff L_i \longleftrightarrow L_j$
	\end{itemize}
	et 
	\begin{itemize}
		\item $A D_i(\alpha)\iff C_i \longleftarrow \alpha C_i$
		\item $AT_{i,j}(\alpha)\iff C_i \longleftarrow C_i \alpha C_j$
		\item $AP_{(i,j)}\iff C_i \longleftrightarrow C_j$
	\end{itemize}
\end{theorem}

\begin{proposition}
	$\sigma\mapsto P_{\sigma}$ est un morphisme de groupes injectif de $\mathfrak{S}_n$ dans $GL_n(K)$.
\end{proposition}

\section*{II. Structure de $GL(E)$, sous-groupe orthogonal}
\subsection*{A. Structure de groupe}

\begin{theorem}[Pivot de Gauss - \textnormal{[Rb] 191}]
	Pour toute matrice de rang $r$, il existe une suite d'opérations élémentaires qui transforme cette matrice en la matrice $J_{n,r} = \diag(I_r, O_{n-r})$.
	Plus précisément, si $\rg A = n$, alors il existe $\sigma\in\mathfrak{S}_n$ et des matrices de transvection $T_1, \dots, T_p$ telles que $A = P_{\sigma}T_1\dots T_p D_{\alpha}$ où $D_{\alpha}$ est la matrice de dilatation $D_{\alpha}$ de rapport $\alpha = \det A$.
\end{theorem}

\begin{corollary}[\textnormal{[Rb] 154, 153}]
	\begin{itemize}
		\item Les matrices de transvection et de dilatation engendrent $GL_n(K)$ ;
		\item Les matrices de transvection engendrent $SL_n(K)$.
	\end{itemize}
\end{corollary}

\begin{corollary}[\textnormal{[Rb] 141}]
	$GL(E) / SL(E) \cong K^{\times}$
\end{corollary}

\begin{corollary}[\textnormal{[Rb] 141}]
	\begin{itemize}
		\item $Z(GL(E)) = K^{\times}\id_E$ (c'est l'ensemble des homothéties) ;
		\item $Z(SL(E)) = \mathbb{U}_n(K)\id_E$, où $\mathbb{U}_n(K)  \left\{\lambda\in K^{\times} \mid \lambda^n = 1\right\}$.
	\end{itemize}
\end{corollary}

\subsection*{B. Le groupe spécial orthogonal}
Soit $q$ une forme quadratique sur $E$, de forme polaire $\varphi$. Supposons $\car K \neq 2$.

\begin{definition}[\textnormal{[P] 123-124}]
	\begin{itemize}
		\item Le \emph{groupe orthogonal} de $(E,q)$ est $O(q) = \left\{u\in\mathcal{L}(E)\mid q\circ u = q\right\}$
		\item Le \emph{groupe spécial orthogonal} de $(E,q)$ est $SO(q) = \left\{u\in O(q)\mid \det u = 1\right\}$
		\item Lorsque $\varphi$ est le produit scalaire canonique relativement à une base donnée, on note $O(E)=O(q)=\left\{u\in\mathcal{L}(E)\mid {}^tu\circ u = \id_E\right\}$ et $SO(E)=SO(q)=\left\{u\in O(E)\mid \det u = 1\right\}$.
		\item On note également $O_n(K) = \left\{M\in\mathcal{M}_n(K)\mid {}^tMM=I_n\right\}$ et $SO_n(K)  \left\{M\in O_n(K)\mid \det M = 1\right\}$.
	\end{itemize}
\end{definition}

\begin{proposition}[\textnormal{[Rb] 722}]
	Si $\mathcal{B}$ est une base orthonormale de $E$, alors $u\in O(E)\iff \Mat_{\mathcal{B}}(u)\in O_n(K)$.
\end{proposition}

\begin{theorem}[de réduction des isométries - \textnormal{[Rb] 727}]
	Soit $u\in O(\IRn)$. Il existe une base orthonormale $\mathcal{B}$ de $\IRn$ telle que $\Mat_{\mathcal{B}}(u)=\diag(R(\theta_1),\dots,R(\theta_r),\varepsilon_1,\dots,\varepsilon_p)$
	où $R(\theta_i) = \begin{pmatrix}
		\cos \theta_i & -\sin \theta_i \\
		\sin \theta_i & \cos \theta_i
	\end{pmatrix}$ et $\varepsilon_i = \pm 1$.
\end{theorem}

\begin{remark}[\textnormal{[P] 146}]
	$SO_2(\IR) = \left\{R(\theta)\mid \theta\in\IR\right\}\cong \IR/2\pi\IZ$.
\end{remark}

\begin{tcolorbox}[
    breakable, % Allows the theorem to split across pages
    colback=developpement, % The background color
    colframe=gray!0!black, % The frame color
    boxrule=0pt, % The frame thickness
    arc=1mm, % Sharp corners
	boxsep=0pt,
	left=0pt, right=0pt, top=0pt, bottom=0pt
]
\begin{theorem}[\textnormal{[C] 50}]
	\label{106dev1}
	Soient $p$ premier, $r\geq 1$ et $q = p^r$.
	\begin{align*}
		SO_2(\IF_q) \cong \begin{cases}
			\IZ/(q-1)\IZ\quad\text{si $-1$ est un carré mod $q$} \\
			\IZ/(q+1)\IZ\quad \text{sinon}
		\end{cases}
	\end{align*}
\end{theorem}
\end{tcolorbox}

\begin{definition}[\textnormal{[P] 125}]
	Soit $u\in O(q)$ telle que $u^2 = \id_E$.
	
	On dit que $u$ est une \emph{réflexion} si $\dim(\Ker(u+\id_E)) = 1$, \emph{i.e.} si $u$ est une symétrie par rapport à un hyperplan.
	
	On dit que $u$ est une \emph{renversement} si $\dim(\Ker(u+\id_E)) = 2$, \emph{i.e.} si $u$ est une symétrie par rapport à un plan.
\end{definition}


\begin{tcolorbox}[
    breakable, % Allows the theorem to split across pages
    colback=developpement, % The background color
    colframe=gray!0!black, % The frame color
    boxrule=0pt, % The frame thickness
    arc=1mm, % Sharp corners
	boxsep=0pt,
	left=0pt, right=0pt, top=0pt, bottom=0pt
]
On suppose désormais que $E$ est un $\IR$-espace vectoriel de dimension finie $n\geq 1$, et que $q$ est définie positive.
\begin{theorem}[\textnormal{[P] 143}]
	\label{106dev21}
	Tout élément de $O(q)$ est produit d'au plus $n$ réflexions.
\end{theorem}
\begin{lemma}
	\label{106dev22}
	Si $n\geq 3$, alors pour toutes réflexions $\tau_1$ et $\tau_2$, il existe deux renversements $\sigma_1$ et $\sigma_2$ tels que $\tau_1\tau_2 = \sigma_1\sigma_2$.
\end{lemma}
\begin{theorem}
	\label{106dev23}
	Pour $n\geq 3$, tout élément de $SO(q)$ est produit d'au plus $n$ renversements.
\end{theorem}
\end{tcolorbox}

\begin{remark}
	Ces théorèmes restent vrais si $E$ est un espace vectoriel de dimension finie sur un corps $K$ de caractéristique $\neq 2$, et si $q$ est non dégénérée (Cartan, Dieudonné).
\end{remark}

\section*{III. Topologie dans $GL(E)$}
Dans ce paragraphe, $K$ désigne $\IR$ ou $\IC$.

\begin{proposition}[\textnormal{[Rb] 160-161}]
	$GL(E)$ est ouvert dans $(\mathcal{L}(E), \vertiii\cdot)$ et $u\mapsto u^{-1}$ est continue.
\end{proposition}

\begin{proposition}
	\begin{itemize}
		\item $GL_n(\IC)$ et $SL_n(K)$ sont connexes ;
		\item $GL_n(\IR)$ a deux composantes connexes.
	\end{itemize}
\end{proposition}

\begin{proposition}
	$O_n(\IR)$ et $SO_n(\IR)$ sont compacts.
\end{proposition}

\begin{theorem}[Décomposition polaire - \textnormal{[Rb] 740}]
		\begin{align*}
			O_n(\IR)\times S_n^{++}(\IR) &\to GL_n(\IR) \\
			(H,S) &\mapsto HS
		\end{align*}
		est un homéomorphisme.
\end{theorem}

\section*{Développements}
\begin{itemize}
	\item Développement 1 : Théorème \ref{106dev1}
	\item Développement 2 : Théorème \ref{106dev21}, Lemme \ref{106dev22} et Théorème \ref{106dev23}
\end{itemize}

\section*{Références}
\begin{itemize}
	\item[Rb] \emph{Mathématiques pour l'agrégation - Algèbre et géométrie}, Jean-Étienne Rombaldi, 2e édition
	\item[P] \emph{Cours d'algèbre}, Perrin
	\item[B] \emph{Algèbre et géométrie: CAPES et Agrégation}, Pierre Burg
	\item[C] \emph{Nouvelles histoires hédonistes de groupes et géométries}, P. Caldero, J. Germoni
\end{itemize}


\chapter*{120 : Anneaux $\IZnZ$. Applications.}
Dans toute la leçon, $n\in\IN\setminus\left\{0,1\right\}$ et $p$ est un nombre premier.
\setcounter{definition}{0}
\section*{I. L'anneau $\IZnZ$}
\subsection*{A. Rappels d'arithmétique des entiers}
\begin{theorem}[division euclidienne - \textnormal{[R] 279}]
	$\forall (a,b)\in\IZ^2,\,\exists ! (q,r)\in\IZ^2 \,\colon$
	\begin{align*}
		\begin{cases}
			a=bq+r \\
			0\leq r< |b|
		\end{cases}
	\end{align*}
\end{theorem}

\begin{definition}[\textnormal{[R] 279}]
	Soit $(a,b)\in\IZ^2$. On dit que $a$ est \emph{congru à} $b$ modulo $n$,
	et on note $a\equiv b \, [n]$ si $n$ divise $b-a$.
\end{definition}

\begin{proposition}[\textnormal{[R] 280}]
	Soit $(a,b,c,d)\in\IZ^4$ tel que $a\equiv b\,[n]$ et $c\equiv d\,[n]$.
	Alors $a+c\equiv b+d\,[n]$ et $ac\equiv bd \,[n]$.
\end{proposition}

\subsection*{B. Construction}
\begin{lemma}
	Tout idéal de $\IZ$ est principal, et admet un unique générateur positif.
\end{lemma}

\begin{definition}[\textnormal{[R] 280}]
	Le quotient de l'anneau $(\IZ, +,\times)$ par son idéal $n\IZ$ est l'anneau noté $\IZnZ$. On note $\overline{a}$ l'image de $a\in\IZ$ dans $\IZnZ$.
\end{definition}

\begin{remark}
	$\overline{a} = \overline{b} \iff a\equiv b\,[n]$
\end{remark}

\begin{proposition}[\textnormal{[R] 280}]
	$\IZnZ = \left\{\overline{0},\overline{1},\dots,\overline{n-1}\right\}$, et les lois sont données par Prop 3 et Rq 6.
\end{proposition}

\begin{example}
	$\IZ/3\IZ = \left\{\overline{0},\overline{1},\overline{2}\right\} =\left\{\overline{9},\overline{64},\overline{-7}\right\}$,
	et on a $\overline{1}+\overline{2}=\overline{1+2}=\overline{3}=\overline{0}$, mais aussi $\overline{1}\times\overline{2} = \overline{1\times 2}=\overline{2}$.
\end{example}

\subsection*{C. Structure d'anneau}
\begin{proposition}[\textnormal{[R] 283}]
	L'ensemble des inversibles $\IZnZ$ est :
	$$\left(\IZnZ\right)^{\times} = \left\{\overline{k}\in\IZnZ\mid k\wedge n = 1\right\}$$

	L'ensemble des diviseurs de $O$ de $\IZnZ$ est :
	$$D_0\left(\IZnZ\right) = \IZnZ \setminus \left[\left(\IZnZ\right)^{\times}\cup \left\{0\right\}\right]$$
\end{proposition}

\begin{example}
	$\left(\IZ/8\IZ\right)^{\times} = \left\{\overline{1},\overline{3},\overline{5},\overline{7}\right\}$, et 
	$D_0\left(\IZ/8\IZ\right) = \left\{\overline{2},\overline{4},\overline{6}\right\}$.
\end{example}

\begin{proposition}[\textnormal{[R] 241 et 281}]
	Les idéaux propres de $\IZnZ$ sont les $d\IZnZ$ avec $d\mid n$, $d\notin \left\{1,n\right\}$.
	De plus, $\left(d\IZnZ, +\right)\cong \left(\IZ/\frac{n}{d}\IZ, +\right)$.
\end{proposition}

\begin{corollary}
	$\IZnZ$ est principal.
\end{corollary}

\begin{corollary}
	L'ensemble des générateurs de $\IZnZ$ est $\left(\IZnZ\right)^{\times}$.
\end{corollary}

\begin{example}
	Les idéaux propres de $\IZ/6\IZ$ sont $2\IZ/6\IZ$ et $3\IZ/6\IZ$, respectivement isomorphes à $\IZ/3\IZ$ et $\IZ/2\IZ$.
\end{example}

\begin{proposition}[\textnormal{[R] 295-282}]
	$\forall n,m\geq 2$, 
	$$\Hom_{gr}(\IZnZ, \IZ/m\IZ)\cong \IZ/(n\wedge m)\IZ,$$
	$$\Aut(\IZnZ)\cong \left(\IZnZ\right)^{\times},$$
	$$\Hom_{Ann}(\IZnZ, \IZ/m\IZ)\cong  \begin{cases}
		\left\{k\text{ mod } n \mapsto k \text{ mod } m\right\} \quad\text{si } m\mid n \\
		\emptyset\quad\text{sinon}
	\end{cases}$$
\end{proposition}

\subsection*{D. Le corps $\IZ/p\IZ$}
\begin{theorem}
	Les assertions suivantes sont équivalentes :
	\begin{enumerate}
		\item $\IZnZ$ est un corps ;
		\item $\IZnZ$ est intègre ;
		\item $n$ est premier.
	\end{enumerate}
\end{theorem}

\begin{corollary}[\textnormal{[R] 292}]
	$\left(\IZ/p\IZ\right)^{\times} \cong \IZ/(p-1)\IZ$.
\end{corollary}

\begin{cexample}
	C'est très faux pour $n$ non premier !
	$\left(\IZ/8\IZ\right)^{\times} = \left\{\overline{1},\overline{3},\overline{5},\overline{7}\right\}$ n'a même pas $7$ éléments !
\end{cexample}

\section*{II. Structure de $\left(\IZnZ\right)^{\times}$}
\subsection*{A. Préambule : le théorème des restes chinois}

\begin{tcolorbox}[
    breakable, % Allows the theorem to split across pages
    colback=developpement, % The background color
    colframe=gray!0!black, % The frame color
    boxrule=0pt, % The frame thickness
    arc=1mm, % Sharp corners
	boxsep=0pt,
	left=0pt, right=0pt, top=0pt, bottom=0pt
]
\begin{theorem}[des restes chinois - \textnormal{[R] 285}]
	\label{120dev1}
	Soit $(a_1,\dots, a_d)\in \left(\IN\setminus \left\{0,1\right\}\right)^d$.
	Les entiers, $a_1,\dots, a_d$ sont deux à deux premiers si, et seulement si, les anneaux 
	$\IZ/a_1\dots a_d\IZ$ et $\IZ/a_1\IZ \times \dots \times \IZ/a_d\IZ$ sont isomorphes.

	Le cas échéant, il existe $(u_1,\dots,u_d)\in\IZ^d$ tel que $\sum_{i=1}^{d} a_ib_i =1$, où 
	$b_i = \frac{a_1\dots a_d}{a_i}$. L'application :

	\begin{align*}
		\overline{\varphi} \,\colon \IZ/a_1\dots a_d\IZ &\to \IZ/a_1\IZ \times \dots \times \IZ/a_d\IZ \\
		x\,mod\,a_1\dots a_d &\mapsto (x\,mod\,a_1,\dots, x\, mod\, a_d)
	\end{align*}
	est un isomorphisme d'anneaux, de réciproque :
	\begin{align*}
		\overline{\varphi}^{-1}\,\colon (x_1\,mod\,a_1,\dots, x_d\, mod\, a_d) \mapsto \sum_{i=1}^{d} x_i a_i b_i\, mod\, a_1\dots a_d
	\end{align*}
\end{theorem}
\end{tcolorbox}

\subsection*{B. Fonction indicatrice d'Euler}
\begin{definition}[\textnormal{[R] 283}]
	\emph{L'indicatrice d'Euler} est :
	$\varphi\,\colon n \mapsto \#\left(\IZnZ\right)^{\times} = \#\left\{k\in\llbracket 1,n\rrbracket \mid k\wedge n = 1\right\}$.
\end{definition}

\begin{example}
	$\varphi(8) = 4$ d'après Exemple 10.
\end{example}

\begin{proposition}[\textnormal{[R] 288}]
	Si $a\wedge b=1$, alors $\varphi(ab)=\varphi(a)\varphi(b)$.
	Pour tout $\alpha \in \IN^*$, $\varphi(p^{\alpha}) = p^{\alpha - 1}(p-1)$.
\end{proposition}

\begin{corollary}[\textnormal{[R] 288}]
	Si $n=p_1^{\alpha_1}\dots p_r^{\alpha_r}$ est la décomposition de $n$ en produit de facteurs premiers,
	alors :
	$$\varphi(n) = \prod_{i=1}^{r} p^{\alpha_i - 1}(p-1) = n\prod_{i=1}^{r}\left(1 - \frac{1}{p_i}\right)$$
\end{corollary}

\begin{example}
	$\varphi(90) = \varphi(3^2)\varphi(2)\varphi(5) = 3(3-1)(2-1)(5-1) = 24$
\end{example}

\begin{theorem}[d'Euler - \textnormal{[R] 283}]
	Si $a\wedge n = 1$, alors $a^{\varphi(n)} \equiv 1\,[n]$.
\end{theorem}

\begin{theorem}[de Fermat - \textnormal{[R] 284}]
	Si $a\wedge p = 1$, alors $a^{p-1} = 1\, [p]$.
	De manière générale, $a^p\equiv a\,[p]$.
\end{theorem}

\begin{proposition}[\textnormal{[R] 284}]
	$$n = \sum_{d\mid n} \varphi(d)$$
\end{proposition}


\begin{tcolorbox}[
    breakable, % Allows the theorem to split across pages
    colback=developpement, % The background color
    colframe=gray!0!black, % The frame color
    boxrule=0pt, % The frame thickness
    arc=1mm, % Sharp corners
	boxsep=0pt,
	left=0pt, right=0pt, top=0pt, bottom=0pt
]
\begin{theorem}[\textnormal{[R] 292}]
	\label{120dev2}
	Si $p\geq 3$, alors $\forall \alpha \geq 1$, $\left(\IZ/p^{\alpha}\IZ\right)^{\times}$ est cyclique.
\end{theorem}
\end{tcolorbox}

\begin{theorem}[ADMIS - \textnormal{[R] 294}]
	$\left(\IZnZ\right)^{\times}$ est cyclique si, et seulement si, $n\in \left\{2,4,p^{\alpha}, 2p^{\alpha}\right\}$ avec $p\geq 3$ (premier) et $\alpha\geq 1$.
\end{theorem}

\section*{III. Applications}
\subsection*{A. Résolution de systèmes de congruence}
\begin{theorem}[\textnormal{[R] 290}]
	L'équation $ax\equiv b\,[n]$ d'inconnue $x\in\IZ$ admet des solutions si, et seulement si, $a\wedge n \mid b$.

	Le cas échéant, $S(ax\equiv b\,[n]) = \frac{b}{a\wedge n}x_0 + \frac{n}{a\wedge n}\IZ$, où $x_0$ est une solution particulière de l'équation.
\end{theorem}

\begin{remark}
	Le théorème des restes chinois permet de résoudre des systèmes de congruences.
\end{remark}


\begin{tcolorbox}[
    breakable, % Allows the theorem to split across pages
    colback=developpement, % The background color
    colframe=gray!0!black, % The frame color
    boxrule=0pt, % The frame thickness
    arc=1mm, % Sharp corners
	boxsep=0pt,
	left=0pt, right=0pt, top=0pt, bottom=0pt
]
\begin{example}[\textnormal{[R] 291}]
	\label{120dev12}
	$S\left(\begin{cases}
		x\equiv 2\,[4] \\
		x\equiv 3\,[5] \\
		x\equiv 1\,[9]
	\end{cases}\right) = 118 + 180\IZ$
\end{example}
\end{tcolorbox}

\begin{remark}[\textnormal{[R] 291}]
	$S\left(\begin{cases}
		x\equiv x_1\,[a_1] \\
		x\equiv x_2\,[a_2]
	\end{cases}\right) =\begin{cases}
		\emptyset\quad\text{si } a_1\wedge a_2 \nmid x_1-x_2 \\
		x_0 + (a_1\vee a_2)\IZ\quad\text{sinon}
	\end{cases}$
\end{remark}

\subsection*{B. Carrés de $\IZ/p\IZ$}
Soit $c\,\colon \overline{x}\in\IZ/p\IZ \mapsto \overline{x}^2$.
On s'intéresse à $\im c$.

\begin{proposition}
	Tous les éléments de $\IZ/2\IZ$ sont des carrés.
\end{proposition}
On supposera désormais $p\geq 3$.

\begin{proposition}[\textnormal{[R] 426}]
	Soit $l\,\colon \overline{x}\in\IZ/p\IZ \mapsto \overline{x}^{\frac{p-1}{2}}$.
	\begin{itemize}
		\item $\forall \overline{x}\in\IZ/p\IZ$, $c\circ l(\overline{x}) = l\circ c(\overline{x}) = \overline{1}$
		\item $\Ker c = \im l =\left\{\pm 1\right\}$ et $\im c = \Ker l$.
	\end{itemize}
\end{proposition}

\begin{corollary}
	Il y a $\frac{p+1}{2}$ carrés dans $\IZ/p\IZ$.
\end{corollary}

\begin{theorem}[de Wilson - \textnormal{[R] 325}]
	$n$ est premier $\iff (n-1)! \equiv -1\, [n]$
\end{theorem}

\begin{proposition}[\textnormal{[P] 75}]
	$-1$ est un carré modulo $p$ si, et seulement si, $p\equiv 1\,[4]$. Le cas échéant $-1\equiv (2\times 3\times\dots\times \frac{p-1}{2})^2\,[p]$.
\end{proposition}

\begin{theorem}[des deux carrés de Fermat - \textnormal{[P] 56}]
	$p$ s'écrit comme somme de deux carrés d'entiers si, et seulement si, $p=2$ ou $p \equiv 1\,[4]$.
\end{theorem}

\subsection*{C. Algorithme de chiffrement RSA}

\begin{algorithm}[\textnormal{[G] 37}]
	Alice veut envoyer à Bob un message représenté par un nombre 
entier $m$, en tout sécurité.
\begin{itemize}
	\item Bob choisit en secret deux nombres premiers distincts $p$ et $q$ et calcule leur produit $n=pq$.
	\item Il choisit ensuite un entier $c<\varphi(n)=(p-1)(q-1)$ premier à $\varphi(n)$.
	\item Il trouve ensuite un entier $d$ tel que $cd \equiv 1\,[\varphi(n)]$.
	\item La clé publique de Bob est $(n,c)$, qu'il donne à Alice, et sa clé privée est $(n,d)$, qu'il garde secrète.
	\item Alice envoie à Bob le message $m^c$ mod $n$.
	\item Pour décoder le message, Bob calcule $\left(m^c\right)^d\equiv m\,[n]$.
\end{itemize}
\end{algorithm}

\section*{Développements}
\begin{itemize}
	\item Développement 1 : Théorème \ref{120dev1} (restes chinois) et exemple \ref{120dev12}
	\item Développement 2 : Théorème \ref{120dev2} (cyclicité des inversibles de $\IZ/p^{\alpha}\IZ$)
\end{itemize}

\section*{Références}
\begin{itemize}
	\item[Rb] \emph{Mathématiques pour l'agrégation - Algèbre et géométrie}, Jean-Étienne Rombaldi, 2e édition
	\item[P] \emph{Cours d'algèbre}, Perrin
	\item[G] \emph{Les maths en tête - Algèbre et probabilités}, Xavier Gourdon, 3e édition
\end{itemize}

\chapter*{121 : Nombres premiers. Applications.}
\setcounter{definition}{0}
Pour un entier $n$, $\Div(n)$ désigne l'ensemble des diviseurs positifs de $n$.	

\section*{I. Résultats fondamentaux sur les nombres premiers}
\subsection*{A. Notion de nombre premier, propriétés élémentaires}

\begin{definition}[\textnormal{[R] 303}]
	On dit que $p\in\IN$ est \emph{premier} si $\Div(p)=\left\{1,p\right\}$.
	On dit que $n$ est \emph{composé} si $n\neq 0$ et si $\exists a\in\IN\setminus\left\{1,n\right\}\,\colon a\mid n$.
\end{definition}

Dans la suite, $\mathcal{P}$ désignera l'ensemble des nombres premiers.

\begin{lemma}[d'Euclide]
	$\forall(a,b)\in\IN^2$, $\forall p\in\mathcal{P}$, $p\mid ab \implies (p\mid a)$ ou $(p\mid b)$.
\end{lemma}

\begin{lemma}[\textnormal{[R] 303}]
	$\forall n\geq 2,\, \exists p\in\mathcal{P}\,\colon p\mid n$
\end{lemma}

\begin{proposition}[\textnormal{[R] 304}]
	Tout entier composé $n$ admet un facteur premier entre $2$ et $\sqrt{n}$.
\end{proposition}

\begin{theorem}[fondamental de l'Arithmétique - \textnormal{[R] 306}]
	$\forall n\in\IN^*,\,\exists ! \left(v_p(n)\right)_{p\in\mathcal{P}}\in\IN^{\mathcal{P}} \,\colon$ 
	$$n = \prod_{p\in\mathcal{P}} p^{v_p(n)}$$
	Cette écriture est appelée \emph{"(la) décomposition en produit de facteurs premieres de $n$"}.
\end{theorem}

\begin{definition}[\textnormal{[R] 306}]
	Dans la décomposition en produit de facteurs premiers de $n$, l'entier $v_p(n)$ ($p\in\mathcal{P}$) est appelé \emph{valuation $p$-adique de $n$}.
\end{definition}

\begin{proposition}[\textnormal{[R] 307}]
	$\forall(a,b)\in\left(\IN^*\right)^2,\, a\mid b\iff \forall p\in\mathcal{P},\, v_p(a)\leq v_p(b)$
\end{proposition}

\begin{proposition}[\textnormal{[R] 319}]
	$\forall(a,b)\in\left(\IN^*\right)^2,\, v_p(ab) = v_p(a)+v_p(b)$
\end{proposition}

\begin{proposition}[\textnormal{[R] 307}]
	$\forall(a,b)\in\left(\IN^*\right)^2,\,\forall p\in\mathcal{P},$
	\begin{align*}
		v_p(a\vee b) &= \max (v_p(a),v_p(b)) \\
		v_p(a\wedge b) &= \min (v_p(a),v_p(b))
	\end{align*}
\end{proposition}

\subsection*{B. Répartition des nombres premiers}
\begin{theorem}[Euclide - \textnormal{[R] 305}]
	Il existe une infinité de nombres premiers.
\end{theorem}

\begin{theorem}[de la progression arithmétique, Dirichlet, ADMIS]
	Pour tout $(a,b)\in\left(\IN^*\right)^2$ tel que $a\wedge b = 1$, il existe une infinité de nombres premiers congrus à $a$ modulo $b$.
\end{theorem}

\begin{conjecture}[des nombres premiers jumaux]
	Il existe une infinité de nombres premiers $p$ tels que $p+2$ est premier.
\end{conjecture}

\begin{proposition}
	Il existe des intervalles de longueur arbitrairement grande ne contenant aucun nombre premier.
\end{proposition}

\begin{theorem}[Bertrand - ADMIS - \textnormal{[R] 325}]
	Il existe toujours un nombre premier compris entre n'importe quel entier naturel non nul et son double.
\end{theorem}

\begin{theorem}[des nombres premiers - ADMIS - \textnormal{[R] 308}]
	$$\#\mathcal{P}\cap \llbracket 1,n\rrbracket \sim_{x\to +\infty} \frac{n}{\ln n}$$
\end{theorem}

\section*{II. Tests de primalité}

\begin{proposition}[Crible d'Ératosthène - ANNEXE]
	Le procédé suivant permet de trouver la liste croissante des nombres premiers : on part de la liste des entiers plus grands que $2$. À chaque itération,
	on garde le plus petit nombre, et on supprime tous ses multiples.
\end{proposition}

\begin{proposition}
	$n$ est premier si, et seulement si, $\forall d\leq \lfloor\sqrt{n}\rfloor$, $d\nmid n$.
	La complexité au pire de ce test est donc en $O(\sqrt{n})$.
\end{proposition}

\begin{theorem}[de Fermat]
	Si $p$ est premier, alors $\forall a\in \IN$, $a\wedge p=1\implies a^{p-1} \equiv 1\,[p]$.
\end{theorem}

\begin{remark}
	On en déduit donc un test de non primalité.
\end{remark}

\begin{definition}[\textnormal{[R] 329}]
	Un nombre $n$ composé satisfaisant le test du théorème de Fermat est appelé \emph{nombre de Carmichaël}.
\end{definition}

\begin{example}[\textnormal{[R] 329}]
	$561$ est un nombre de Carmichaël.
\end{example}

\begin{theorem}[de Korselt - \textnormal{[R] 330}]
	$n$ est un nombre de Carmichaël si, et seulement si, pour tout diviseur premier $p$ de $n$, $(p-1)\mid(n-1)$ et $p^2\nmid n$.
\end{theorem}

\begin{theorem}[de Wilson - \textnormal{[R] 326}]
	$n$ est premier si, et seulement si, $(n-1)!\equiv -1\,[n]$.
	C'est un test de primalité qui requiert $n-1$ multiplications dans $\IZnZ$.
\end{theorem}

\section*{III. Applications des nombres premiers}
\subsection*{A. Fonctions spéciales}

\begin{definition}[\textnormal{[R] 283}]
	\emph{L'indicatrice d'Euler} est :
	$\varphi\,\colon n \mapsto \#\left(\IZnZ\right)^{\times} = \#\left\{k\in\llbracket 1,n\rrbracket \mid k\wedge n = 1\right\}$.
\end{definition}

\begin{proposition}[\textnormal{[R] 288}]
	$\forall(a,b)\in\left(\IN^*\right)^2$, $a\wedge b=1$, alors $\varphi(ab)=\varphi(a)\varphi(b)$.
	Pour tout $\alpha \in \IN^*$, $\varphi(p^{\alpha}) = p^{\alpha - 1}(p-1)$.
\end{proposition}

\begin{corollary}[\textnormal{[R] 288}]
	$\forall n\in\IN^*$,
	$$\varphi(n) = \prod_{\substack{p\in\mathcal{P}\\v_p(n)\geq 1}} p^{v_p(n)-1}(p-1) = n\prod_{\substack{p\in\mathcal{P}\\v_p(n)\geq 1}} \left(1 - \frac{1}{p}\right)$$	
\end{corollary}

\begin{definition}
	La \emph{fonction $\zeta$ de Riemann} est définie par :
	\begin{align*}
		\zeta\,\colon \left\{z\in\IC\mid \Re(z)> 1\right\} &\to \IC \\
		s &\mapsto \sum_{n=0}^{+\infty} \frac{1}{n^s}
	\end{align*}
\end{definition}

\begin{proposition}[\textnormal{[KG] 461}]
	On a :
	$$\zeta(s) = \prod_{p\in\mathcal{P}}\frac{1}{1 - \frac{1}{p^s}}$$
	Cette écriture est appelé \emph{"produit eulérien"}.
\end{proposition}

\begin{theorem}[\textnormal{[KG] 461, [R] 343}]
	$\sum_{p\in\mathcal{P}}\frac{1}{p} = +\infty$
\end{theorem}

\begin{definition}[\textnormal{[R] 331}]
	La \emph{fonction de Moëbius} est définie par :
	\begin{align*}
		\mu\,\colon n\in\IN^* \mapsto \begin{cases}
			1\quad \text{si } n =1 \\
			(-1)^r\quad \text{si } n=p_1\dots p_r,\text{ avec $p_1,\dots,p_r$ distincts} \\
			0\quad\text{sinon}
		\end{cases}
	\end{align*}
\end{definition}

\begin{theorem}[Cesàro - ADMIS \textnormal{[R] 334}]
	La probabilité de choisir au hasard $r\geq 2$ entiers entre $1$ et $n$ qui sont premiers entre eux vaut $\frac{1}{\zeta(r)}$.
\end{theorem}

\subsection*{B. Algorithme de chiffrement RSA}

\begin{theorem}[d'Euler - \textnormal{[R] 283}]
	$\forall (a,b)\in\left(\IN^*\right)^2$, si $a\wedge n = 1$, alors $a^{\varphi(n)} \equiv 1\,[n]$.
\end{theorem}

De la complexité des tests de primalité découle la grande difficulté de la recherche de la décomposition en produit de facteurs premiers d'un entier donné.
Ce principe est à la base de la sécurité de l'algorithme de chiffrement RSA, détaillé ci-dessous :

\begin{algorithm}[\textnormal{[G] 37}]
	Alice veut envoyer à Bob un message représenté par un nombre 
entier $m$, en tout sécurité.
\begin{itemize}
	\item Bob choisit en secret deux nombres premiers distincts $p$ et $q$ et calcule leur produit $n=pq$.
	\item Il choisit ensuite un entier $c<\varphi(n)=(p-1)(q-1)$ premier à $\varphi(n)$.
	\item Il trouve ensuite un entier $d$ tel que $cd \equiv 1\,[\varphi(n)]$.
	\item La clé publique de Bob est $(n,c)$, qu'il donne à Alice, et sa clé privée est $(n,d)$, qu'il garde secrète.
	\item Alice envoie à Bob le message $m^c$ mod $n$.
	\item Pour décoder le message, Bob calcule $\left(m^c\right)^d\equiv m\,[n]$.
\end{itemize}
\end{algorithm}

\subsection*{C. Corps finis}
\begin{definition}[\textnormal{[R] 415}]
	La \emph{caractéristique} d'un anneau $A$ est l'unique générateur positif du noyau du morphisme $\varphi\,\colon\IZ\to A$, $n\mapsto n1_A$.
\end{definition}

\begin{lemma}[\textnormal{[R] 415}]
	La caractéristique d'un corps est nulle ou première.
\end{lemma}

\begin{example}
	$\IZpZ$ est un corps de caractéristique $p$.
\end{example}

\begin{theorem}[\textnormal{[R] 421}]
	Il existe un corps fini de cardinal $q$ si, et seulement si, $q$ est une puissance d'un nombre premier. 
	Le cas échéant, un tel corps est unique à isomorphisme près, et on note $\IF_q$ le corps fini à $q$ éléments.
	Par ailleurs, $p = \car \IF_q$ est un nombre premier, et $q$ est une puissance de $p$.
\end{theorem}

\subsection*{D. Le théorème des deux carrés de Fermat}

\begin{tcolorbox}[
    breakable, % Allows the theorem to split across pages
    colback=developpement, % The background color
    colframe=gray!0!black, % The frame color
    boxrule=0pt, % The frame thickness
    arc=1mm, % Sharp corners
	boxsep=0pt,
	left=0pt, right=0pt, top=0pt, bottom=0pt
]
\begin{lemma}[\textnormal{[P] 75}]
	\label{121dev11}
	$-1$ est un carré dans $\IF_p$ si, et seulement si, $p\equiv 1\,[4]$.
\end{lemma}

\begin{theorem}[des deux carrés de Fermat - \textnormal{[P] 56}]
	\label{121dev12}
	Soit $E = \left\{n\in\IN^* \mid \exists (a,b)\in\IN^2\,\colon n = a^2 + b^2 \right\}$
	Alors, $n\in E \iff \forall p\in\mathcal{P},\, p\equiv 3\,[4] \implies v_p(n)$ est pair.
\end{theorem}
\end{tcolorbox}

\subsection*{E. En théorie des groupes}
\begin{definition}[\textnormal{[R] 22}]
	Un $p$-groupe est un groupe de cardinal une puissance de $p$.
\end{definition}

\begin{proposition}[\textnormal{[R] 22}]
	Si un $p$-groupe $G$ agit sur un ensemble fini $X$, alors $\# X \equiv \#X^G\,[p]$
	où $X^G$ est l'ensemble des éléments de $X$ fixes par l'action de $G$.
\end{proposition}

\begin{corollary}[\textnormal{[R] 23}]
	Le centre d'un $p$-groupe n'est pas trivial.
\end{corollary}

\begin{definition}[\textnormal{[U] 85}]
	Soit $G$ un groupe fini de cardinal $p^{\alpha}m$, $m\wedge p=1$. Un $p$-Sylow de $G$
	est un sous-$p$-groupe de $G$ de cardinal $p^{\alpha}$.
\end{definition}

\begin{theorem}[de Sylow - ADMIS \textnormal{[U] 87}]Soit $G$ un groupe d'ordre $p^{\alpha}m$, $m\wedge p = 1$. Alors,
	\begin{enumerate}
		\item $\Syl_p(G)\neq \empty$
		\item $G$ agit transitivement sur $\Syl_p(G)$ par conjugaison
		\item $n_p\equiv 1\, [p]$
	\end{enumerate}
\end{theorem}

\begin{tcolorbox}[
    breakable, % Allows the theorem to split across pages
    colback=developpement, % The background color
    colframe=gray!0!black, % The frame color
    boxrule=0pt, % The frame thickness
    arc=1mm, % Sharp corners
	boxsep=0pt,
	left=0pt, right=0pt, top=0pt, bottom=0pt
]
\begin{theorem}[\textnormal{[R] 292}]
	\label{121dev2}
	Si $p\geq 3$, alors $\forall \alpha \geq 1$, $\left(\IZ/p^{\alpha}\IZ\right)^{\times}$ est cyclique.
\end{theorem}
\end{tcolorbox}

\begin{proposition}[\textnormal{[R] 23}]
	Tout groupe d'ordre $p^2$ est abélien.
\end{proposition}

\section*{Développements}
\begin{itemize}
	\item Développement 1 : Lemme \ref{121dev11}, et théorème \ref{121dev12}
	\item Développement 2 : Théorème \ref{121dev2} (cyclicité des inversibles de $\IZ/p^{\alpha}\IZ$)
\end{itemize}

\section*{Références}
\begin{itemize}
	\item[Rb] \emph{Mathématiques pour l'agrégation - Algèbre et géométrie}, Jean-Étienne Rombaldi, 2e édition
	\item[U] \emph{Théorie des groupes}, Félix Ulmer
	\item[G] \emph{Les maths en tête - Algèbre et probabilités}, Xavier Gourdon, 3e édition
	\item[KG] \emph{De l'intégration aux probabilités}, Olivier Garet, Aline Kurtzmann, 2e édition augmentée
\end{itemize}
\begin{figure}[!htb]
	\centering
	\includegraphics[trim={0 0 0 0},clip,width=1\linewidth]{img/eratosthene.pdf}
	\caption{Crible d'Eratosthène}
\end{figure}
\end{document}
